% ---------------
% PREAMBLE
% ---------------
\newif\iflatextortf

\iflatextortf
	% tell latex2tortf if this is an article or report
 	\documentclass[12pt,letterpaper]{article}
	\input{NRELLatex2rtf.tex}
\else
	%\documentclass[report,tagged]{nrel}
	\documentclass[report]{nrel}
\fi

% -----------------------------------
% DOCUMENT PROPERTIES
% -----------------------------------
\title{SAM Photovoltaic Model Technical Reference}
\author{Paul Gilman, Aron Dobos, Nicholas DiOrio, Janine Freeman, Steven Janzou, David Ryberg}


% -----------------------------------
% DOCUMENT PROPERTIES
% -----------------------------------
\usepackage{array}
\usepackage{dcolumn}
\newcolumntype{.}{D{.}{.}{2.7}}

% -----------------------------------
% MULTI-LETTER VARIABLE NAMES
% -----------------------------------
\newcommand\GCR{\ensuremath{\mathit{GCR}}}
\newcommand\AOI{\ensuremath{\mathit{AOI}}}

% -----------------------------------
% DIRECTORIES
% -----------------------------------
\graphicspath{{figures/}}

% -----------------------------------
% START DOCUMENT
% -----------------------------------
\begin{document}

%%%%%%%%%%%%%%%%%%%%%%%%%%%%%%%%%%%%%%%%%%%%%%%%%%
%%%%%%%%%%%%%%%%%%%%%%%%%%%%%%%%%%%%%%%%%%%%%%%%%%
\frontmatter
\chapter*{Executive Summary}

This manual describes the photovoltaic performance model in the System Advisor Model (SAM). The U.S. Department of Energy's National Renewable Energy Laboratory maintains and distributes SAM, which is available as a free download from \url{https://sam.nrel.gov}. These descriptions are based on SAM 2015.1.30 (SSC 41).

SAM is a techno-economic feasibility model for renewable energy projects. It is designed for a range of different users, including project developers, system designers, policy makers, financial planners, and academic researchers.

SAM's photovoltaic performance model is available both as part of the SAM desktop application, and in the SAM software development kit (SDK). This manual is intended for people who want to understand SAM's photovoltaic model, or for people who are using the SDK to develop their own applications.

SAM runs on Windows and OS X operating systems, and is a user interface that performs the following functions:

\begin{itemize}
\item Organizes and displays the performance and financial model inputs in a user-friendly interface.
\item Manages tasks associated with running model simulations.
\item Provides options for ``advanced" simulations that involve multiple simulation runs for parametric and sensitivity studies.
\item Stores arrays of model results.
\item Calculates secondary results such as monthly and annual totals, capacity factor, system performance factor, and system losses.
\item Displays tables and graphs of results.
\item Allows for exporting data in different formats, CSV, graph images, Microsoft Excel, and a PDF report.
\end{itemize}

The SAM SDK is a package containing the SAM Simulation Core (SSC) libraries and a set of software development tools that allow model developers to create their own interfaces to the simulation modules as either web or desktop applications.

%%%%%%%%%%%%%%%%%%%%%%%%%%%%%%%%%%%%%%%%%%%%%%%%%%
%%%%%%%%%%%%%%%%%%%%%%%%%%%%%%%%%%%%%%%%%%%%%%%%%%
\mainmatter
\tableofcontents
\listoffigures
\listoftables

%%%%%%%%%%%%%%%%%%%%%%%%%%%%%%%%%%%%%%%%%%%%%%%%%%
%%%%%%%%%%%%%%%%%%%%%%%%%%%%%%%%%%%%%%%%%%%%%%%%%%
\chapter{Nomenclature}

The organization of this manual is based roughly on the Sandia National Laboratories PV Performance Modeling Collaborative (PVPMC) website ``Modeling Steps" \citep{pvpmc}. The nomenclature and general descriptions also draw from the ``PVCDROM" electronic document on the pveducation.org website hosted by the Arizona State University Solar Power Labs \citep{pvcdrom}.

The variable names in this manual are listed in tables at the beginning of each section. In some cases, the same variable name might be used in two different sections of the manual to represent two different quantities. That is because we have tried to preserve the variable names used in the original sources. For example, $E$ is used to represent solar irradiance in most equations, and for the band-gap energy in two equations in the Sandia module model section.

For solar irradiance values, the letter $E$ indicates data from the weather file, $I$ indicates irradiance incident on the photovoltaic array before soiling and shading, and $G$ indicates effective irradiance incident on the array (at the top of the module cover) after soiling and shading. The subscripts $b$, $d$, and $g$ indicate beam, diffuse, and global irradiance values.

The variable $P$ indicates an electrical power value in Watts or kilowatts.

%%%%%%%%%%%%%%%%%%%%%%%%%%%%%%%%%%%%%%%%%%%%%%%%%%
%%%%%%%%%%%%%%%%%%%%%%%%%%%%%%%%%%%%%%%%%%%%%%%%%%
\chapter{Photovoltaic Performance Model Overview}

SAM's photovoltaic performance model combines module and inverter models (see Table~\ref{tab-submodels}) with supplementary code to calculate a photovoltaic power system's AC output over one year given a weather file and data describing the physical characteristics of the module, inverter, and array. The main models are listed in Table~\ref{tab-submodels} with a citation of the publication that originally described the modeling approach or an indication that the model uses standard equations, or was developed by the National Renewable Energy Laboratory (NREL), in which case the description in this manual is the only documentation of the modeling approach.

\begin{table}
\begin{center}
\caption{The SAM Photovoltaic Performance Model Modules}
\begin{tabular}{ll}
\midrule
Module & Reference\\
\midrule
Weather file reader & NREL\\
Sun position & \citet{michalsky1988}, \citet{iqbal1983}, NREL\\
Surface angles & standard geometry\\
Backtracking for one-axis trackers & NREL\\
Isotropic incident irradiance model & \citet{liu1963}\\
HDKR incident irradiance model & \citet{duffie2013}, \citet{reindl1988}\\
Perez 1990 incident irradiance model & \citet{perez1988}, \citet{perez1990}\\
Self shading model for fixed arrays & \citet{deline2013a}\\
Self shading model for one-axis trackers & NREL \\
Sandia module model & \citet{king2004}\\
CEC module model & \citet{desoto2004a}\\
Simple efficiency module model & NREL\\
Subarray mismatch calculator & NREL \\
Sandia inverter model & \citet{king2007}\\
Part load inverter model & NREL\\
\hline
\end{tabular}
\label{tab-submodels}
\end{center}
\end{table}

The photovoltaic performance model can simulate any size of system, from a small rooftop array and a single inverter to a large system with multiple subarrays and banks of inverters.

The model calculates the system's AC electrical output over one year. It reads solar resource and temperature data from a weather file describing the resource at the system's location for the year, and uses them with inputs describing the system's design in equations to calculate module and inverter conversion efficiencies and energy losses.

The modeled system must consist of a single type of photovoltaic module and a single type of inverter -- it cannot combine different sizes or brands of modules and inverters. The array may consist of up to four subarrays, each with its own set of parameters for tracking, surface angles, shading and soiling, and DC losses. Each subarray can have a different number of modules, but all subarrays must have the same number of modules per string so that all subarrays have the same nominal DC voltage, which serves as the inverter's nominal input voltage.

The array must be connected either to a single inverter or to a bank of inverters connected to each other in parallel. It is not possible to model a system with subarrays connected to different inverters.

The module model (Section~\ref{sec-module}) and inverter model (Section~\ref{sec-inverter}) calculate solar-energy-to-DC-electricity and DC-to-AC electricity conversion efficiencies, respectively, and account for losses associated with each component. The self-shading models (Section~\ref{sec-selfshad}) calculate losses caused by shading of modules in the array by neighboring modules. The photovoltaic performance model does not explicitly calculate the remaining system losses. They are represented by user-specified inputs:

\begin{itemize}
\item{Beam and diffuse shading losses for near-object shading of the array (Section~\ref{sec-nearobjectshad}). These can be specified for each hourly or subhourly time step, month-by-hour (288 values), or sun azimuth angle by elevation (number of values varies), and may be generated by shading analysis equipment and software.}
\item{Monthly soiling losses for dust and other accumulation on the array (Section~\ref{sec-soiling}).}
\item{DC losses for module mismatch, DC wiring and connections, tracking, and other losses associated with the array (Section~\ref{sec-dclosses}).}
\item{AC losses for AC wiring and transformer losses (Section~\ref{sec-aclosses}).}
\end{itemize}

The photovoltaic model uses the same algorithm to calculate self-shading losses for fixed arrays and one-axis trackers (Section~\ref{sec-selfshad}). It does not calculate self-shading losses for two-axis or azimuth-axis trackers. The self-shading algorithm calculates a reduction in diffuse POA irradiance and a DC loss factor to account for the performance impact of the reduction in beam POA irradiance.

The model does not calculate module mismatch losses within a subarray. For systems with more than one subarray, an optional algorithm can estimate mismatch losses between the subarrays (Section~\ref{sec-mismatch}).

\section{Model Algorithm}

This section describes the basic algorithm of SAM's photovoltaic performance model. The details of each step listed below are described in the sections that follow. See Figure~\ref{fig-pvsamschematic} for a basic block diagram of the model. Note that the block diagram does not include the subarray mismatch and string voltage calculations described in the steps below to make the diagram easier to follow.

\begin{figure}
\begin{center}
\includegraphics[scale=0.88]{pvsam-schematic}
\caption{Photovoltaic Performance Model Simplified Block Diagram}
\label{fig-pvsamschematic}
\end{center}
\end{figure}
 
The simulation model performs the following calculations for each time step in one year:

\begin{enumerate}

\item{For each of up to four subarrays:}

  \begin{enumerate}

  \item{Calculate sun angles from date, time, and geographic position data from the weather file. (Section~\ref{sec-sunangles})} %irrad.calc() called from cmod_pvsamv1.cpp 1213 from inside subarray loop. solarpos() called in irrad.calc() lib_irradproc.cpp 855 etc

  \item{Calculate the nominal beam and diffuse irradiance incident on the plane of array (POA irradiance). This depends on the solar irradiance data in the weather file, sun angle calculations, user-specified subarray parameters such as tracking and orientation parameters, and backtracking option for one-axis trackers. (Section~\ref{sec-nominalincidentirradiance})} %irrad.calc() called as above. incidence() called in irrad.calc() lib_irradproc.cpp 900 etc

  \item{Apply the user-specified beam and diffuse near-object shading factors to the nominal POA irradiance. (Section~\ref{sec-nearobjectshad})} %cmod_pvsamv1.cpp 1235-1249

  \item{For subarrays with one-axis tracking and self-shading enabled, calculate and apply the self-shading loss factors to the nominal POA beam and diffuse irradiance. (Section~\ref{sec-selfshad})} %cmod_pvsamv1.cpp 1254 call shade_fraction_1x() in lib_irradproc.cpp for beam shading loss, 1259-1318 for sky and ground diffuse factors

  \item{Apply user-specified monthly soiling factors to calculate the effective POA irradiance on the subarray. (Section~\ref{sec-soiling})} %cmod_pvsamv1.cpp 1328

  %\item{Record subarray outputs.} %cmod_pvsamv1.cpp 1338

  \end{enumerate}

\item{If there is a single subarray (Subarray 1) with no tracking (fixed) and self-shading is enabled, calculate the reduced diffuse POA irradiance and self-shading DC loss factor. (Section~\ref{sec-selfshadalg})} %sscalc.exec() called in cmod_pvsamv1 1378

\item{Determine subarray string voltage calculation method (Section~\ref{sec-dcstringvoltage}).} \label{item-mismatch} %cmod_pvsamv1.cpp 1403

\item{For each of up to four subarrays, run the module model with the effective beam and diffuse POA irradiance and module parameters as input to calculate the DC output power, module efficiency, DC voltage, and cell temperature of a single module in the subarray.} %call pvinput_t in() and out() from lib_pvmodel.cpp to set inputs and outputs cmod_pvsamv1.cpp 1459, then run module model 1471

\item{Calculate the subarray string voltage using the method determined in Step \ref{item-mismatch}.} %lib_pvsamv1.cpp 1486

\item{Loop through the subarrays to calculate the array DC power (Section~\ref{sec-arraydcoutput}):} %lib_pvsamv1.cpp 1491

  \begin{enumerate}

  \item{For Subarray 1, apply the fixed self-shading DC loss to the module DC power if it applies.} %lib_pvsamv1.cpp 1498

  \item{For each subarray, calculate the subarray gross DC power by multiplying the module DC power by the number of modules in the subarray.} %lib_pvsamv1.cpp 1502

  \item{For each subarray, calculate subarray DC power by multiplying the gross subarray power by the DC loss.} %lib_pvsamv1.cpp 1505

  \item{For each subarray, calculate the subarray string voltage by multiplying the module voltage by the number of modules per string.} %lib_pvsamv1.cpp 1511

  \item{Calculate the array DC power by adding up the subarray values.}

  \end{enumerate}

\item{Run the inverter model to calculate the gross AC power and inverter conversion efficiency (Section~\ref{sec-inverter}).} %lib_pvsamv1.cpp 1515

%\item{Calculate irradiance values for results} %lib_pvsamv1.cpp 1535

\item{Calculate the AC power by applying the AC loss to the gross AC power. (Section~\ref{sec-netacoutput})} %lib_pvsamv1.cpp 1570

\end{enumerate}

\section{Equipment Libraries and Weather Files} \label{sec-libraries}

SAM comes with libraries that store module and inverter parameters representing the equipment's physical properties. The solar resource library stores weather files with data representing the solar resource at different locations. The data in these libraries are copies of data managed by different organizations.

The following is a list of module and inverter libraries used by SAM's photovoltaic performance model with the source of data each library contains:
\begin{itemize}
\item \textbf{CEC Modules:} California Energy Commission Eligible Photovoltaic Modules \citep{gsc2014b}
\item \textbf{CEC Inverters:} California Energy Commission Eligible Inverters \citep{gsc2014c}
\item \textbf{Sandia Modules:} Sandia National Laboratories Module Database \citep{sandia-testeval}
\end{itemize}

SAM's solar resource library contains data from the following sources:
\begin{itemize}
\item NREL National Solar Radiation Database 1961-1990 (TMY2) \citep{nsrdb}
\item NREL National Solar Radiation Database 1991-2010 Update (TMY3) \citep{nsrdb}
\item U.S. DOE EnergyPlus Weather Data \citep{epw}
\item NREL Solar Prospector \citep{solarprospector}
\end{itemize}

Note that these data collections are not part of SSC. If you are developing a model using the SAM software development kit and want to use parmaeters from the libraries, you will have to write your own code to read the data. The module and inverter libraries are text files that use a simple comma-separated format. The solar resource library contains weather data files in the SAM CSV format \citep{help-weatherfileformats}.

\section{System Sizing} \label{sec-sizing}

SAM's System Design input page has an option that automatically calculates the number of modules per string, number of parallel strings in the array, and number of inverters given a desired array size in DC kW and a DC-to-AC nameplate capacity ratio. The sizing algorithm is described in \citet{help-sizing} and uses the module and inverter reference parameters for the capacity calculations.

The system sizing algorithm is not available in SSC.

%%%%%%%%%%%%%%%%%%%%%%%%%%%%%%%%%%%%%%%%%%%%%%%%%%
%%%%%%%%%%%%%%%%%%%%%%%%%%%%%%%%%%%%%%%%%%%%%%%%%%
\chapter{Irradiance and Weather Data}\label{sec-irradianceweatherdata}

The photovoltaic performance model requires a weather file with solar resource data for one year and information about the location of the measurement site. The data may be hourly, or subhourly with a time resolution of up to one minute. See Table~\ref{tab-wfdata} for a list of weather data fields.

In SSC, you can provide either the path to a weather file (\texttt{solar\_resource\_file}, or a table containing the weather data (\texttt{solar\_resource\_data}). 

The model uses data from the weather file to calculate the following values for each time step in the year:

\begin{itemize}
\item Sun position angles (Section~\ref{sec-sunposition}).
\item Plane-of-array (POA) irradiance (Section~\ref{sec-incidentirradiance}).
\item Photovoltaic cell temperature (Section~\ref{sec-celltempoptions}).
\item Array DC loss due to snow cover (Section~\ref{sec-snow}).
\end{itemize}

Solar resource data must include solar irradiance, wind speed, and ambient temperature data, and may include optional snow depth and albedo data. Location data must include latitude, longitude, time zone and elevation, and may include optional fields for location name or number, city, state, region, and country. 

As explained in Section~\ref{sec-incidentirradiance}, by default, SAM calculates the irradiance incident on the subarray from the beam (DNI) and diffuse (DHI) components of the solar irradiance. For weather files that do not contain data from those components, SAM can estimate the DNI and DHI from the other irradiance components. You must choose the appropriate \textbf{Weather File Irradiance Data} option on the Location and Resource input page (\texttt{irrad\_mode} value in SSC):

\begin{itemize} 
\item DNI and DHI (\texttt{irrad\_mode}~=~0) is the default option: SAM uses the beam (direct normal) an diffuse horizontal irradiance from the weather file, and ignores the GHI (global horizontal irradiance) data if it exists.
\item DNI and GHI (\texttt{irrad\_mode}~=~1): SAM uses the beam and global horizontal data from the weather file and calculates the DHI using Equation~\ref{eqn-diffhoriz}.
\item GHI and DHI (\texttt{irrad\_mode}~=~2): SAM uses the global horizontal and diffuse horizontal data from the weather file and calculates the DNI using Equation~\ref{eqn-beam}.
\item Plane of array (POA) (\texttt{irrad\_mode}~=~3 or 4): SAM uses the POA data in the weather file and bypasses the incident irradiance calculations, unless shading is enabled for the subarray as described in  Section~\ref{sec-poa}.
\end{itemize}

Each row in the weather file must have a Year, Month, Day, and Hour column indicating the time for sun position calculations. Sub-hourly data must have an additional Minute column. For hourly data, the Minute column is optional, and may be used to indicate the minute for sun position calculations (Section~\ref{sec-timeconvention}). 

SAM can read files in the formats listed below. For descriptions of the formats see \citep{help-weatherfileformats}.

\begin{itemize}
\item SAM CSV: Comma-separated text format developed by NREL for use with SAM. This is also the file format for files from the NREL National Solar Radiation Database.
\item TMY3: Comma-separated text format originally developed by NREL for the National Solar Radiation Database 1991-2010 update.
\item TMY2: Text format originally developed by NREL for the National Solar Radiation Database 1961 - 1990 data set.
\item EPW: Text format derived from the TMY2 format for the EnergyPlus building modeling software. 
\end{itemize}

SAM's photovoltaic model reads and stores the weather file data shown in Table~\ref{tab-wfdata}. It ignores any additional data elements that may be included in the different weather file formats. Some additional data elements, such as relative humidity, are required for SAM's concentrating solar power performance models, but not for the photovoltaic model. SAM also ignores supporting data, such as the source and uncertainty flags in the TMY3 file format.

\begin{table}
\begin{center}
\caption{Weather Data}
\begin{tabular}{lll}
\midrule
Symbol & Field & Description\\
\midrule
\multicolumn{3}{c}{Metadata Location Description}\\
- & Location ID* & numerical identifier [\textit{722780}]\\
 - & City* & location name [\textit{Phoenix Sky Harbor Intl Ap}]\\
- & State* & two-letter state abbreviation [\textit{AZ}]\\
$\mathit{tz}$ &Time zone & hours E of GMT [\textit{-7.0}] \\
$\mathit{lat}$ & Latitude & decimal degree N of 0 [\textit{33.450}] \\
$\mathit{lon}$ & Longitude & decimal degree E of 0 [\textit{-11.983}]\\
$\mathit{elv}$ & Elevation & meters above sea level [\textit{337}]\\
\midrule
\multicolumn{3}{c}{Time Series Data Records}\\
$\mathit{yr}$& Year & typical year [\textit{1988}] \\
$\mathit{mo}$ & Month & typical month (1-12) \\
$\mathit{day}$ & Day & day of year (1-365) \\
$\mathit{hr}$ & Hour & hour of day (0-23) in local time \\
$\mathit{min}$ & Minute & minute of hour (0-59) \\
$E_g$ & Global horizontal irradiance (W/m$^2$) & total radiation on a horizontal surface \\
$E_b$ & Direct normal irradiance (W/m$^2$) & direct radiation on a surface normal to the sun \\
$E_d$ & Diffuse horizontal irradiance (W/m$^2$) & radiation on a horizontal surface from the sky \\
POA & Plane-of-array irradiance & radiation on the plane of the array\\
$v_{wind}$ & Wind speed (m/s) & wind speed \\
$T_a$ & Dry-bulb temperature ($^{\circ}$C) & ambient dry-bulb temperature \\
$D_{snow}$ & Snow depth (cm)$\ast$ & depth of snow \\
$\mathit{\rho}$ & Ground reflectance$\dagger$  & ground reflectance factor or albedo \\
- & Dew-point temperature ($^{\circ}$C)$\ddagger$  & dew-point temperature \\
- & Atmospheric pressure (mbar)$\ddagger$  & atmospheric pressure \\
\midrule
\multicolumn{3}{l}{\textit{Table Notes}}\\
\multicolumn{3}{l}{The italicized metadata values in brackets are sample values.}\\
\multicolumn{3}{l}{SAM requires either any two irradiance components or POA data.}\\
\multicolumn{3}{l}{$\ast$ Snow depth data is required to model snow losses (Section~\ref{sec-snow}).}\\
\multicolumn{3}{l}{$\dagger$ When albedo data is missing, SAM applies a default value of 0.2 (Section~\ref{sec-groundreflected}).}\\
\multicolumn{3}{l}{$\ddagger$ Required only for the heat transfer cell temperature model (Section~\ref{sec-tcheattransfer}).}\\
\end{tabular}
\label{tab-wfdata}
\end{center}
\end{table}

\section{Time Period and Resolution}\label{sec-timeconvention}

The weather file contains one year of time series solar resource data. The photovoltaic performance model can run simulations with up to one-minute time steps. The time between time steps may be between one and 60 minutes, as long as 8760 minutes per hour is a multiple of the time step length in minutes. For example, the model will run with a weather file that contains one-minute data, 10-minute data, 15-minute data, or 60-minute (hourly) data.

The number of data rows in the file determines the simulation time step. For example, a file with 8760 rows of weather data (not including header rows) would result in an hourly simulation. A weather file for a 15-minute simulation time step should have 35,040 data rows in addition to the header rows. A one-minute data file would have 525,600 data rows.

The first data row in the weather file is for the first time step after midnight on January 1, local standard time. For hourly data, the first row is for the hour ending at 1 a.m. on January 1. For 15-minute data, the first row is for the quarter hour ending at 1:15 a.m. on January 1.

By default for hourly weather data, SAM calculates the sun position (and photovoltaic array surface angles) for the mid-point of the time step. For example, the sun position for the 10:00 a.m. hour would be its position at 10:30 a.m. However, if the weather file contains hourly data and includes a Minute column, SAM instead uses the minute indicated to calculate the sun position for each hour. Sub-hourly weather data must include a Minute column indicating the for sun postion calculations.

\section{Irradiance}

%% Also address questions about spectrum and field of view

Solar irradiance is a measure of the instantaneous power from the sun on a surface over an area, typically given in the SI units of watts per square meter (W/m$^2$). In the weather files from the National Solar Radiation Database \citep{nsrdb}, each irradiance value is the total solar radiation in the 60 minutes ending at a given time step. These values represent the average solar power over a given hour in watt-hours per hour per square meter (Wh/h$\cdot$m$^2$). In SAM these values are expressed in the mathematically equivalent W/m$^2$ (or converted to kW/m$^2$).

The weather file stores time series values for the three components of solar irradiance:

\begin{itemize}
\item Global horizontal irradiance (GHI): The total solar irradiance on a surface parallel to the ground (horizontal).
\item Direct normal irradiance (DNI): The portion of the solar irradiance that reaches a surface normal to the sun in a direct line from the solar disk (typically assuming a measurement device with a $5^{\circ}$ field of view), also called beam normal.
\item Diffuse horizontal irradiance (DHI): The solar irradiance on a horizontal surface from the sky excluding the solar disc.
\end{itemize}

The photovoltaic performance model requires direct normal and diffuse horizontal irradiance data to calculate the plane-of-array irradiance (Section~\ref{sec-incidentirradiance}). However, it includes functions to estimate either the direct normal (see Equation \ref{eqn-beam}) or diffuse horizontal (see Equation \ref{eqn-diffhoriz}) irradiance when the weather file is missing those data elements. To use a weather file that is missing either the DNI or DHI data, you must choose the appropriate \textbf{Advanced} options on the Location and Resource input page (\texttt{irrad\_mode} in SSC).

Weather files from the NREL National Solar Radiation Database include all three components of the solar irradiance: Direct normal, diffuse horizontal, and global horizontal. However, the photovoltaic performance model will recognize a file that contains any of the following combinations of irradiance data:

\begin{itemize}
\item Direct normal and diffuse horizontal.
\item Direct normal and global horizontal.
\item Global horizontal and diffuse horizontal.
\item Total plane-of-array irradiance measured by either a reference cell or pyranometer.
\end{itemize}

\section{Weather Observations}

The photovoltaic performance model uses ambient temperature and wind speed data from the weather file to estimate the effect of photovoltaic cell temperature on the array's performance. Although wind direction does have an effect on cell temperature, the photovoltaic performance model assumes that its cumulative effect over the year is negligible \citep{king2004}.

The air temperature or dry-bulb temperature is the temperature in degrees Celsius (\degree C) of ambient air measured by a thermometer exposed to the air but shielded from the sun and rain. 

SAM assumes that the wind speed measurements from the weather file are in m/s and measured at a height of 10 m above the ground.

%%%%%%%%%%%%%%%%%%%%%%%%%%%%%%%%%%%%%%%%%%%%%%%%%%
%%%%%%%%%%%%%%%%%%%%%%%%%%%%%%%%%%%%%%%%%%%%%%%%%%
\chapter{Sun Position}\label{sec-sunposition}

SAM calculates the sun altitude, zenith, and declination angles that define its position for each time step in the weather file. SAM's sun position algorithm is based on the method described in \citet{michalsky1988}, which in turn is based on the Astronomical Almanac's algorithm for the period 1950-2050. NREL modified the Michalsky algorithm to calculate sun azimuth angles for locations south of the equator using the approach described in \citep{iqbal1983}. The algorithm is further discussed in \citet{stackoverflow2012}.

The overall sun position algorithm steps are:
% lib_irradproc 835-887
\begin{enumerate}
\item Calculate the effective time in hours for the time step.
\item Calculate the sun angle for the time step.
\item Determine the current day's sunrise and sunset time.
\item Determine the sunup flag for the time step.
\item Calculate the extraterrestrial radiation for the time step.
\end{enumerate}

Although all of the sun angle input and results variables in SAM are in degrees, most of the internal equations use angle values in radians.

Table~\ref{tab-sunposvars} lists the sun position algorithm's inputs and outputs. Figure~\ref{fig-sunangles}, adapted from \citet{dunlap2007}, shows the sun angles. The inputs are for each time step, and come from the weather file. 

The sun position algorithm does not include an air mass equation. Instead, each algorithm that needs the value uses its own equation to calculate the value: The Perez incident sky diffuse model uses Equation~\ref{eqn-perezam}, and the Sandia and CEC module models use the same equation, shown in Equations~\ref{eqn-sandiaam} and \ref{eqn-cecam}, respectively. 

\begin{table}
\begin{center}
\caption{Sun Position Variable Definitions}
\begin{tabular}{lll}
\midrule
Symbol & Description / \textbf{Name in SAM} & Name in SSC\\
\midrule
\multicolumn{3}{c}{Inputs}\\
$\mathit{tz}$ & time zone (hrs E of GMT, -12 < $tz$ < 12) & \texttt{tz}\\
$\mathit{lat}$ & latitude (decimal $^\circ$N of equator, 0 < $lat$ < 90) & \texttt{lat}\\
$\mathit{lon}$ & longitude (decimal $^\circ$E of GMT, -180 < $lon$ < 180) & \texttt{lon}\\
$\mathit{yr}$ & year (e.g., 1988) & \texttt{year}\\
$\mathit{mo}$ & month year (1-12) & \texttt{month}\\
$\mathit{day}$ & day of year (1-365) & \texttt{day}\\
$\mathit{hr}$ & hour of day local time (0-23) & \texttt{hour}\\
$\mathit{min}$ & minute of hour (0-59) & \texttt{minute}\\
\midrule
\multicolumn{3}{c}{Outputs}\\
$Z$ & \textbf{Solar zenith angle} (deg) & \texttt{sun\_zen}\\
$\alpha$ & \textbf{Solar altitude angle} (deg) & \texttt{sun\_elv}\\
$\delta$ & solar declination angle (deg) & \texttt{sun\_dec}\\
$\gamma$ & \textbf{Solar azimuth angle} (deg) & \texttt{sun\_azm}\\
$\mathit{sunup}$ & \textbf{Sun up over horizon}  (0/1/2/3)& \texttt{sunup}\\
\midrule
\end{tabular}
\label{tab-sunposvars}
\end{center}
\end{table}

\begin{figure}
\begin{center}
\includegraphics[scale=0.6]{sun-angles}
\caption{Sun Angles}
\label{fig-sunangles}
\end{center}
\end{figure}

\section{Effective Time}

The first step in the sun position algorithm is to determine the effective time of the current time step. SAM reads the time stamp data for the current time step in the weather file (represented by a data row) as a year, month, day, hour, and minute value. %lib_irradproc.cpp 102

The standard weather files from the National Solar Radiation Database \citep{nsrdb} start at $hr=1$. The algorithm converts the first hour number $hr=0$: 
\begin{equation}
hr=hr-1
\end{equation}

For hourly weather files that do not contain a Minute column, the algorithm uses the midpoint of the time step for sun position calculations (except for the hours containing sunrise and sunset, see Section~\ref{sec-sunriseset}) so that $min=30$. Otherwise, for weather data that contains a Minute time step, $min$ is the value for the current time step from the weather file.

%lib_irradproc.cpp 102
The Julian day of year $\mathit{jdoy}$ is the number of days since Noon on January 1 of the current year (strictly speaking, it should be called the \textit{ordinal date}). Because some weather files are typical year files, the time stamps in a given file may not all use the same year. For example, all January time stamps may be 1988, while all February time stamps may be 2005 \citep{tmy3}.

To account for leap years:
%lib_irradproc.cpp 26
\begin{equation}
k = 
\left\{
   \begin{array}{ll}
      1 & \text{if $\mathit{year}\mod4=0$}\\
      0 & \text{if $\mathit{year}\mod4\neq0$}\\
   \end{array}
\right. 
\end{equation}

Note that this accounts for leap years to correctly calculate effective time, but is separate from the energy simulation, which does not account for leap years.

%lib_irradproc.cpp 31
SAM calculates the number of days since January 1 $a$ from the number of days in each of the months (January = 31, February = 28, March = 31, etc.) before the current month, and the number of days since the first of the current month.

The Julian day of year is then:
%lib_irradproc.cpp 36
\begin{equation}
\mathit{jdoy}= 
\left\{
   \begin{array}{ll}
      \mathit{day} + a & \text{for January and February}\\
      \mathit{day} + a + k & \text{for March through December}
   \end{array}
\right. 
\end{equation}

The current decimal time of day expressed as an offset from UTC depends on the hour and minute of the current time stamp, and the location's time zone. Valid values for the time zone are $-12<\mathit{tz}<12$.
%lib_irradproc.cpp 103
\begin{equation}\label{eqn-tutc}
t_{utc} = \mathit{hr} + \frac{\mathit{min}}{60} - \mathit{tz}
\end{equation}

For some combinations of time stamp and time zone, Equation~\ref{eqn-tutc} may yield a value less than zero or greater than 24 hours, in which case the following correction applies:
%lib_irradproc.cpp 104
\begin{equation}
t_{utc} = \left\{
   \begin{array}{ll}
       t_{utc} + 24 & 
      \text{if $t_{utc}<0$}\\
      t_{utc} - 24 & 
      \text{if $t_{utc}>24$}
   \end{array}
\right. 
\end{equation}
The Julian date $\mathit{julian}$ of the current hour is the Julian day of the preceding noon plus the number of hours since then. The Julian day is defined as the number of days since Noon on January 1, 2000:
% time irradproc.cpp 117
\begin{equation}\label{eqn-jday}
\mathit{julian} = 32916.5 + 365(\mathit{yr}-1949) + \frac{\mathit{yr}-1949}{4} + \mathit{jdoy} + \frac{t_{utc}}{24} - 51545
\end{equation}

\section{Sun Angles} \label{sec-sunangles}

The sun angle equations are from \citet{michalsky1988}. The sun angles (Figure~\ref{fig-sunangles}) are the altitude angle $\alpha$, declination angle $\delta$, and zenith angle $Z$. SAM also calculates the sun azimuth angle $\gamma$ for use in the incident irradiance calculations. The solar declination angle is not used in the incident irradiance calculations, but is required to calculate the sun azimuth angle. The bold font in Table~\ref{tab-sunposvars} indicates that SAM reports the sun zenith, altitude, and azimuth angles in the results.

The first step in the sun angle calculation for a given time step is to determine the ecliptic coordinates of the location, which define the photovoltaic array's position on the earth relative to the sun. The ecliptic coordinate variables are the mean longitude, mean anomaly, ecliptic longitude, and obliquity of the ecliptic. The algorithm uses ecliptic coordinates instead of equatorial coordinates to include the effect of the earth's inclination in the sun angle calculations.

Where limits are indicated for the equations below, if the variable's value falls outside of the limits, SAM adjusts the value. For example, for a value $x$ with the limits $0\leq\mathit{x}<360\degree$, SAM divides $x$ by $360\degree$, and checks to see whether the remainder is less than zero, and if it is, adds $360\degree$ to the remainder:
%lib_irradproc 120
\begin{align}\label{eqn-eclipticlimits}
a &= \mathit{x} - 360\degree~\text{trunc} \left(\frac{x}{360\degree}\right)\notag\\
\mathit{x}&= \left\{
  \begin{array}{ll}
    a & \text{if $a\geq0$}\\
    a+360\degree & \text{if $a<0$}
  \end{array}
\right.
\end{align}

%lib_irradproc 119-145 (mnlong is not converted to radians)
Mean longitude in degrees ($0\leq\mathit{mnlong}<360\degree$). Note that the mean longitude is the only value not converted to radians:
\begin{equation}\label{eqn-mnlong}
\mathit{mnlong} = 280.46 + 0.9856474~\mathit{julian}
\end{equation}

Mean anomaly in radians ($0\leq\mathit{mnanom}<2\pi$):
\begin{equation}\label{eqn-mnanom}
\mathit{mnanom} = \frac{\pi}{180}\left(357.528 + 0.9856003~\mathit{julian}\right)
\end{equation}

Ecliptic longitude in radians ($0\leq\mathit{eclong}<2\pi$):
\begin{equation}\label{eqn-eclong}
\mathit{eclong} = \frac{\pi}{180}\left[\mathit{mnlong} + 1.915\sin \mathit{mnanom} + 0.02\sin(2\mathit{mnanom})\right]
\end{equation}

Obliquity of ecliptic in radians:
\begin{equation}\label{eqn-oblqec}
\mathit{obleq} = \frac{\pi}{180}\left(23.439 - 0.0000004~\mathit{julian}\right)
\end{equation}

The next step is to calculate the celestial coordinates, which are the right ascension and declination.

The right ascension in radians:
%lib_irradproc.cpp 137
\begin{equation}
\mathit{ra} = \left\{
\begin{array}{ll}
\arctan\left(  \frac{ \cos\mathit{\mathit{obleq}}\sin{\mathit{eclong}}}{\cos\mathit{eclong}}  \right) + \pi & \text{if $\cos\mathit{eclong}<0$}\\
\arctan\left(  \frac{ \cos\mathit{\mathit{obleq}}\sin{\mathit{eclong}}}{\cos\mathit{eclong}}  \right) + 2\pi & \text{if $\cos\mathit{obleq}\sin\mathit{eclong}<0$}\\
\end{array}
\right.
\end{equation}

The solar declination angle in radians:
\begin{equation}\label{eqn-dec}
\delta = \arcsin \left( \sin \mathit{obleq} \sin \mathit{eclong} \right)
\end{equation}

Next are the local coordinates, which require calculating the hour angle. 

The Greenwich mean siderial time in hours ($0\leq\mathit{gmst}<24$) with limits applied as shown in Equation~\ref{eqn-eclipticlimits} depends on the current time at Greenwich $t_{utc}$ from Equation~\ref{eqn-tutc}, and the Julian day from Equation~\ref{eqn-jday}:
%lib_irradproc.cpp 147
\begin{equation}
\mathit{gmst}= 6.697375 + 0.0657098242~\mathit{julian} + t_{\mathit{utc}}
\end{equation}

Local mean siderial time in hours ($0\leq\mathit{lmst}<24$):
%lib_irradproc.cpp 152
\begin{equation}
\mathit{lmst}= \mathit{gmst} + \frac{\mathit{lon}}{15}
\end{equation}

The hour angle in radians ($-\pi<\mathit{HA}<\pi$):
%lib_irradproc.cpp 158
\begin{align}
b &= 15\frac{\pi}{180}\mathit{lmst} -\mathit{ra}\notag\\
\mathit{HA}&= \left\{
  \begin{array}{ll}
  b + 2\pi & \text{if $b<-\pi$}\\
  b - 2\pi & \text{if $b>\pi$}
  \end{array}
\right.
\end{align}

The sun altitude angle in radians, not corrected for refraction:
%lib_irradproc.cpp 166
\begin{align} \label{eqn-elv}
a &= \sin\delta \sin\left(\frac{\pi}{180}\mathit{lat}\right) + \cos\delta \cos\left(\frac{\pi}{180}\mathit{lat}\right) \cos\mathit{HA}\notag\\
\alpha_0 &=\left\{
  \begin{array}{ll}
    \arcsin a & \text{if $-1 \leq a \leq 1$}\\
    \frac{\pi}{2} & \text{if $a>1$}\\
    -\frac{\pi}{2} & \text{if $a<-1$}
  \end{array}
\right.
\end{align}

The sun altitude angle $\alpha$ in radians corrected for refraction is (these equations use the uncorrected sun altitude angle in degrees, indicated by the subscript $d$):
% lib_irradproc 194
\begin{align}\label{eqn-elvcorr}
\alpha_{0d} &= \frac{180}{\pi}\alpha_0\notag\\
r &= \left\{
\begin{array}{ll}
3.51561\left(0.1594 + 0.0196\alpha_{0d} + 0.00002\alpha_{0d}^2\right) \left(1 + 0.505\alpha_{0d} + 0.0845\alpha_{0d}^2\right)^{-1} & \text{if $\alpha_{0d}>-0.56$}\\
0.56 & \text{if $\alpha_{0d} \leq -0.56$}
\end{array}
\right.\notag\\
\alpha & = \left\{
\begin{array}{ll}
\frac{\pi}{2} & \text{if $\alpha_{0d} + r > 90$}\\
\frac{\pi}{180}\left(\alpha_{0d} + r\right) & \text{if $\alpha_{0d} + r \leq 90$}
\end{array}
\right.
\end{align}

The sun azimuth angle $\gamma$ in radians is from \citep{iqbal1983} rather than \citep{michalsky1988} because the latter is only for northern hemisphere locations:
%lib_irradproc 174
\begin{align}\label{eqn-solaraz}
a &= \frac{\sin\alpha_0 \sin\left(\frac{\pi}{180}\mathit{lat}\right) - \sin\delta}{\cos\alpha_0 \cos\left(\frac{\pi}{180}\mathit{lat}\right)}\notag\\
b &= \left\{
\begin{array}{ll}
\arccos a & \text{if $-1 \leq a \leq 1$}\\
\pi & \text{if $\cos\alpha_0=0$, or if $a < -1$}\\
0 & \text{if $a > 1$}
\end{array}
\right.\notag\\
\gamma &= \left\{
\begin{array}{ll}
b & \text{if $\mathit{HA} < -\pi$}\\
\pi - b & \text{if $-\pi \leq \mathit{HA} \leq 0$, or if $\mathit{HA} \geq \pi$}\\
\pi + b & \text{if $0 < \mathit{HA} < \pi$}
\end{array}
\right.
\end{align}

The sun zenith angle $Z$ in radians:
%lib_irradproc 228
\begin{equation}\label{eqn-zen}
Z = \frac{\pi}{2}-\alpha
\end{equation}

\section{Sunrise and Sunset Hours}\label{sec-sunriseset}

The photovoltaic model assumes that the array generates electricity starting in the hour that contains the sunrise and ending in the hour that contains the sunset.

To determine whether the current time step contains a sunrise, the sunrise hour angle is:
%lib_irradproc.cpp 210
\begin{align}\label{eqn-sunriseha}
a &= -\tan \mathit{lat}~\tan\delta\notag\\
\mathit{HAR} &= 
\left\{
   \begin{array}{lll}
      0 & \text{if $a\geq1$} & \text{sun is down}\\
      \pi & \text{if $a\leq-1$} & \text{sun is up}\\
      \arccos a & \text{if $-1<a<1$} & \text{sunrise}
   \end{array}
\right. 
\end{align}

The equation of time in hours:
%lib_irradproc.cpp 204
\begin{align}\label{eqn-eot}
a &= \frac{1}{15}\left(\mathit{mnlong} - \frac{\pi}{180}\mathit{ra}\right)\notag\\
\mathit{EOT} &= 
\left\{
   \begin{array}{ll}
     a & \text{if $-0.33 \leq a \leq 0.33$}\\
     a + 24 & \text{if $a<-0.33$}\\
     a - 24 & \text{if $a>0.33$}
   \end{array}
\right. 
\end{align}

The sunrise hour in local standard decimal time:
%lib_irradproc.cpp 219
\begin{equation}
h_{sunrise} = 12 - \frac{1}{15} \frac{180}{\pi}\mathit{HAR} - \left(\frac{\mathit{lon}}{15} - \mathit{tz}\right)-\mathit{EOT}
\end{equation}

And, the sunset hour in local standard time: 
\begin{equation}
h_{sunset} = 12 + \frac{1}{15} \frac{180}{\pi}\mathit{HAR} - \left(\frac{\mathit{lon}}{15} - \mathit{tz}\right)-\mathit{EOT}
\end{equation}

For hourly data, the sunrise minute is at the midpoint between the minute during which the sun rose and the end of the time step. The sunset minute is at the midpoint between the beginning of the time step and the minute during which the sun set: 
%adapted from irradproc.cpp 847 for 60 minute (3600 second) time step

\begin{align}
\begin{array}{ll}
\mathit{m}_\mathrm{sunrise}&=60 \left(1 - (h_{sunrise} -\mathit{hr}) + \frac{(t_{\mathrm{sunrise}} - hr)}{2}\right)\\
\mathit{m}_{\mathrm{sunset}}&=\frac{60 (h_{\mathrm{sunset}} - \mathit{hr})}{2}
\end{array}
\end{align}

For sub-hourly data, the sunrise and sunset minute is determined by the minute indicated by the time stamp:
\begin{align}
\begin{array}{ll}
\mathit{m}_\mathrm{sunrise}&=\mathit{min}\\
\mathit{m}_\mathrm{sunset}&=\mathit{min}
\end{array}
\end{align}

\section{Sunup Flag}\label{sec-sunup}

The sunup flag indicates whether the sun is above or below the horizon in the current time step. The photovoltaic model uses this value to determine whether to calculate the cell temperature in the current time step. It only reports the sunrise and sunset hour ($\mathit{sunup}=2$ and $\mathit{sunup}=3$, respectively) for hourly data. Its value is determined from the sunrise hour angle (Equation~\ref{eqn-sunriseha}):

\begin{align}\label{eqn-sunrise}
sunup &= \left\{
\begin{array}{ll}
0 & \text{sun is down}\\
1 & \text{sun is up}\\
2 & \text{sunrise}\\
3 & \text{sunset}
\end{array}
\right.
\end{align}

\section{Extraterrestrial Radiation}\label{sec-hextra}

Extra terrestrial radiation $H$ is the solar radiation at the top of the earth's atmosphere in $\mathrm{W/m^2}$. SAM uses the value in the incident irradiance calculations described in Section~\ref{sec-incidentirradiance}.
 
The extraterrestrial radiation equation is from Chapter 1.10 of \citet{duffie2013}:
%lib_irradproc.cpp 229
\begin{align}\label{eqn-hextra}
G_{\mathrm{on}} &= 1367 \left[1+0.033\cos\left( \frac{\pi}{180}\frac{360\mathit{doy}}{365} \right)\right]\notag\\
H &= \left\{
\begin{array}{ll}
G_{\mathrm{on}} \cos Z & \text{if $0 < Z < \frac{\pi}{2}$ (sun is up)}\\
G_{\mathrm{on}} & \text{if $Z=0$}\\
0 & \text{if $Z<0$, or if $Z>\frac{\pi}{2}$}
\end{array}
\right.
\end{align}

\section{True Solar Time and Eccentricity Correction Factor}

The sun position algorithm calculates two values that are not used by the photovoltaic performance model. 

True solar or apparent time in decimal hours:
%lib_irradproc.cpp 225
\begin{equation}
t_{\mathrm{truesolar}} = \mathit{hr} + \frac{min}{60} + \frac{\mathit{lon}}{15} - \mathit{tz} + \mathit{EOT}
\end{equation}

The eccentricity correction factor:
%lib_irradproc.cpp 222
\begin{equation}
E_0 = [ 1.00014 - 0.01671\cos\mathit{mnanom} - 0.00014\cos(2\mathit{mnanom}) ]^{-2}
\end{equation}

%%%%%%%%%%%%%%%%%%%%%%%%%%%%%%%%%%%%%%%%%%%%%%%%%%
%%%%%%%%%%%%%%%%%%%%%%%%%%%%%%%%%%%%%%%%%%%%%%%%%%
\chapter{Surface Angles}\label{sec-surfaceangles}

SAM considers each subarray in the system to be a flat surface with one tilt angle $\beta_s$ and one azimuth angle $\gamma_s$ that define the surface orientation. The surface angles depend on whether the subarray is fixed, or mounted on one-axis, two-axis, or azimuth-axis trackers. Surfaces with one-axis trackers have a third surface angle $\sigma$ defining its rotation around the tracking axis.

The surface angle equations are based on standard geometric relationships defined by the surface orientation and sun angles (Section~\ref{sec-sunangles}).

SAM calculates each subarray's surface angles for each time step of the simulation. For systems with trackers, the surface angles in a given time step are fixed over the time step. The surface angles are for the same minute of the hour as the sun position angles: Either the minute indicated in the weather file, or for hourly weather data with no Minute column, the midpoint of the hour (Section~\ref{sec-timeconvention}).

Table~\ref{tab-surfaceanglevars} defines the variables used for the equations in this section. Figure~\ref{fig-arrayorientation}, adapted from \citet{dunlap2007}, shows how the surface angles are defined.

%1205			irr.set\_sky\_model( skymodel, alb );
%				if ( radmode == 0 ) irr.set\_beam\_diffuse( wf.dn, wf.df );
%				else if (radmode == 1) irr.set\_global\_beam( wf.gh, wf.dn );

\begin{table}
\begin{center}
\caption{Surface Angle Variable Definitions}
\begin{tabular}{lll}
\midrule
Symbol & Description / \textbf{Name in SAM} & Name in SSC\\
\midrule
\multicolumn{3}{c}{Inputs}\\
- & \textbf{Fixed}, \textbf{1 Axis}, \textbf{2 Axis}, \textbf{Azimuth Axis} & \texttt{track\_mode}\\
$\beta_0$ & \textbf{Tilt} (deg)& \texttt{tilt}\\
$\gamma_0$ & \textbf{Azimuth} (deg) & \texttt{azimuth}\\
$\theta_{\mathrm{lim}}$ & \textbf{Tracker rotation limit} (deg) & \texttt{rotlim}\\
$Z$ & sun zenith angle (deg) & \texttt{sun\_zen}\\
$\gamma$ & sun azimuth angle (deg) & \texttt{sun\_azm}\\
- & \textbf{Backtracking }& \texttt{backtrack}\\
$\GCR$ & \textbf{Ground coverage ratio (GCR)} & \texttt{GCR}\\
- & \textbf{Beam and diffuse}, \textbf{Total and beam} & \texttt{irrad\_mode}\\
- & \textbf{Isotropic}, \textbf{HDKR}, \textbf{Perez} & \texttt{sky\_model}\\
\midrule
\multicolumn{3}{c}{Outputs}\\
$\AOI$ & incidence angle & \texttt{incidence}\\
$\beta_s$ & \textbf{Subarray [\textit{n}] Surface tilt} (deg) & \texttt{surf\_tilt}\\
$\gamma_s$ & \textbf{Subarray [\textit{n}] Surface azimuth} (deg) & \texttt{surf\_azm}\\
$\theta$ & \textbf{Subarray [\textit{n}] Axis rotation for 1 axis trackers} (deg) & \texttt{axis\_rotation}\\
$\theta_0$ & \textbf{Subarray [\textit{n}] Ideal axis rotation for 1 axis trackers} (deg) & -\\
$\Delta\theta$ & backtracking difference from ideal rotation (deg) & \texttt{bt\_diff}\\
\midrule
\end{tabular}
\label{tab-surfaceanglevars}
\end{center}
\end{table}

\begin{figure}
\begin{center}
\includegraphics[scale=0.6]{surface-angles}
\caption{Surface Angles}
\label{fig-arrayorientation}
\end{center}
\end{figure}

\section{Angle of Incidence} \label{sec-theta}

The angle of incidence $\AOI$ is the sun incidence angle defined as the angle between beam irradiance and a line normal to the subarray surface (See Figure~\ref{fig-arrayorientation}). It is a function of the sun azimuth angle $\gamma$, sun zenith angle $Z$, surface azimuth angle $\gamma_s$, and the surface tilt angle $\beta_s$:
%lib_irradproc.cpp 292
\begin{align}\label{eqn-inc}
a &= \sin Z \cos(\gamma - \gamma_s) \sin\beta_s + \cos Z \cos\beta_s \notag\\
\AOI &= \left\{
\begin{array}{ll}
\pi & \text{if $a<-1$}\\
0 & \text{if $a>1$, or for two-axis tracking}\\
\arccos a & \text{if $-1 \leq a \leq 1$}
\end{array}
\right.
\end{align}

\section{Fixed, Azimuth and Two-axis Tracking}

The surface azimuth and tilt angle values depend on the tracking option as described below.

For a fixed surface (no tracking):
\begin{align}\label{eqn-fxtilt}
\beta_s &= \beta_0 \notag\\
\gamma_s &= \gamma_0
\end{align}

For azimuth tracking, the surface tilt angle is fixed, and the surface azimuth angle follows the sun azimuth angle:
\begin{align}\label{eqn-aztilt}
\beta_s &= \beta_0 \notag\\
\gamma_s &= \gamma
\end{align}

For two-axis tracking, the surface tilt and azimuth angles follow the sun zenith and azimuth angles, respectively:
\begin{align}\label{eqn-2xtilt}
\beta_s & = Z \notag\\
\gamma_s & = \gamma
\end{align}

\section{One-axis Tracking}

One-axis tracking involves the tilt angle $\beta$, azimuth angle $\gamma$ and a third surface angle $\theta$ describing the surface's rotation about the tracking axis and a rotation angle limit $\theta_{lim}$.

 

SAM offers three shade modes for the one-axis tracking option on the PV Subarrays input page:
\begin{itemize}
\item \textbf{Self-shaded} accounts for shading of modules in one row by those in neighboring rows (see Section~\ref{sec-selfshad}), but does not model backtracking.
\item \textbf{Backtracking} adjusts the tracking angle to avoid shading of modules in one row by those in neighboring rows (see Section~\ref{sec-backtrack1x}).
\item \textbf{None} does not account for self-shading or backtracking.
\end{itemize}

All one-axis surface angle equations use angle values converted from degrees to radians.

The one-axis tracking tilt angle in radians (See Sections~\ref{sec-rot1x} and \ref{sec-backtrack1x} for rotation angle equations):
%lib_irradproc.cpp 386
\begin{align}
a & = \cos\beta_0 \cos\theta\notag\\
\beta_s &= \left\{
\begin{array}{ll}
\pi & \text{if $a<-1$}\\
0 & \text{if $a>1$}\\
\arccos a & \text{if $-1 \leq a \leq 1$}
\end{array}
\right.
\end{align}

The one-axis tracking surface azimuth angle in radians:
%lib_irradproc.cpp 394
\begin{align}
a & = \frac{\sin\theta}{\sin\beta}\notag\\
\gamma_s &= \left\{
\begin{array}{ll}
\pi & \text{if $\beta_s=0$}\\
\frac{3\pi}{2} + \gamma_0 & \text{if $a<-1$}\\
\frac{\pi}{2} + \gamma_0 & \text{if $a>1$}\\
\gamma_0 - \pi - \arcsin{a} & \text{if $\theta < -\frac{\pi}{2}$}\\
\gamma_0 + \pi - \arcsin{a} & \text{if $\theta > \frac{\pi}{2}$}\\
\gamma_0 + \arcsin{a} & \text{if $-1 \leq a \leq 1$ or $-\frac{\pi}{2} \leq \theta \leq \frac{\pi}{2}$}
\end{array}
\right.
\end{align}

The surface azimuth angle in radians must be between zero and $2\pi$:
%lib_irradproc.cpp 409
\begin{equation}
\gamma_s = \left\{
\begin{array}{ll}
\gamma_s - 2\pi & \text{if $\gamma_s > 2\pi$}\\
\gamma_s + 2\pi & \text{if $\gamma_s < 0$}
\end{array}
\right.
\end{equation}

\subsection{Rotation Angle for One-axis Trackers}\label{sec-rot1x}

For a subarray with one-axis tracking, SAM assumes that the tracker rotates about an axis tilted from the horizontal at the surface tilt angle along the line that defines the surface azimuth angle shown in Figure~\ref{fig-arrayorientation}. 

The \textbf{tracker rotation angle} $\theta$ is the angle of the subarray surface from the horizontal about the tracking axis. For a surface tilted as shown Figure~\ref{fig-arrayorientation}, the rotation angle is the angle as viewed from the raised end of the surface, with a negative rotation angle indicating counter-clockwise rotation from the horizontal. For a surface in the northern hemisphere with a surface azimuth angle of $180^\circ$, a negative rotation angle is for a surface facing east toward the morning sun with $-90^\circ$ for a vertical, east-facing surface. A positive rotation angle is for an array facing the afternoon sun, with $90^\circ$ for a vertical, west-facing surface.  In the southern hemisphere, for a surface with an azimuth angle of $0^\circ$, a negative rotation limit indicates a west-facing surface.

The \textbf{tracker rotation limit} $\theta_{lim}$ is the physical limit of the tracker's motion about the tracking axis in degrees with valid values between $0$ and $\pm 90^\circ$. (Note that SAM's user interface restricts the rotation limit values to $\pm 85\degree$.) For example, a tracker in the northern hemisphere with a rotation limit of 45$^\circ$ would start tracking at a rotation angle of $-45^\circ$ in the sunrise time step, and stop tracking at a rotation angle of $45^\circ$ in the sunset time step.

SAM adjusts the surface rotation angle as follows:
\begin{itemize}
\item If the tracker rotation angle exceeds the user-specified tracker rotation limit, then the tracker rotation is set to the limit.
\item If backtracking is enabled and the array is self-shaded, then the surface tilt angle is adjusted to minimize self-shading (See Section~\ref{sec-backtrack1x}).
\end{itemize}

The one-axis tracking equations described below use angle values converted from degrees to radians. SAM displays surface angle values in the results in degrees.

The ``ideal rotation angle" $\theta_0$ for one-axis tracking is the rotation angle without application of the rotation limit $\theta_{\mathrm{lim}}$ or backtracking:
%lib_irradproc.cpp 323
\begin{align}
a &= \frac{\sin Z \sin(\gamma - \gamma_s)}{\sin Z \cos(\gamma - \gamma_s) \sin\beta + \cos Z \cos\beta} \notag\\
\theta_{\mathrm{01}} &= \left\{
\begin{array}{ll}
-\frac{\pi}{2} & \text{if $a<-99,999.9$}\\
\frac{\pi}{2} & \text{if $a>99,999.9$}\\
\arctan a & \text{if $-99,999.9 \leq a \leq 99,999.9$}
\end{array}
\right.
\end{align}

The following corrections ensure that the ideal rotation angle is in the correct quadrant (II or III) when the surface azimuth angle $\gamma_s$ is less than or greater than $\pi$:
%lib_irradproc.cpp 335
\begin{equation}
\theta_0 = \left\{
\begin{array}{ll}
\theta_{\mathrm{01}} + \pi & \text{if $\gamma_s < \pi$ and $\gamma < \gamma_s \leq \gamma_s + \pi$ and $\theta_0 < 0$ (quadrant II positive rotation)} \notag\\
\theta_{\mathrm{01}} - \pi & \text{if $\gamma_s < \pi$ and $\gamma < \gamma_s \leq \gamma_s + \pi$ and $\theta_0 > 0$ (quadrant III negative rotation)} \notag\\
\theta_{\mathrm{01}} + \pi & \text{if $\gamma_s > \pi$ and $\gamma < \gamma_s \leq \gamma_s - \pi$ and $\theta_0 < 0$ (quadrant II positive rotation)} \notag\\
\theta_{\mathrm{01}} - \pi & \text{if $\gamma_s > \pi$ and $\gamma < \gamma_s \leq \gamma_s - \pi$ and $\theta_0 > 0$ (quadrant III negative rotation}
\end{array}
\right.
\end{equation}

The rotation angle in radians is the ideal rotation angle limited by the rotation limit:
%lib_irradproc.cpp 364
\begin{equation}\label{eqn-rot1x}
\theta = \left\{
\begin{array}{ll}
\theta_0 & \text{if $-\theta_{\mathrm{lim}} \leq \theta_0 \leq \theta_{\mathrm{lim}}$}\\
-\theta_{\mathrm{lim}} & \text{if $\theta_0 < -\theta_{\mathrm{lim}}$}\\
\theta_{\mathrm{lim}} &  \text{if $\theta_0 > \theta_{\mathrm{lim}}$}
\end{array}
\right.
\end{equation}

\subsection{Backtracking for One-axis Trackers}\label{sec-backtrack1x}

Backtracking is a technique used with some one-axis trackers to minimize self-shading of neighboring rows of photovoltaic modules during times that the sun is low in the sky. When neighboring rows shade each other, the tracker rotates toward the horizontal to reduce the size of shadows on the array.

SAM's backtracking for one-axis tracking algorithm was developed by NREL for SAM. It involves the following steps:
%lib_irradproc.cpp 373, 1258
\begin{enumerate}
\item Determine the shaded fraction of the subarray (Section~\ref{sec_selfshad1x}) given the ground coverage ratio $\GCR$, sun angles (Section~\ref{sec-sunangles}), surface angles (Section~\ref{sec-surfaceangles}), and surface rotation angle $\theta$ (Equation~\ref{eqn-rot1x}).
\item If the shaded fraction is greater than zero, adjust the tracker rotation angle $\theta$ toward $0^\circ$ by $1^\circ$.
\item Repeat Steps 1-2 until $F_{\mathrm{shad1x}}=0$, or one hundred times, whichever comes first (to avoid an infinite loop).
\end{enumerate}

The backtracking rotation difference $\Delta\theta$ reported by SSC as \texttt{bt\_diff}:
%lib_irradproc.cpp 379
\begin{equation}
\Delta\theta = \theta - \theta_0
\end{equation}

%%%%%%%%%%%%%%%%%%%%%%%%%%%%%%%%%%%%%%%%%%%%%%%%%%
%%%%%%%%%%%%%%%%%%%%%%%%%%%%%%%%%%%%%%%%%%%%%%%%%%
\chapter{Incident Irradiance}\label{sec-incidentirradiance}

The incident irradiance, also called plane-of-array irradiance or POA irradiance, is the solar irradiance incident on the plane of the photovoltaic array in a given time step. SAM calculates the incident irradiance for the sunrise, sun-up, and sunset time steps. The incident angle algorithm calculates the beam and diffuse irradiance incident on the photovoltaic subarray surface for a given sun position, latitude, and surface orientation. For each time step in the simulation, the incident irradiance algorithm steps are:

%lib_irradproc.cpp 896-938
\begin{enumerate}
\item Calculate the angle of incidence (Section~\ref{sec-theta}).
\item Calculate the beam irradiance on a horizontal surface.
\item Check to see if the beam irradiance on a horizontal surface exceeds the extraterrestrial radiation.
\item Calculate the incident beam irradiance.
\item Calculate the sky diffuse horizontal irradiance using one of the three sky diffuse irradiance methods.
\item Calculate the ground-reflected irradiance.
\end{enumerate}

\begin{table}
\begin{center}
\caption{Incident Irradiance Variable Definitions}
\begin{tabular}{lll}
\midrule
Symbol & Description / \textbf{Name in SAM} & Name in SSC\\
\midrule
\multicolumn{3}{c}{Inputs}\\
$E_b$ & \textbf{Beam irradiance} (W/m$^2$)& \texttt{beam}\\
$E_d$ & \textbf{Diffuse irradiance} (W/m$^2$) & \texttt{diffuse}\\
$E_g$ & \textbf{Global horizontal irradiance} (W/m$^2$) & \texttt{global}\\
$H$ & extraterrestrial irradiance (W/m$^2$) & -\\
$\mathit{\rho}$ & \textbf{Albedo} (ground reflectance)& \texttt{albedo}\\
$\AOI$ & \text{angle of incidence} (deg) & \texttt{incidence}\\
$\beta_s$ & \textbf{Subarray [\textit{n}] Surface tilt} (deg) & \texttt{tilt}\\
$Z$ & \textbf{Solar zenith angle}  (deg)& \texttt{sun\_zen}\\
- & \textbf{Beam and diffuse}, \textbf{Total and beam}, \textbf{Total and diffuse} & \texttt{irrad\_mode}\\
- & \textbf{Isotropic}, \textbf{HDKR}, \textbf{Perez} & \texttt{sky\_model}\\
- & \textbf{Use albedo in weather file if it is specified} & \texttt{use\_wf\_albedo}\\
\midrule
\multicolumn{3}{c}{Outputs}\\
$I_b$ & incident beam irradiance (W/m$^2$)& \texttt{poa\_beam}\\
$I_d$ & incident sky diffuse irradiance (W/m$^2$)& \texttt{poa\_skydiff}\\
$I_r$ & incident ground-reflected irradiance (W/m$^2$) & \texttt{poa\_gnddiff}\\
$D_i$ & isotropic component of incident diffuse irradiance (W/m$^2$) & \texttt{poa\_skydiff\_iso}\\
$D_c$ & circumsolar component of incident diffuse irradiance (W/m$^2$) & \texttt{poa\_skydiff\_cir}\\
$D_h$ & horizon brightening component of incident diffuse irradiance (W/m$^2$)& \texttt{poa\_skydiff\_hor}\\
\midrule
\end{tabular}
\label{tab-incidentirradiancevars}
\end{center}
\end{table}

\section{Bypassing POA Calculations}\label{sec-poa}

The photovoltaic performance model allows you to use measured plane-of-array (POA) irradiance data as input. When POA data is available, you can create a weather file with a column with the heading "POA" that contains the data. When you choose one of the POA options below, SAM uses the POA data from the weather file instead of calculating it from  beam normal and horizontal diffuse irradiance data \citep{freeman2016}.

To use POA data as input, choose the appropriate \textbf{PV Albedo and Radiation} option on the Location and Resource page: 

\begin{itemize}
\item \textbf{POA from reference cell} ($\texttt{irrad\_proc}=3$ in SSC) for data measured by a reference photovoltaic cell in the same plane as the array surface.
\item \textbf{POA from pyranometer} ($\texttt{irrad\_proc}=4$ in SSC) for data measured by a pyranometer in the same plane as the array surface.
\end{itemize}

If you apply beam shading losses (Section~\ref{sec-nearobjectshad}) to POA data, SAM must decompose the POA irradiance into beam and diffuse irradiance components in order to apply the shading loss to the beam component. 

\section{Incident Beam Irradiance}\label{sec-incidentbeam}

The incident beam irradiance is solar energy that reaches the surface in a straight line from the sun:
%lib_irradproc.cpp 520 (isotropic), 602 (perez), 481 (hdkr)
\begin{equation}
I_b = E_b\cos\AOI
\end{equation}

The beam irradiance on a horizontal surface:
%lib_irradproc.cpp 905
\begin{equation}\label{eqn-hbeam}
I_{\mathrm{bh}}=E_b\cos Z
\end{equation}

%cmod_pvsamv1.cpp 1213
SAM compares $I_{\mathrm{bh}}$ to the extraterrestrial radiation $H$ (Equation~\ref{eqn-hextra}). If $I_{\mathrm{bh}}>H$, it generates an error flag that causes the calculations to stop and SAM to display the error message``failed to compute irradiation on surface."

The \textbf{Radiation Components} option on the Location and Resource input page (\texttt{irrad\_mode} in SSC) allows you to choose whether to use the beam irradiance  irradiance from the weather file, or whether to calculate it from the global horizontal and diffuse horizontal irradiance values:
%lib_irradproc.cpp 1001
\begin{equation}\label{eqn-beam}
E_b = \left\{
\begin{array}{ll}
E_b & \text{Use weather file value if option is \textbf{Beam and diffuse} ($\texttt{irrad\_mode}=0)$}\\
(E_g - E_d)/\cos Z & \text{Calculate for \textbf{Total and diffuse} ($\texttt{irrad\_mode}=2)$}
\end{array}
\right.
\end{equation}

\section{Incident Sky Diffuse Irradiance}

Incident sky diffuse irradiance $I_d$ is solar energy that has been scattered by molecules and particles in the earth's atmosphere before reaching the surface of the subarray.

The sky diffuse model calculates values of the components of the sky diffuse irradiance (isotropic, circumsolar, and horizon brightening), but those components are not used by the photovoltaic model. The equations are provided in the sections below for reference.%???

The \textbf{Tilted Surface Radiation Model} option on the Location and Resource input page in SAM (\texttt{sky\_model} in SSC) allows you to choose from three different sky diffuse irradiance models:

\begin{itemize}
\item The \textbf{Isotropic} model is the simplest of the three models, and assumes that diffuse radiation is uniformly distributed across the sky, called isotropic diffuse radiation ($\mathtt{sky\_model} = 0$).
\item \textbf{HDKR}, named for the algorithm developed by Hay, Davies, Klucher and Reindl in 1979, also assumes isotropic diffuse radiation, but accounts for the higher intensity of circumsolar diffuse radiation, which is the diffuse radiation in the area around the sun ($\mathtt{sky\_model} = 1$).
\item The \textbf{Perez} model uses a more complex computational method than the other two methods that accounts for both isotropic and circumsolar diffuse radiation, as well as horizon brightening ($\mathtt{sky\_model} = 2$). 
\end{itemize}

The \textbf{Radiation Components} option on the Location and Resource input page (\texttt{irrad\_mode} in SSC) allows you to choose whether to use the diffuse horizontal irradiance from the weather file, or whether to calculate it from the global horizontal and beam normal irradiance values:
%lib_irradproc.cpp 914
\begin{equation}\label{eqn-diffhoriz}
E_d = \left\{
\begin{array}{ll}
E_d & \text{Use weather file value if option is \textbf{Beam and diffuse} ($\texttt{irrad\_mode}=0)$}\\
E_g - E_b\cos Z & \text{Calculate for \textbf{Total and beam} ($\texttt{irrad\_mode}=1)$}
\end{array}
\right.
\end{equation}

\subsection{Isotropic Model}

The isotropic diffuse sky irradiance \citep{liu1963}:
%lib_irradproc 521
\begin{equation}
I_d = E_d\frac{1 + \cos\beta_s}{2}
\end{equation}

The isotropic model does not account for circumsolar diffuse irradiance or horizon brightening:
\begin{align}
D_i &= I_d \notag\\
D_c &= 0 \notag\\
D_h &= 0
\end{align}

\subsection{HDKR Model}

The Hay, Davies, Klucher, Reindl (HDKR) sky diffuse model is from Chapter 2.16 of \citet{duffie2013} and \citet{reindl1988}.

The total irradiance incident on a horizontal surface:
%lib_irradproc.cpp 465
\begin{equation}
I_{\mathrm{gh}} = I_{\mathrm{bh}} + E_d
\end{equation}

The ratio of incident beam to horizontal beam:
%lib_irradproc.cpp 469
\begin{equation}
R_b = \frac{\cos\AOI}{\cos Z}
\end{equation}

The anisotropy index for forward scattering circumsolar diffuse irradiance depends on the beam horizontal irradiance $I_{\mathrm{bh}}$ and extraterrestrial irradiance $H$ (Equation~\ref{eqn-hextra}):
%lib_irradproc.cpp 470
\begin{equation}
A_i = \frac{I_{\mathrm{bh}}}{H}
\end{equation}

The modulating factor for horizontal brightening correction:
%lib_irradproc.cpp 471
\begin{equation}
f =\sqrt{\frac{I_{\mathrm{bh}}}{I_{\mathrm{gh}}}}
\end{equation}

The horizon brightening correction factor:
%lib_irradproc.cpp 472
\begin{equation}
s =\sin^3\frac{\beta_s}{2}
\end{equation}

The circumsolar, isotropic, and horizon brightening components of the diffuse sky radiation are then:
%lib_irradproc.cpp 477
\begin{align}
\mathit{cir} &= E_d A_i R_b \notag\\
\mathit{iso} &= E_d \left( 1 - A_i \right) \frac{1 + \cos \beta_s}{2} \notag\\
\mathit{isohor} &= \mathit{iso} \left( 1 + f s \right)
\end{align}

The isotropic, circumsolar, and horizon brightening components of the incident diffuse irradiance:
\begin{align}
D_i &= \mathit{iso} \notag\\
D_c &= \mathit{cir} \notag\\
D_h &= \mathit{isohor} - \mathit{iso}
\end{align}

The HDKR sky diffuse irradiance:
%lib_irradproc.cpp 482
\begin{equation}
I_d = \mathit{isohor} + \mathit{cir}
\end{equation}

\subsection{Perez 1990 Model}\label{sec-perez}

SAM's Perez sky diffuse irradiance model was adapted from PVWatts Version 1 \citep{dobos2013a} and is described in \citet{perez1988} and \citep{perez1990}. The SAM (and PVWatts) implementation includes a modification of the Perez model that treats diffuse radiation as isotropic for $87.5\degree \leq Z \leq 90\degree$. See also \citep{pvmcperez} for a general description of the model.

The Perez model differs from the isotropic and HDKR models in that it uses the empirical  coefficients in Table~\ref{tab-perezcoeffs} derived from measurements over a range of sky conditions and locations instead of mathematical representations of the sky diffuse components.

The parameters $a$ and $b$ describe the view of the sky from the perspective of the surface:
%lib_irradproc.cpp 622 656
\begin{align}
a &= \max( 0, \cos \AOI ) \notag\\
b & =\max( \cos85\degree, \cos Z) 
\end{align}

The sky clearness $\epsilon$ with $\kappa=5.534\times10^{-6}$ and the sun zenith angle
 $Z$ in degrees:
%lib_irradproc.cpp 584 647
\begin{equation}\label{eqn-perezepsilon}
\epsilon =\frac{(E_d + E_b ) / E_d + \kappa Z^{3} }{1 +  \kappa Z^{3}}
\end{equation}

The absolute optical air mass with the incidence angle $b$ and sun zenith angle $Z$ in degrees:
%lib_irradproc.cpp 644
\begin{equation}\label{eqn-perezam}
\mathit{AM}_0 = \left[ \cos b + 0.15~( 93.9\degree - Z )^{-1.253} \right]^{-1}
\end{equation}

The sky clearness $\Delta$ assumes an extraterrestrial irradiance value of $1,367.0~\mathrm{W/m^2}$:
%lib_irradproc.cpp 645
\begin{equation}
\Delta = E_d \frac{\mathit{AM}_0}{1367}
\end{equation}

The coefficients $F_1$ and $F_2$ are empirical functions of the sky clearness $\epsilon$ and describe circumsolar and horizon brightness, respectively. The sun zenith angle $Z$ is in radians:
%lib_irradproc.cpp 653
\begin{align}\label{eqn-perezF1F2}
F_1 & = \max \left[ 0,  \left( f_{\mathrm{11}}(\epsilon) + \Delta f_{\mathrm{12}}(\epsilon) + Z f_{\mathrm{13}}(\epsilon) \right) \right ] \notag\\
F_2 &= f_{\mathrm{21}}(\epsilon) + \Delta f_{\mathrm{22}}(\epsilon) + Z f_{\mathrm{23}}(\epsilon)
\end{align}

SAM uses a lookup table with empirical values shown in Table~\ref{tab-perezcoeffs} to determine the value of the $f$ coefficients in Equation~\ref{eqn-perezF1F2} for a given sky clearness coefficient $\epsilon$ from Equation~\ref{eqn-perezepsilon}.

%lib_irradproc.cpp 570
\begin{table}
\begin{center}
\caption{Perez Sky Diffuse Irradiance Model Coefficients}
\begin{tabular}{l|...|...}
\hline
\multicolumn{1}{c|}{} & 
\multicolumn{1}{c}{$f_{\mathrm{11}}$}  & 
\multicolumn{1}{c}{$f_{\mathrm{12}}$} & 
\multicolumn{1}{c|}{$f_{\mathrm{13}}$} & 
\multicolumn{1}{c}{$f_{\mathrm{21}}$} & 
\multicolumn{1}{c}{$f_{\mathrm{22}}$} & 
\multicolumn{1}{c}{$f_{\mathrm{23}}$} \\
\hline
$\epsilon \leq 1.065$ & -0.0083117 & 0.5877285 & -0.0620636 & -0.0596012 & 0.0721249 & -0.0220216\\
$\epsilon \leq 1.23$ & 0.1299457 & 0.6825954 & -0.1513752 & -0.0189325 & 0.065965 & -0.0288748 \\
$\epsilon \leq 1.5$  & 0.3296958 & 0.4868735 & -0.2210958 & 0.055414 & -0.0639588 & -0.0260542 \\
$\epsilon \leq 1.95$ & 0.5682053 & 0.1874525 & -0.295129 & 0.1088631 & -0.1519229 & -0.0139754 \\
$\epsilon \leq 2.8$ & 0.873028 & -0.3920403 & -0.3616149 & 0.2255647 & -0.4620442 & 0.0012448 \\
$\epsilon \leq 4.5$  & 1.1326077 & -1.2367284 & -0.4118494 & 0.2877813 & -0.8230357 & 0.0558651 \\
$\epsilon \leq 6.2$ & 1.0601591 & -1.5999137 & -0.3589221 & 0.2642124 & -1.127234 & 0.1310694 \\
$\epsilon > 6.2$ & 0.677747 & -0.3272588 & -0.2504286 & 0.1561313 & -1.3765031 & 0.2506212 \\
\hline
\end{tabular}
\label{tab-perezcoeffs}
\end{center}
\end{table}

The isotropic, circumsolar, and horizon brightening components of the sky diffuse irradiance:
\begin{equation}
\begin{array}{ll}
D_i = E_d (1-F_1) \frac{1+\cos\beta}{2} & \text{if $0\degree \leq Z \leq 87.5\degree$} \notag\\
D_c = E_d F_1 \frac{a}{b}&\notag\\
D_h = E_d F_2 \sin\beta &
\end{array}
\begin{array}{ll}
D_i =\frac{1+\cos\beta}{2} & \text{if $87.5\degree < Z < 90\degree$}~~\text{(isotropic only)} \notag\\
D_c = 0 &\\
D_h = 0 &
\end{array}
\end{equation}

The Perez incident diffuse irradiance:
\begin{equation}
I_d = D_i + D_c + D_h
\end{equation}

\section{Incident Ground-reflected Irradiance}\label{sec-groundreflected}

The incident ground-reflected irradiance is solar energy that reaches the array surface after reflecting from the ground. The ground reflects light diffusely, so the ground-reflected irradiance is diffuse irradiance. It is a function of the beam normal irradiance and sun zenith angle, sky diffuse irradiance, and ground reflectance (albedo) \citep{liu1963}:
%lib_irradproc.cpp 522
\begin{equation}
I_{r}= \mathit{\rho} \left( E_b \cos Z + E_d \right) \frac{( 1 - \cos\beta )}{2}
\end{equation}

The albedo is either the set of twelve user-specified values from the Location and Resource input page, or the value from the weather file, depending on whether the option on the Location and Resource input page \textbf{Use albedo in weather file if it is specified} is checked ($\texttt{use\_wf\_albedo=1}$ in SSC) and there is albedo data in the weather file. NSRDB TMY3 weather files typically include albedo values, while the other formats do not. Otherwise ($\texttt{use\_wf\_albedo=1}$ in SSC), SAM uses the twelve monthly albedo values from the Location and Resource input page ($\texttt{albedo}$ in SSC), assuming that the albedo is constant over a single month.

%%%%%%%%%%%%%%%%%%%%%%%%%%%%%%%%%%%%%%%%%%%%%%%%%%
%%%%%%%%%%%%%%%%%%%%%%%%%%%%%%%%%%%%%%%%%%%%%%%%%%
\chapter{Effective POA Irradiance}\label{sec-effectiveirradiance}

The effective, or plane-of-array (POA) irradiance is the incident irradiance less losses due to near-object shading, self shading, and soiling. A set of user-specified adjustment factors represents near-object shading and soiling, and must be generated outside of SAM, either by separate computer models, or by measuring equipment and its associated software. SAM calculates the self-shading factors (Section~\ref{sec-selfshad}) for fixed arrays and for subarrays with one-axis tracking.

\begin{table}
\begin{center}
\caption{Effective POA Irradiance Variable Definitions}
\begin{tabular}{lll}
\midrule
Symbol & Description / \textbf{Name in SAM} & Name in SSC\\
\midrule
\multicolumn{3}{c}{Inputs}\\
$I_b$ & incident beam irradiance (W/m$^2$)  & \texttt{poa\_beam}\\
$I_d$ & incident sky diffuse irradiance (W/m$^2$)& \texttt{poa\_skydiff}\\
$I_r$ & incident ground-reflected irradiance (W/m$^2$)& \texttt{poa\_gnddiff}\\
$L_h$ & time series beam shading loss percentages (\%) & \texttt{poa\_beam}\\
$L_{mh}$ & hour-by-month beam shading loss (\%)& \texttt{subarray[\textit{n}]\_shading:mxh}\\
$L_{azal}$ & sun azimuth-by-altitude beam shading loss  (\%)& \texttt{subarray[\textit{n}]\_shading:azal}\\
$L_{dns}$ & near object sky diffuse shading loss (\%)  & \texttt{subarray[\textit{n}]\_shading:diff}\\
$L_{soilng}$ & monthly soiling loss (\%)  & \texttt{subarray[\textit{n}]\_soiling}\\
\midrule
\multicolumn{3}{c}{Outputs}\\
$I$ & \textbf{...Nominal POA total irradiance} (W/m$^2$)& \texttt{...poa\_nom}\\
$G_{g,shad}$ & \textbf{...POA total irradiance after shading only} (W/m$^2$)& \texttt{...poa\_shaded}\\
$G_b$ & \textbf{...POA beam irradiance after shading and soiling} (W/m$^2$)& \texttt{...poa\_eff\_beam}\\
$G_d$ & \textbf{...POA diffuse irradiance after shading and soiling} (W/m$^2$)& \texttt{...poa\_eff\_skydiff}\\
$G$ & \textbf{...POA total irradiance after shading and soiling} (W/m$^2$)& \texttt{...poa\_eff}\\
$S_{bns}$ & \textbf{...Beam irradiance shading factor} & \texttt{...beam\_shading\_factor}\\
$O$ & \textbf{...Soiling loss} & \texttt{...soiling\_derate}\\
\midrule
\multicolumn{3}{l}{Ellipses (...) indicate \textbf{Subarray [\textit{n}]} in SAM, and \texttt{hourly\_subarray[\textit{n}]\_} in SSC}\\
\multicolumn{3}{l}{For self-shading variable definitions, see Section~\ref{sec-selfshad}}\\
\end{tabular}
\label{tab-effectiveirradiancevars}
\end{center}
\end{table}

The effective irradiance is the solar energy that reaches the top of the module cover. The module model (Section~\ref{sec-module}) accounts for the effect of the module cover on the energy that reaches the photovoltaic cell including angle-of-incidence and reflection losses. Each module model uses a different approach to calculating these losses.

\section{Nominal Global Incident Irradiance}\label{sec-nominalincidentirradiance}

The nominal global incident irradiance $I$ in $\mathrm{kW/m^2}$ is the sum of the incident beam irradiance, incident sky diffuse irradiance, and incident ground-reflected irradiance:
%cmod_pvsamv1.cpp 1226
\begin{equation}
I = I_{b} + I_{d} + I_{r}
\end{equation}

\section{Near Object Shading}\label{sec-nearobjectshad}

%pvsamv1.cpp  1240
Near-object shading is a reduction in the POA incident irradiance by objects near the array, such as buildings, poles, trees, hills, etc. Near-object shading reduces both the beam and diffuse POA irradiance.

SAM represents the reduction in beam POA irradiance using a set of beam shading losses. A beam shading loss is a percentage that represents a reduction in the POA beam irradiance for a given time step. A shading loss of zero represents no shading. A shading loss of 100\% represents complete shading.

SAM stores a set of shading losses for each subarray. The losses are specified in the Edit Shading Data window, which opens from the Shading input page when you click the \textbf{Edit Shading} button for a subarray.

SAM reads beam shading losses from three different lookup tables that make it possible to use shading data from different sources, such as shade analysis tools and modeling software:
\begin{itemize} 
\item \textbf{Time series} is a column of shading losses, one for each time step of the year (8760 for hourly simulations).
\item \textbf{Hour by month} is a 24-by-12 table of 288 shading losses with one value for each hour of the day by month. For each month, SAM applies the same shading loss to a given hour of the day.
\item \textbf{Solar azimuth by altitude} is a table of shading losses for a range of sun azimuth and altitude angles. SAM uses bilinear interpolation to calculate the shading loss for a given time step based on the solar position angles and four shading losses from the table.
\end{itemize}

SAM's user interface provides access to conversion functions that allow for importing shading data from files created by either the PVsyst, Solmetric SunEye, or SolarPathfinder\texttrademark~software as described in \citet{help-shading}. The conversion functions translate the data from the files into the appropriate table listed above. These functions are not available in SSC.

The shading loss inputs are expressed as percentages. SAM converts each percentage to a factor using the equation:
\begin{equation}
S=1-\frac{L}{100\%}
\end{equation}

The beam near-object shading factor for a given time step is the product of the values in the three beam shading loss factors:
%pvsamv1.cpp 1235
\begin{equation}\label{eqn-sbns}
S_{bns}=S_{hr}~S_{mh}~S_{azalt}
\end{equation}

The sky diffuse near-object shading factor $S_{dns}$ is a single value that represents a reduction in the POA diffuse irradiance for the entire year.

\subsection{3D Shade Calculator}\label{sec-3dshad}

The 3D shade calculator calculates a set of beam and diffuse shade factors from a three-dimensional representation of the photovoltaic array and nearby shading objects. Objects in the 3D scene are represented by a set of polygons whose coordinates are specified in 3D space (x,y,z).  Polygons associated with the photovoltaic array or modules are marked as ``active surfaces'', so that the calculator can determine for which surfaces shading and blocking should be evaluated.  The geometry system uses a right-handed coordinate system in which $+x$ points east, $+y$ points north, and $+z$ points normal out of the ground towards the sky. 

The interactive 3D editor, shown in Figure~\ref{fig:3dscene}, allows for creating a complex scene from a set of predefined objects: boxes, cylinders, trees, roof panes, and photovoltaic surfaces.  Each object is represented in 3D by a set of planar polygons that are generated automatically from the object's set of properties that define its position, size, and shape.  

\begin{figure}
\begin{center}
\scalebox{0.7}{\includegraphics{figures/3dscene.png}}
\caption{3D scene editor for shading calculations}
\label{fig:3dscene}
\end{center}
\end{figure}

Active surfaces representing photovoltaic modules can be grouped together logically to represent a string or subarray.  When calculating the time series shade loss fractions at hourly or subhourly time steps, the losses for each string or subarray are reported independently.  If no strings or subarrays are specified in the 3D scene editor, the total active surface shade loss is reported.  This allows the 3D editor to be used for SAM's simple PVWatts performance model, or for solar water heating applications, in addition to the detailed photovoltaic model described here.

The output from the shade calculator is a array of shading factors for each subarray in the system.  If the nonlinear shading loss lookup table model, discussed subsequently, is enabled, string level losses are calculated for each subarray.  The shade calculations are performed entirely in the SAM user interface, and the time series shade loss arrays are passed to the SSC model for the PV system performance calculations.

The calculation procedure for linear shading losses for each active surface at each time step is as follows:

\begin{enumerate}
\item Calculate the solar zenith and azimuth angles for the current time stamp (month, day, hour, minute) and location (latitude, longitude, time zone).
\item Create a list of polygons from the scene objects in untransformed 3D space.
\item Calculate the axis $x$, $y$, and $z$ rotation angles corresponding to a rotation of the scene associated with the current sun position.  In other words, the 3D scene is rotated so that it is shown as if it were observed from the sun.
\item Using this 3D rotation matrix, calculate the rotated coordinates of each polygon vertex to transform the whole scene into the line of sight from the sun using a parallel projection.  This assumes the sun is infinitely far away and that the sun rays reaching the scene are essentially parallel to one another.
\item Using a backface culling algorithm, eliminate polygons from the scene that face away from the sun and are not visible.
\item Apply the binary space partitioning (BSP) algorithm to sort all the remaining visible polygons in back-to-front order.  Polygons at the ``front'' are closest to the sun.
\item Discard the transformed $z$ coordinate to effectively ``flatten'' the scene into two dimensions, as if it were viewed from the infinitely far away sun.
	\item For each active surface:
	\begin{enumerate}
	\item Create a list of polygons (regular and other active surfaces) that are in front of the active surface, that is, between it and the sun.
	\item Using a 2D polygon clipping algorithm, determine the intersection between the active surface and the list of potential blocking polygons.
	\item The resulting intersection polygon's area divided by the active surface polygon area gives the linear shading fraction for that active surface at the current time (sun position).
	\end{enumerate}
\item If subarrays and strings are specified, determine the appropriate shade fractions for each piece from the sum total of the intersected (shaded) areas divided by the sum total of the unshaded active surface areas associated with each string or subarray.   If no strings or subarrays are specified in the geometry, the total intersected (shaded) area divided by the total unshaded active surface area is reported as the overall average shading fraction.
\end{enumerate}

Once the linear shade fraction for each portion of the photovoltaic array is known at each time step, it is converted into a loss percentage and automatically entered into the SAM beam irradiance shading loss table in the user interface. These values are passed directly to the SSC simulation model for calculating system electricity generation.

For systems with microinverters or module level power electronics such as DC-DC optimisers, the linear shading losses are appropriate estimates of overall energy lost due to shading.  However, for systems with strings of modules connected in parallel to a single inverter maximum power point tracking input, irregular obstruction shading can result in additional losses due to limiting effects associated with mismatch among the strings.  For this situation, MacAlpine~\cite{macalpine2015} proposed a computationally efficient method to estimate the nonlinear impact of partially shaded strings without requiring a full bottoms-up electrical cell-by-cell model.  This model is enabled by selecting the ``database lookup'' option in the time series beam shading table dialog in SAM.  In brief, the model uses a precalculated lookup table that, given: 1) number of parallel strings, 2) linear shade fraction on each string, and 3) fraction of diffuse irradiance on each string, estimates the total DC power loss due to the irregular shading on parallel strings.  The trivariate lookup table is built into the SAM model and can handle up to eight parallel strings.  Refer to~\cite{macalpine2015} for additional details on this model.

The 3D shading calculator also estimates the loss of available diffuse irradiance to the active photovoltaic surfaces due to blocking of the sky dome by obstructions.  The procedure is as follows:
\begin{enumerate}
\item Divide the hemispherical sky dome into sections of some number of azimuth and zenith angle divisions.  By default, SAM uses 1 degree increments in both directions, for a total of 360 * 90 = 32400 divisions.
\item For each position, transform the scene to the solar azimuth and zenith angle.
\item Calculate the intersected (shaded) areas on each active surface polygon using the same procedure as for the beam irradiance shading loss.
\item Integrate the shade loss over all positions in the sky dome by multiplying the observed shaded fraction with the solid angle of spherical integration $\sin(\theta_z)$.  This accounts for the fact that the sky dome divisions at the top of the hemisphere are much smaller than near $\theta_z\approx 0$.
\end{enumerate}

This procedure results in an estimate of the view factor of the active surfaces in each subarray to the sky dome, and hence the effective diffuse irradiance loss.  The model assumes an isotropic diffuse sky, and does not adjust for increased circumsolar diffuse irradiance.

\section{Self Shading}\label{sec-selfshad}

SAM can model the self shading that occurs when photovoltaic modules are arranged in parallel rows, and modules in one row cause a shadow on modules in a neighboring row. SAM can model self shading for fixed subarrays (no tracking) and subarrays with one-axis tracking. SAM does not model self shading for subarrays with two-axis or azimuth-axis tracking.

To enable the self shading model for a subarray with no tracking (fixed) or one-axis tracking in SAM, after specifying the tracking and orientation options and ground coverage ratio (GCR) on the System Design page, on the Shading and Snow page, for the subarray's \textbf{Self shading type} input, choose \textbf{Standard (Non-linear)} for a subarray with crystalline silicon photovoltaic cells, or choose \textbf{Thin film (Linear)} for a subarry with thin-film modules. SAM does not automatically choose the correct self-shading option based on the module properties. For a subarray with one-axis tracking, you may also want to enable backtracking  (Section~\ref{sec-backtrack1x}) on the System Design page.

In SSC, to enable self shading, set the \texttt{pvsamv1} input variable $\texttt{subarray[\textit{n}]\_shade\_mode}=1$ for the non-linear option, and $\texttt{subarray[\textit{n}]\_shade\_mode}=2$ for the linear option. Set the value to zero to ignore self shading.

The non-linear self-shading option is for modules with mono- or poly-crystalline silicon photovoltaic cells, and assumes that modules consist of photovoltaic cells with three bypass diodes so that a shaded module's output decreases in steps of one third of the module's total output. The linear option is for thin-film modules, and assumes that a shaded module's output decreases linearly with the portion of the module that is shaded. 

When you use POA irradiance data in the weather file, SAM must run a POA decomposition model to estimate the POA beam irradiance. If the measured POA data already accounts for the effect of self shading, then you should not enable self shading in SAM.

The self-shading algorithm is described in detail in Section~\ref{sec-selfshadalg}.

The self-shading algorithm calculates three factors for the non-linear option:
\begin{itemize}
\item Sky diffuse self-shading factor $S_{dss}$ that reduces the POA diffuse irradiance
\item Ground-reflected diffuse self-shading factor $S_{gss}$ that also reduces the POA  diffuse irradiance
\item DC self-shading factor $F_{dcss}$ that accounts for the effect of self shading on the POA beam irradiance by reducing DC electrical output of the subarray
\end{itemize}

The self-shading algorithm calculates a single factor for the linear option:
\begin{itemize}
\item Linear self-shading factor $S_{bss}$ that reduces the POA beam irradiance
\end{itemize}

\section{Effective Irradiance after Shading Only}

SAM reports the effective irradiance after shading only (and before soiling) for reference. It is an intermediate value that is not used in any calculations.

The beam, sky diffuse, and ground-reflected components of the effective irradiance after shading only are:
\begin{align}
G_{bshad} &= I_b~S_{bns}~S_{1xss}~S_{bss}\notag\\
G_{dshad} &= I_d~S_{dns}~S_{dss} \notag\\
G_{rshad} &= I_r~S_{rss}
\end{align}

Where the self shading loss factors apply as follows:
\begin{align*}
S_{1xss}&~\text{non-linear option with one-axis tracking}\\
S_{bss}& ~\text{linear option}\\
S_{dss}& ~\text{non-linear option with fixed subarray}\\
S_{rss}& ~\text{non-linear option with fixed subarray}
\end{align*}

The global effective irradiance after shading only:
\begin{equation}
G_{shad} = G_{bshad} + G_{dshad} + G_{rshad}
\end{equation}

\section{Soiling}\label{sec-soiling}

Soiling losses are caused by a reduction in the incident irradiance due to dust and dirt on the module surface. SAM accounts for soiling losses using a set of monthly soiling loss percentages that are user inputs on the Losses input page (\texttt{subarray[\textit{n}]\_soiling} in SSC). The soiling loss applies to both the beam and diffuse components of the incident irradiance.

SAM converts each soiling loss percentage to a factor:
\begin{equation}
O=1-\frac{L}{100\%}
\end{equation}

For each time step of each month, SAM applies that month's soiling factor $O$ to all components of the incident irradiance.

\section{Effective Irradiance after Shading and Soiling}

The effective irradiance after shading and soiling is the irradiance value that the module models use as input (see Section~\ref{sec-module}).

The components of the effective irradiance after shading and soiling are, where $O$ is the soiling factor:
\begin{align} \label{eqn-effectivebeamdiffuse}
G_b &= G_{bshad}~O \notag\\
G_d &= G_{dshad}~O\notag\\
G_r &=  G_{rshad}~O
\end{align}

SAM reports the global effective irradiance $G$ and the total diffuse effective irradiance $G_{d,t}$ in the results. (Note that the module models use the component values in Equation~\ref{eqn-effectivebeamdiffuse} as input rather than the global value.):
\begin{align}
G &= G_b + G_d + G_r\notag\\
G_{d,t} & = G_d + G_r
\end{align}

SAM also reports the effective beam shading factor $S_b$ in the results ($S_{bns}$) is the near-object shading factor from Equation~\ref{eqn-sbns}:
%cmod_pvsamv1.cpp 1336
\begin{equation}
S_b = \left\{
\begin{array}{ll}
S_{bns}~S_{ss}~O &\text{for fixed self shading}\\
S_{bns}~O &\text{for one-axis self shading} 
\end{array}
\right.
\end{equation}

%%%%%%%%%%%%%%%%%%%%%%%%%%%%%%%%%%%%%%%%%%%%%%%%%%
%%%%%%%%%%%%%%%%%%%%%%%%%%%%%%%%%%%%%%%%%%%%%%%%%%
\chapter{Self Shading Algorithm}\label{sec-selfshadalg}

%cmod_pvsamv1.cpp 1039: sscalc is a class of type selfshade_t, which is defined
%in lib_pvshade.h 81 and implemented in lib_pvshade.cpp 153.
This section describes SAM's self-shading model. See Section~\ref{sec-selfshad} for a general description and instructions for enabling the model. The algorithm is presented here as a separate for clarity because of the length of the algorithm.

The non-linear option of the self-shading model is intended for modeling systems with  crystalline silicon modules and assumes that blocking diodes cause the module's response to partial shading of the module to be non-linear. The non-linear option assumes that each module contains three blocking diodes ($d=3$) and characterizes the non-linear response with an empirically determined factor applied to the subarray's electrical output and a reduction in POA diffuse irradiance.

The linear model option is intended for systems with thin-film photovoltaic modules, and assumes that the output of a partially shaded module responds linearly with a reduction in POA irradiance. The linear option characterizes effect of self shading on the module as a reduction in the POA beam irradiance equal to the fraction of the module surface that is shaded.

%self shading inputs ssinputs pvsamv1 update 76.
\begin{table}
\begin{center}
\caption{Self Shading Variable Definitions}
\begin{tabular}{lll}
\midrule
Symbol & Description / \textbf{Name in SAM} & Name in SSC\\
\midrule
\multicolumn{3}{c}{Inputs}\\
$M_{total}$ & \textbf{Modules in subarray} & -\\
$M_{string}$ & Number of modules per string & -\\
$M_{side}$ & \textbf{Number of modules along side of row} & \texttt{subarray[\textit{n}]\_nmody}\\
$M_{bottom}$ & \textbf{Number of modules along bottom of row} & \texttt{subarray[\textit{n}]\_nmodx}\\
$N_{strings}$ & Number of strings in parallel & \texttt{strings\_in\_parallel}\\
$\sigma_{\mathit{lim}}$ & \textbf{Tracker rotation limit} & \texttt{subarray\textit{n}\_rotlim} \\
$\GCR$ & \textbf{GCR}, ground coverage ratio & \texttt{subarray[\textit{n}]\_gcr}\\
$A_m$ & module area (m$^2$)& \texttt{[spe,cec,6par,snl]\_area}\\
$\beta_s$ & surface tilt angle (deg)& \texttt{subarray[\textit{n}]\_tilt} \\
$\gamma_s$ & surface azimuth angle (deg)& \texttt{subarray[\textit{n}]\_azimuth} \\
$Z$ & sun zenith angle & -\\
$\gamma$ & sun azimuth angle (deg)& - \\
$E_b$ & direct normal irradiance (W/m$^2$) & - \\
$G_b$ & POA beam irradiance (W/m$^2$)& - \\
$G_d$ & POA diffuse irradiance (sky and ground) (W/m$^2$)& - \\
$\rho$ & albedo*& \texttt{albedo} \\
\midrule
\multicolumn{3}{c}{Outputs: Non-linear option}\\
$S_{dss}$ & sky diffuse factor  / \textbf{Self-shading diffuse derate}& \texttt{[...]\_ss\_diffuse\_derate}\\
$S_{rss}$ & ground diffuse factor  / \textbf{Self-shading reflected derate}& \texttt{[...]\_ss\_reflected\_derate}\\
$F_{dcss}$& DC loss factor / \textbf{Self-shading derate} & \texttt{[...]\_ss\_derate}\\
\midrule
\multicolumn{3}{c}{Outputs: Linear option}\\
$S_{bss}$ & beam diffuse factor  / \textbf{Self-shading linear derate}& \texttt{[...]\_ss\_linear\_derate}\\
\midrule
\multicolumn{3}{l}{Ellipses (...) indicate \texttt{subarray[\textit{n}]}.}\\
\multicolumn{3}{l}{For the non-linear option, SAM sets the linear output to 1.}\\
\multicolumn{3}{l}{For the linear option, SAM sets the non-linear outputs to 1.}\\
\multicolumn{3}{l}{*Albedo may be from weather file or set to default value of 0.2 if file does not contain albedo data.}\\
\end{tabular}
\label{tab-selfshadinputs}
\end{center}
\end{table}

The self-shading model calculates different loss factors depending on the shade model option, and whether the array is fixed or uses one-axis tracking:

\begin{itemize}
\item For a subarray with the non-linear self shading option:
\begin{enumerate}
\item Reduction in POA sky diffuse and POA ground-reflected diffuse irradiance.
\item Reduction in the subarray's DC output to account for the effect of self shading on POA beam irradiance.
\end{enumerate}
\item For a subarray with the the linear self shading option:
\begin{enumerate}
\item Reduction in POA beam irradiance
\end{enumerate}
\end{itemize}

Table~\ref{tab-selfshadinputs} and Figure~\ref{fig-selfshaddimensions} show the variables used to calculate the fixed self-shading factors. The equations and algorithm are described in \citep{appelbaum1979} and \citep{deline2013a}.
%lib_pvshade.cpp 153 selfshade_t::init, also see cmod_pvsamv1.cpp 250 and 1047

\section{Self Shading Assumptions}\label{sec-selfshadassumptions}

\begin{figure}
\begin{center}
\includegraphics[scale=0.33]{self-shading-shadow-dimensions}
\caption{Shadow Dimensions for Portrait Module Orientation}
\label{fig-selfshaddimensions}
\end{center}
\end{figure}

The self-shading model makes a set of simplifying assumptions to minimize the number of inputs:

\begin{itemize}
\item Modules have an aspect ratio of $R_{aspect}=1.7$, so that the length of the long side is 1.7 times the length of the short side. 
\item The module area $A$ is the product of the module's total length and width. The value may be from one of the module libraries or a user input.
\item For the non-linear option, each module has three bypass diodes $d=3$ so that the module is made up of three ``submodules." A submodule is a string of photovoltaic cells in the module protected by a single bypass diode. For example, a 60-cell module would consist of three submodules of 20 cells each.
\item A partially shaded submodule behaves in the same way as it would if the entire submodule were uniformly shaded.
\item Modules are wired together in strings parallel to the ground (horizontal strings) so that the number of modules along the bottom of a subarray row is a multiple of the number of modules per string.
\item Parallel strings in the subarray are uniformly shaded.
\end{itemize}

Based on the simplifying assumptions listed above, the number of rows $N_{rows}$ in a subarray is:
%pvsamv1 update 767
\begin{equation}
N_{rows} = \left\lfloor \frac{M_{total}}{M_{side} M_{bottom}} \right\rfloor
\end{equation}

where the total number of modules in the array $M_{total}$ is:

\begin{equation}
M_{total}=M_{string}~N_{strings}
\end{equation}

If $M_{total}$ is not an even multiple of the product $M_{side}~N_{bottom}$, SAM generates a simulation message indicating that the self-shading configuration does not match the subarray configuration, but still calculates the DC loss factor as described below.

SAM calculates the module length and width from the module area $A$ in $\text{m}^2$  and the fixed aspect ratio assumption  $R_{aspect}=1.7$:
%pvsamv1 update 1111
\begin{align}
W &= \sqrt{ \frac{A_m}{R_{aspect}}}\notag\\
L &= W \mathit{R_{aspect}}
\end{align}

The module area is either from the module library, or an input, depending on the module model (see Section~\ref{sec-moduleoptions}).

The length of the side of a row (distance from the bottom edge to the top edge of a row):
%pvsamv1 update 1119
%this is Deline B = Applebaum A
\begin{equation}
B=
  \left\{
    \begin{array}{ll}
      M_{side}~L & \mbox{portrait module orientation}\\
      M_{side}~W & \mbox{landscape module orientation}
    \end{array}
  \right.
\end{equation}

The distance between bottom edges of neighboring rows:
%pvsamv1 update 1123
\begin{equation}
R = \frac{B}{\GCR} 
\end{equation}

where $\GCR$ is the ground coverage ratio input.

The sun altitude angle:
\begin{equation}
\alpha = 90^\circ-Z
\end{equation}

%%%%%%%%%%%%%%%%%%%%%%%%%%%%%%%%%%%%%%%%%%%%%%%%%%
\section{Non-linear Option: Diffuse POA Irradiance Reduction} \label{sec-selfshaddiffuse}

Some diffuse radiation is always blocked by neighboring rows of modules, regardless of whether the rows block beam radiation \citep{goswami1989}. The diffuse radiation consists of a sky diffuse component and a ground-reflected component, both of which are affected in this way.

For the non-linear option, for the purposes of self-shading calculations, SAM assumes isotropic sky diffuse irradiance, so that the total horizontal diffuse irradiance $G_{dh}$ is:
%lib_pvshad update 168
\begin{equation}\label{eqn-selfshaddiffhor}
G_{dh} = G_d \left(\frac{2}{1+\cos\beta_s}\right)
\end{equation}

The beam horizontal irradiance is:
%lib_pvshad1 update 169
\begin{equation}
G_{bh} = E_b\cos Z 
\end{equation}

%Equations~\ref{eqn-selfshad-skydiff1}~-~\ref{eqn-selfshad-skydiff2} for calculating the reduction in sky diffuse POA irradiance are from \citep{passias1984}. The diffuse horizontal radiation $G_{dh}$ is from Equation~\ref{eqn-selfshaddiffhor}.

%The front row of modules (row nearest the sun) is not obscured, and the sky diffuse POA %irradiance on that row is a function of the surface tilt angle $\beta_s$:
%\begin{equation}\label{eqn-selfshad-skydiff1}
%G_d(\beta_s)=G_{dh}\cos^2\frac{\beta_s}{2}
%\end{equation}

The reduction in sky diffuse POA irradiance incident on the portion of the subarray shaded by neighboring rows is a function of the surface tilt angle $\beta_s$ and mask angle $\psi$ with a derating term $\frac{N_{rows}-1}{N_{rows}}$ for the number of interior rows in the array:
%lib_pvshad update 172
\begin{equation}\label{eqn-selfshad-skydiff2}
G_{sky,\text{red}} = G_d - G_{dh}\left(1 - \cos^2\frac{\psi}{2}\right)\frac{N_{rows} - 1}{N_{rows}}
\end{equation}

\begin{figure}
\begin{center}
\includegraphics[scale=0.66]{self-shading-mask-angle}
\caption{Side View of Two Rows with Self-shading Mask Angle Variables}
\label{fig-selfshadingmaskangle}
\end{center}
\end{figure}

%mask_angle_calc_method is an input on the Array page
The shade mask angle is the minimum array tilt angle at which the view of the sky at a given point along the side of the row is obstructed by a neighboring row as shown in Figure~\ref{fig-selfshadingmaskangle} from \citet{passias1984}:
%pvsamv1 update 342
\begin{equation}
\psi = \arctan \frac{ (B)\sin\beta_s } { R - B\cos\beta_s }
\end{equation}

SAM uses the "worst case" approach described in \citep{deline2013b}, where the diffuse irradiance mask angle is calculated for the bottom of the array rather than averaged across the entire array.

The sky diffuse shading factor for the non-linear option is then:
%lib_pvshad update 173
\begin{equation}
S_{dss} = \frac{G_{sky,\text{red}}}{G_d}
\end{equation}

%The equations for calculating the reduction in ground-reflected POA irradiance are from \citep{goswami1989}. The diffuse horizontal radiation $G_{dh}$ is from Equation~\ref{eqn-selfshaddiffhor}.

Ground diffuse irradiance is beam irradiance reflected onto modules in the subarray from the ground. The first row in the subarray is not affected by self shading. The length of ground in front of each shaded row that reflects beam radiation onto the row, with $Y$ constrained to a minimum value of 0.00001:
%lib_pvshad update 182
\begin{equation}
Y = R - B \left(\frac{\sin( 180^\circ - \alpha - \beta_s )}{\sin\alpha} \right)
\end{equation}

The view factor on the first row $\mathit{F_1}$, and the beam and diffuse reflected component factors $\mathit{F_2}$ and $\mathit{F_3}$, respectively:
%lib_pvshad update 181, 184-185
\begin{align}
F_1 &= \mathit{\rho} \sin^2\frac{\beta_s}{2} \notag\\
F_2 &= \frac{\mathit{\rho}}{2} \left( 1 + \frac{Y}{B} - \sqrt{ \frac{Y^2}{B^2} - \frac{2Y}{B} \cos(180^\circ - \beta_s) + 1 }\right) \notag\\
F_3 &= \frac{\mathit{\rho}}{2} \left( 1 + \frac{R}{B} - \sqrt{ \frac{R^2}{B^2} - \frac{2R}{B} \cos(180^\circ - \beta_s) + 1 } \right)
\end{align}

The reduced ground-reflected diffuse POA irradiance on the entire subarray:
%lib_pvshad update 188
\begin{equation}
G_{gnd,red}=\left(\frac{F_1+F_2(N_{rows}-1)}{N_{rows}}\right)G_{bh}+\left(\frac{F_1+F_3(N_{rows}-1)}{N_{rows}}\right)G_{dh}
\end{equation}

The ground  diffuse shading factor for the non-linear option is then:
%pvsamv1 update 187-193
\begin{equation}
S_{rss}=
 \left\{
     \begin{array}{ll}
      S_{rss} = \dfrac{G_{gnd,red}}{F_1 (G_{bh} +  G_{dh})}
      & \mbox{if $F_1 (G_{bh} + G_{dh}) > 0$}\\
      1 
      & \mbox{if $F_1 (G_{bh} + G_{dh}) \leq 0$}
    \end{array}
  \right.
\end{equation}

%%%%%%%%%%%%%%%%%%%%%%%%%%%%%%%%%%%%%%%%%%%%%%%%%%
\section{Shadow Dimensions} \label{sec-selfshaddimensions}

The shadow dimensions are defined by a shadow height  $H_S$ and shadow displacement $g$ as shown in Figure~\ref{fig-selfshaddimensions}. The equations are based on \citet{appelbaum1979}.

The shadow height $H_S$ is required to calculate the reduction in POA beam irradiance for both the non-linear (Equations~\ref{eqn-ssdimland} and~\ref{eqn-ssdimport}) and linear  (Equation~\ref{eqn-ssbeam}) options. 

The shadow dimensions equations assume that a surface azimuth angle $\gamma_s$ of zero is for a south-facing surface in the northern hemisphere, and for a north-facing surface in the southern hemisphere. Because this differs from SAM's convention where $\gamma_s=0$ is for a north-facing surface and $\gamma_s=180^\circ$ is for a south-facing surface, the equations use the effective surface azimuth $\gamma_{s,eff}$:
%lib_pvshad update 366
\begin{equation}
\gamma_{s,eff} = \gamma - \gamma_s
\end{equation}
For an array that is not horizontal and for hours when the sun is up, the shaded portion of the array is:
%lib_pvshad update 372
%B defined above is equivalent to A in Applebaum equations
\begin{align} \label{eqn-pxpy}
P_y &= B \left(\cos\beta_s + \cos \gamma_{s,eff} \frac{\sin\beta_s}{\tan(90-Z)}\right)\notag\\
P_x &= B~\sin\beta_s \left(\frac{\sin\gamma_{s,eff}}{\tan(90-Z)}\right)
\end{align}

The shadow displacement $g$:
%lib_pvshad update 386
\begin{equation}\label{eqn-shadowdisplacement}
g=R\frac{P_x}{P_y}
\end{equation}

with the following constraints:
%lib_pvshad update 388
\[
g=
  \left\{
    \begin{array}{ll}
      \lvert g\rvert &  \mbox{when $g<0$}\\
      0 & \mbox{when $P_y=0$}\\
      0 & \mbox{when $M_{bottom}>M_{string}$}\\
      B & \mbox{when $g>B$}
    \end{array}
  \right.
\]

The shadow height $H_s$ in Figure~\ref{fig-selfshaddimensions} is:
%lib_pvshad update 403
\begin{equation}\label{eqn-ssshadowheight}
H_S=
	\left\{
	\begin{array}{ll}
	B\left(1-\frac{R}{P_y}\right)&\mbox{for fixed subarray}\\
	B\left(S_{1xss}\right)&\mbox{for one-axis tracking (see Equation~\ref{eqn-ss1x})}
	\end{array}
	\right.
\end{equation}

with the following constraints: 
\[
H_S=
  \left\{
    \begin{array}{ll}
      0                      & \mbox{when $P_y=0$}\\
      \lvert H_S \rvert & \mbox{when $g<0$}\\
      B                     & \mbox{when $H_s>B$}
    \end{array}
  \right.
\]


\section{Linear Option: POA Beam Irradiance Reduction} \label{sec-selfshadlinear}

For the linear self-shading option, the reduction in POA beam irradiance is proportional to the relative shaded area of the subarray:

%lib_pvshade relative shaded area for linear shading
\begin{equation}\label{eqn-ssbeam}
S_{bss} = 
	\left\{
	\begin{array}{ll}
		H_s\frac{L-g}{B L}&\mbox{for fixed subarray}\\
		S_{1xss}&\mbox{for one-axis tracking (Equation~\ref{eqn-ss1x})}
	\end{array}
	\right.
\end{equation}

%%%%%%%%%%%%%%%%%%%%%%%%%%%%%%%%%%%%%%%%%%%%%%%%%%
\section{Non-linear Option: DC Loss Factor} \label{sec-selfshaddc}

%S and X are inputs to the selfshade_dc_derate() function
For the non-linear shading option, the effect of the reduction in POA beam irradiance is modeled as a reduction in the subarray's electrical DC output, and depends on the number of diodes in each module, which SAM assumes to be three $d=3$.

When there is self shading, some fraction $X$ of each row in the subarray is shaded. In each row, a fraction $S$ of the number of modules along the side is of the row is shaded \citep{deline2013a}. The equations for $S$ and $X$ are from \citet{deline2013b}.


For landscape module orientation, the case where $H_S>W$ assumes complete shading even though in a given string, some modules will be completely shaded while others are only partly shaded ($\lceil...\rceil$ and $\lfloor...\rfloor$ indicate the ceiling and floor functions, respectively): 
%lib_pvshad update 418, 455 situation 1
\begin{equation}
X = \left\lceil\frac{H_S}{W}\right\rceil \dfrac{R-1}{R M_{side}}\notag
\end{equation}
\begin{equation}\label{eqn-ssdimland}
S=
  \left\{
    \begin{array}{ll}
      \left\lceil \frac{H_S d}{W} \right
      \rceil d^{-1} 
      \left(1-\left\lfloor\dfrac{g}{L}\right\rfloor M^{-1}_{bottom}\right) 
      & \mbox{when $H_S\le W$ for a fixed subarray}\\
      1 
      & \mbox{when $H_S> W$, or for one-axis tracking}
    \end{array}
  \right.
\end{equation}

For portrait module orientation, the configuration shown in Figure~\ref{fig-selfshaddimensions}: 
%lib_pvshad update 433, 455  situation 2
\begin{equation}
X = \left\lceil \frac{H_S}{L} \right\rceil \dfrac{R-1}{R M_{side}}\notag
\end{equation}
\begin{equation}\label{eqn-ssdimport}
S=
 \left\{
     \begin{array}{ll}
      1-\left\lfloor \dfrac{gd}{W}\right\rfloor \left(d M_{bottom}\right)^{-1} 
      & \mbox{for a fixed subarray}\\
      1 
      & \mbox{for one-axis tracking}
    \end{array}
  \right.
\end{equation}

%DELETE AFTER REVIEW
%\begin{align}
%S &= 1-\left\lfloor \frac{gd}{W}\right\rfloor \left(d N_{bottom}\right)^{-1} 
%\end{align}

SAM assumes that each module has three bypass diodes so that $d=3$ (See  Section~\ref{sec-selfshadassumptions}).

%vertical strings not available as of SAM 2014.11.24
%For vertical strings and portrait module orientation ($\mathit{sor}=0$, %$\mathit{mor}=0$), this approximation assumes complete shading for modules on the %very edge of the array that may be only partially shaded: %situation 3
%\begin{align}
%X &= 1-\frac{\left\lfloor\lvert \frac{g}{W}\rvert\right\rfloor}{n} \notag\\
%S &= \left\lceil \frac{H_s}{L} \right\rceil \frac{r-1}{mr} 
%\end{align}

%For vertical strings and landscape module orientation ($\mathit{sor}=0$, %$\mathit{mor}=1$): %situation 4 
%\begin{align}\label{eqn-xs-landvert}
%X &= 1-\frac{\left\lfloor\lvert \frac{g}{L}\rvert\right\rfloor}{n} \notag\\
%S &= \left\lceil \frac{H_sd}{W} \right\rceil \frac{r-1}{dmr} 
%\end{align}

To calculate the beam self-shading DC loss factor, SAM first determines the shadow dimensions (Section~\ref{sec-selfshaddimensions}), and then determines the DC loss factor using empirically determined relationships as described below.

The self-shading DC loss factor accounts for the reduction in beam POA irradiance caused by self shading. Its calculation is based on the proposed simplified analytical model from \citep{deline2013a} and requires the following variables:
\begin{itemize}
\item The shading fractions $S$ and $X$ described in Section~\ref{sec-selfshaddimensions}.
\item Fraction of solar irradiance reaching the shaded submodule $E_e$, which is a function of the reduction in diffuse POA irradiance from the sky and ground described in Section~\ref{sec-selfshaddiffuse}.
\item Submodule fill factor $\mathit{F}_{fill}$ determined from the module nameplate parameters.
\end{itemize}

The ratio of incident diffuse irradiance to total incident irradiance is a measure of the solar irradiance reaching the shaded submodule:
%lib_pvshade 469 (dbh_ratio in selfshade_dc_derate parameter list)
\begin{equation}
R_{dt} = \frac{G_{sky,red}+G_{gnd,red}}{G_b+G_{sky,red}+G_{gnd,red}}
\end{equation}

The fill factor depends on the photovoltaic module's rated maximum power, open-circuit voltage, and short-circuit current:
\begin{equation}
F_{fill}=\frac{P_{mp0}}{V_{oc0}I_{sc0}}
\end{equation}

The DC loss factor equation requires three coefficients, which were determined using the empirical analysis described in \citet{deline2013a}:
%lib_pvshade.cpp 263
\begin{align}
C_1 &= (109F_{fill}-54.3)e^{-4.5X} \notag\\
C_2 &= -6X^2+5X+0.28 \quad \text{(if $X>0.65$, set $X=0.65$)}\notag\\
C_3 &= \max\left[(-0.05R_{dt}-0.01)X+(0.85F_{fill}-0.7)R_{dt}-0.0855F_{fill}+0.05,R_{dt}-1\right]
\end{align}

The DC loss factor for fixed self shading is the ratio of the power of the shaded string to the power of the unshaded string. The choice of equation for calculating the DC loss factor equation depends on the magnitude of the shading factors:
\begin{itemize}
\item $F_{dc1}$ \quad Small $X$ and small $S$
\item $F_{dc2}$ \quad Large $X$
\item $F_{dc3}$ \quad Large $S$
\end{itemize}

The DC loss equations for each shade factor condition are:
%lib_pvshade.cpp 267-280
\begin{align}
F_{dc1} &= 1-C_1S^2-C_2S \notag\\ %eqn5
F_{dc2} &= 
\left\{
   \begin{array}{ll}
      \dfrac{X-S\left(1+0.5 d V^{-1}_{mp}\right)}{X} & \text{if $X>0$}\\
      0 & \text{if $X=0$}\\
   \end{array}
\right. \notag\\ %eqn9
F_{dc3} &= C_3\left(S-1\right)+R_{dt} %eqn10
\end{align}

SAM chooses the maximum of the three factors to calculate the system DC loss factor accounting for the reduction in beam POA irradiance due to self shading, and constrains the factor to between $0$ and $1$:
%lib_pvshade.cpp 274
\begin{equation}\label{eqn-selfshadedcderate}
F_{dcss}=X\max\left(F_{dc1},F_{dc2},F_{dc3}\right)+(1-X)
\end{equation}

%%%%%%%%%%%%%%%%%%%%%%%%%%%%%%%%%%%%%%%%%%%%%%%%%%
\section{One-axis tracking: Shaded Fraction}\label{sec_selfshad1x}
%cmod_pvsamv1.cpp 1254

This section describes SAM's self-shading geometry calculations for one-axis trackers $S_{1xss}$ for subarrays with one-axis tracking. SAM uses the shaded fraction to calculate the shadow height (Equation~\ref{eqn-ssshadowheight}) and the reduction in POA beam irradiance for the linear option (Equation~\ref{eqn-ssbeam}).

This model was developed by NREL specifically for SAM, so this section is the only documentation of the model. The source code for these calculations is available for download. For a link to the source code, see the \citet{source} citation in the References section of this manual.

For the purpose of determining the shaded fraction for subarrays with one-axis tracking, SAM  models the subarray as three surfaces, called panels, with the same tilt, azimuth, and rotation angle as the subarray. The three rectangular surfaces have the same arbitrary dimensions. The algorithm determines the position of the four corners of each panel, and then based on the position of the sun determines whether either of the outside panels blocks beam radiation from the sun to the middle panel.

This approach makes it possible to calculate the shaded fraction of the subarray using only the ground coverage ratio (GCR) in addition to the subarray surface angles (Section~\ref{sec-surfaceangles}) and sun angles (Section \ref{sec-sunangles}). The GCR describes the spacing of rows in the subarray, and is defined as the ratio of area of the subarray when its modules are horizontal to the total ground area, which is defined by the edges of modules at the outer boundary of the array.

\subsection{Sun Position Unit Vectors}

The sun position unit vector $\mathbf{\hat s}$ represents the direction of the sun calculated from the sun azimuth angle $\alpha$ and zenith angle $\zeta$ in degrees (Section~\ref{sec-sunangles}).

The solar altitude angle in degrees is:
%lib_irradproc 1100
\begin{equation}
\alpha = 90\degree - \zeta
\end{equation}

The sun position unit vector depends on the solar altitude and azimuth angles:
%lib_irradproc.cpp 1102
\begin{align}
0 & \leq \alpha \leq 90\degree:
&90^\circ&<\alpha\leq180^\circ:
\notag\\
\mathbf{s} &= 
  \left [
    \begin{array}{c}
      \cos\alpha\sin\beta\\
      \cos\alpha\cos\beta\\
      \sin\alpha\\
    \end{array}
  \right]
&\mathbf{s} &= 
  \left [
    \begin{array}{c}
      \cos\alpha\sin(180^\circ-\beta)\\
      -\cos\alpha\cos(180^\circ-\beta)\\
      \sin\alpha\\
    \end{array}
  \right]
\notag\\
180^\circ&<\alpha\leq270^\circ: 
&270^\circ&<\alpha\leq360^\circ: 
\notag\\
\mathbf{s} &= 
  \left [
    \begin{array}{c}
      -\cos\alpha\sin(\beta-180^\circ)\\
      -\cos\alpha\cos(\beta-180^\circ)\\
      \sin\alpha\\
    \end{array}
  \right]
&\mathbf{s} &= 
  \left [
    \begin{array}{c}
      -\cos\alpha\sin(360^\circ-\beta)\\
      \cos\alpha\cos(360^\circ-\beta)\\
      \sin\alpha\\
    \end{array}
  \right]
\end{align}

The unit vector in the direction of the sun is the Euclidean norm of $\mathbf{s}$:
\begin{equation}
\mathbf{\hat s} = \|\mathbf{s}\|
\end{equation}

\subsection{Panel Vertices}

The next step is to find the vertices (corners) of the three rectangular panels. The panels have the same same orientation as the subarray and with spacing defined by the ground coverage ratio.

First, arbitrarily define the panel length and width:
%lib_irradproc.cpp 983 985
\begin{align}\label{eqn_paneldim}
W&=1\notag\\
L&=10
\end{align}

These panel dimensions make it possible to express the distance between rows $D$ in terms of the ground coverage ratio $\mathit{gcr}$, defined as the ratio between the module area $A_m$ and ground area $A_g$:
%lib_irradproc.cpp 984
\begin{equation}
D=\frac{1}{\mathit{GCR}}-1
\end{equation}

%\begin{align}\label{eqn_gcr}
%A_p &=3LW \notag\\
%A_g &= L(3W+3D) \notag\\
%\mathit{gcr} &= \frac{A_m}{A_g} = \frac{W}{W+D}
%\end{align}

When each of the three panels is in the horizontal position, the four vertices $(x,y,z)$ of the three panels are:
\begin{align}
\mathbf{P}_{0,horiz} &= 
  \left [
    \begin{array}{ccc}
      W/2&0&0\\
      W/2&L&0\\
      -W/2&L&0\\
      -W/2&0&0\\
    \end{array}
  \right] 
    \begin{array}{ccc}
      &&\text{Vertex 0}\\
     &&\text{Vertex 1}\\
     &&\text{Vertex 2}\\
     &&\text{Vertex 3}\\
    \end{array}&\text{Panel 0}
\notag\\
\mathbf{P}_{1,horiz} &= 
  \left [
    \begin{array}{ccc}
      W/2+D+W&0&0\\
      W/2+D+W&L&0\\
      -W/2+D+W&L&0\\
      -W/2+D+W&0&0\\
    \end{array}
  \right]&\text{Panel 1}
\notag\\
\mathbf{P}_{2,horiz} &= 
  \left [
    \begin{array}{ccc}
      W/2+2(D+W)&0&0\\
      W/2+2(D+W)&L&0\\
      -W/2+2(D+W)&L&0\\
      -W/2+2(D+W)&0&0\\
    \end{array}
  \right]&\text{Panel 2}
\end{align}

The vertices for Panel 0 rotated about the $y$-axis are given by the matrix product:
\begin{equation}
  \mathbf{P}_{0,rot} =
  \mathbf{P}_{0,horiz}
  \left[
    \begin{array}{ccc}
      \cos\sigma&0&\sin\sigma\\
      0&1&0\\
      -\sin\sigma&0&\cos\sigma\\
    \end{array}
  \right]
\end{equation}

The vertices for Panels 1 and 2 rotated about the $y$-axis are:
\begin{align}
  \mathbf{P}_{1,rot} &=
  \mathbf{P}_{0,rot}+
  \left [
    \begin{array}{ccc}
      D+W&0&0\\
      D+W&0&0\\
      D+W&0&0\\
      D+W&0&0\\
    \end{array}
  \right]
\notag\\
  \mathbf{P}_{2,rot} &=
  \mathbf{P}_{0,rot}+
  \left [
    \begin{array}{ccc}
      2(D+W)&0&0\\
      2(D+W)&0&0\\
      2(D+W)&0&0\\
      2(D+W)&0&0\\
    \end{array}
  \right]
\end{align}

The vertices for Panels 1, 2 and 3 both rotated about the $y$-axis and tilted from the horizontal at the subarray tilt angle are:
\begin{align}
\mathbf{P}_{0,rot,tilt} &= 
\left(\mathbf{P}_{0,rot}-
  \left [
    \begin{array}{ccc}
      0&L&0\\
      0&L&0\\
      0&L&0\\
      0&L&0\\
    \end{array}
  \right] 
\right)
  \left[
    \begin{array}{ccc}
      1&0&0\\
      0&\cos\beta&\sin\beta\\
      0&-\sin\beta&\cos\beta\\
    \end{array}
  \right]
  +
  \left[
    \begin{array}{ccc}
      0&L&0\\
      0&L&0\\
      0&L&0\\
      0&L&0\\
    \end{array}
  \right]
 \notag\\
\mathbf{P}_{1,rot,tilt} &= 
\left(\mathbf{P}_{1,rot}-
  \left [
    \begin{array}{ccc}
      0&L&0\\
      0&L&0\\
      0&L&0\\
      0&L&0\\
    \end{array}
  \right] 
\right)
  \left[
    \begin{array}{ccc}
      1&0&0\\
      0&\cos\beta&\sin\beta\\
      0&-\sin\beta&\cos\beta\\
    \end{array}
  \right]
  +
  \left[
    \begin{array}{ccc}
      0&L&0\\
      0&L&0\\
      0&L&0\\
      0&L&0\\
    \end{array}
  \right]
\notag\\
\mathbf{P}_{2,rot,tilt} &= 
\left(\mathbf{P}_{2,rot}-
  \left [
    \begin{array}{ccc}
      0&L&0\\
      0&L&0\\
      0&L&0\\
      0&L&0\\
    \end{array}
  \right] 
\right)
  \left[
    \begin{array}{ccc}
      1&0&0\\
      0&\cos\beta&\sin\beta\\
      0&-\sin\beta&\cos\beta\\
    \end{array}
  \right]
  +
  \left[
    \begin{array}{ccc}
      0&L&0\\
      0&L&0\\
      0&L&0\\
      0&L&0\\
    \end{array}
  \right]
\end{align}

Finally, the vertices of Panels 1, 2 and 3 rotated around the $y$-axis, tilted from the horizontal at the subarray tilt angle, and rotated about the $z$-axis at the subarray azimuth angle are:
\begin{align}
\mathbf{P}_{0,rot,tilt,sazm} &= 
\mathbf{P}_{0,rot,tilt}
  \left[
    \begin{array}{ccc}
      \cos\Psi&\sin\Psi&0\\
      -\sin\Psi&\cos\Psi&0\\
      0&0&1\\
    \end{array}
  \right]
 \notag\\
\mathbf{P}_{1,rot,tilt,sazm} &= 
\mathbf{P}_{1,rot,tilt}
  \left[
    \begin{array}{ccc}
      \cos\Psi&\sin\Psi&0\\
      -\sin\Psi&\cos\Psi&0\\
      0&0&1\\
    \end{array}
  \right]
\notag\\
\mathbf{P}_{2,rot,tilt,sazm} &= 
\mathbf{P}_{2,rot,tilt}
  \left[
    \begin{array}{ccc}
      \cos\Psi&\sin\Psi&0\\
      -\sin\Psi&\cos\Psi&0\\
      0&0&1\\
    \end{array}
  \right]
\end{align}

To make things easier to read, matrices $\mathbf{P}_{0}$, $\mathbf{P}_{1}$, and $\mathbf{P}_{2}$ indicate the four vertices for each of the three panels:
\begin{align}
\mathbf{P}_{0} &= 
\mathbf{P}_{0,rot,tilt,sazm} 
 \notag\\
\mathbf{P}_{1} &= 
\mathbf{P}_{1,rot,tilt,sazm} 
\notag\\
\mathbf{P}_{2} &= 
\mathbf{P}_{2,rot,tilt,sazm} 
\end{align}

\subsection{Shading Panel}

The shading panel is either Panel 0 or Panel 2, and is the panel that blocks beam solar radiation from Panel 1. Of the two panels, the shading panel $\mathit{iPanel}$ is the panel nearest the sun's direction.

By definition Panel 1 is between Panel 0 and Panel 2. The following vectors are between Vertex 0 of the three panels. The vectors from Panel 1 to Panel 0 ($toPrev$), and from Panel 2 to Panel 1 ($toNext$) are:
\begin{align}
\mathbf{v}_{toPrev} = \mathbf{P}_{0}(x_0,y_0,z_0) - \mathbf{P}_{1}(x_0,y_0,z_0)\notag\\
\mathbf{v}_{toNext} = \mathbf{P}_{2}(x_0,y_0,z_0) - \mathbf{P}_{1}(x_0,y_0,z_0)
\end{align}

To determine which of Panel 0 and Panel 2 is in the direction of the sun from Panel 1, compare the dot product of the sun unit vector with the vectors to and from Panel 1. The dot product with the larger positive value indicates the panel closest to the sun. The shading panel $\mathit{iPanel}$ is either Panel 0 or Panel 1:
\begin{equation}
P_{\mathit{iPanel}} = \left\{
\begin{array}{ll}
\mathbf{P}_0 & \text{if $\mathbf{v}_{toPrev} \cdot \mathbf{\hat s} \geq \mathbf{v}_{toNext} \cdot \mathbf{\hat s}$} \\
\mathbf{P}_2 & \text{if $\mathbf{v}_{toPrev} \cdot \mathbf{\hat s} < \mathbf{v}_{toNext} \cdot \mathbf{\hat s}$}
\end{array}
\right.
\end{equation}

\subsection{Shaded Fraction}

With the shading panel identified, and the position of the sun and panel angles for the current time step angles, the following vector equations determine whether $\mathit{iPanel}$ is shaded.

The midpoint $\mathit{midP}$ is the moddle of the edge of $\mathit{iPanel}$ closest to the sun. The edge of $\mathit{iPanel}$ closest to the sun depends on whether the shading panel is Panel 0 or Panel 2. If Panel 0 is shading, then the edge of $\mathit{iPanel}$ closest to the sun is the edge connecting Vertices 2 and 3:
%lib_irradproc.cpp 1169
\begin{equation}
\mathbf{v}_{midP} = \frac{\mathbf{P}_{1}(x_2,y_2,z_2) + \mathbf{P}_{1}(x_3,y_3,z_3)}{2}
\end{equation}

If Panel 2 is shading, then the edge of $\mathit{iPanel}$ closest to the sun is defined by Vertices 0 and 1:
\begin{equation}
\mathbf{v}_{midP} = \frac{\mathbf{P}_{1}(x_0,y_0,z_0) + \mathbf{P}_{1}(x_1,y_1,z_1)}{2}
\end{equation}

The vector normal  $\mathit{normal}$ to $\mathit{iPanel}$ is the cross product of any two vectors in the plane of the shading panel:
%lib_irradproc.cpp 1179
\begin{equation}
\mathbf{v}_{normal} = \left[\mathbf{P}_{iPanel}(x_1,y_1,z_1)-\mathbf{P}_{iPanel}(x_0,y_0,z_0)\right]\times\left[\mathbf{P}_{iPanel}(x_3,y_3,z_3)-\mathbf{P}_{iPanel}(x_0,y_0,z_0)\right]
\end{equation}

The intersection $\mathit{sunDot}$ of the $\mathit{iPanel}$ normal vector $\mathit{normal}$ and an infinite vector in the direction of the sun vector $\mathit{sun}$:
%lib_irradproc.cpp 11691189
\begin{equation}\label{eqn_sunDot}
\mathbf{v}_{sunDot}=\mathbf{v}_{normal}\cdot\mathbf{v}_{\hat s}
\end{equation}

%% IF \mathbf{v}_{sunDot}<0.001 THEN shade fraction = 0 because sun vector lies in the plane of the array

The distance $\mathit{t}$ along the sun vector from $\mathit{midP}$ to the plane of $\mathit{iPanel}$:
 %lib_irradproc.cpp 1196
\begin{equation}
t=\mathbf{v}_{normal}\cdot[\mathbf{P}_{iPanel}(x0,y0,z0)-\mathbf{v}_{midP}]
\end{equation}

The point of intersection $\mathit{intersectP}$ of the sun vector infinite plane of  $\mathit{iPanel}$:
%lib_irradproc.cpp 1205
\begin{equation}
\mathbf{v}_{intersectP}=\mathbf{v}_{midP}+t\mathbf{\hat s}
\end{equation}

Determine whether the point is within the bounds of the panel. If either of the following conditions is true, the point is outside of the bounds of the panel:
%lib_irradproc.cpp 1212
\begin{align}\label{eqn_checkshade}
[\mathbf{P}_{iPanel}(x3,y3,z3)-\mathbf{P}_{iPanel}(x0,y0,z0)]\cdot
[\mathbf{v}_{intersectP}-\mathbf{P}_{iPanel}(x0,y0,z0)] <0\notag\\
[\mathbf{P}_{iPanel}(x0,y0,z0)-\mathbf{P}_{iPanel}(x3,y3,z3)]\cdot
[\mathbf{v}_{intersectP}-\mathbf{P}_{iPanel}(x3,y3,z3)] <0
\end{align}

Next, determine whether $\mathit{iPanel}$ is shaded. It is not shaded when either $\mathbf{v}_{intersectP}$ is outside the bounds of the panel (Equation \ref{eqn_checkshade}), or when the sun vector lies within the plane of the subarray, $\mathbf{v}_{sunDot}<0.001$ (Equation \ref{eqn_sunDot}). 

If $\mathit{iPanel}$ is shaded, it is shaded by either Panel 0 or Panel 2. If it is shaded by Panel 0, the upper and lower edge midpoints on the adjacent panel are: 
%lib_irradproc.cpp 1231, 1234
\begin{align}
\mathbf{v}_{mu}=\frac{\mathbf{P}_{0}(x0,y0,z0) + \mathbf{P}_{0}(x1,y1,z1)}{2}\notag\\
\mathbf{v}_{ml} =\frac{\mathbf{P}_{0}(x2,y2,z2) + \mathbf{P}_{0}(x3,y3,z3)}{2}
\end{align}

Otherwise, if $\mathit{iPanel}$ is shaded by Panel 2: 
%lib_irradproc.cpp 1232, 1235
\begin{align}
\mathbf{v}_{mu}=\frac{\mathbf{P}_{2}(x2,y2,z2) + \mathbf{P}_{2}(x3,y3,z3)}{2}\notag\\
\mathbf{v}_{ml} =\frac{\mathbf{P}_{2}(x0,y0,z0) + \mathbf{P}_{2}(x1,y1,z1)}{2}
\end{align}

The geometric shaded fraction:
%lib_irradproc.cpp 1241
\begin{equation}\label{eqn_1xshadfrac}
A_b = \frac{ (\mathbf{v}_{intersectP}-\mathbf{v}_{mu})  \cdot (\mathbf{v}_{ml}-\mathbf{v}_{mu})}{(\mathbf{v}_{ml}-\mathbf{v}_{mu})\cdot (\mathbf{v}_{ml}-\mathbf{v}_{mu})}
\end{equation}

If $\mathit{iPanel}$ is not shaded, then the geometric shading fraction is zero:
\begin{equation}
A_b = 0
\end{equation}

The reduction in POA beam irradiance for one-axis tracking:
%cmod_pvsamv1.cpp 1255
\begin{equation}\label{eqn-ss1x}
S_{1xss} = 1 - A_b
\end{equation}


%%%%%%%%%%%%%%%%%%%%%%%%%%%%%%%%%%%%%%%%%%%%%%%%%%
%%%%%%%%%%%%%%%%%%%%%%%%%%%%%%%%%%%%%%%%%%%%%%%%%%
\chapter{Module DC Output}\label{sec-module}

SAM uses one of four module models and one of three cell temperature models to calculate the DC power output of a single module in each subarray. SAM makes the following assumptions:
\begin{itemize}
\item All modules in the system operate at their maximum power point voltage, except for the subarray mismatch and inverter operating voltage limit losses described below.
\item The subarray maximum power point is determined by the maximum power point of a single module and number of modules per string.
\item All subarrays in the system have the same number of modules per string, and therefore operate at the same voltage.
\item All modules in each subarray operate uniformly. SAM does not calculate module mismatch losses.
\end{itemize}

Because the CEC and IEC 61853 module models (Section~\ref{sec-moduleoptions}) use a continuous function to represent the module's I-V curve, SAM can estimate losses due to subarray mismatch and inverter input operating input voltage limits:
\begin{itemize}
\item For an array with multiple subarrays, the optional PV subarray mismatch option adjusts the maximum power voltage to account for mismatch \textit{between subarrays} as described in Section~\ref{sec-mismatch}.
\item When the inverter specifications include maximum and minimum MPPT voltage limits and the string voltage exceeds those limits, SAM clamps the string voltage to those limits.
\end{itemize}
Calculating the module's DC output in a time step when the sun is up (see Section~\ref{sec-sunup}) is a two step process:
%cmod_pvsamv1.cpp 1466
\begin{enumerate}
\item Calculate the photovoltaic cell temperature. SAM assumes that the temperature of cells in all of the modules in each subarray is uniform.
\item Calculate the module's DC power output based on its physical characteristics, the effective irradiance, and cell temperature.
\end{enumerate}

Table~\ref{tab-modulevars} lists the input and output variables used by each of SAM's module and cell temperature models. Each model uses an additional set of input parameters listed in each model's section below.

\begin{table}
\begin{center}
\caption{Module Model Variable Definitions}
\begin{tabular}{ll}
\midrule
Symbol & Description / \textbf{Name in SAM}\\
\midrule
\multicolumn{2}{c}{\textit{Effective POA Irradiance Inputs}}\\
$G_b$ & effective beam irradiance (W/m$^2$)\\
$G_d$ & effective sky diffuse irradiance (W/m$^2$) \\
$G_r$ & effective ground-reflected diffuse irradiance (W/m$^2$)  \\
$\AOI$ & incidence angle (deg)\\
$Z$ & sun zenith angle (deg)\\
\midrule
\multicolumn{2}{c}{\textit{Inputs from Weather File}}\\
$T_{dry}$ & ambient dry bulb temperature ($^\circ$C) \\
$T_{dew}$ & dew-point temperature ($^\circ$C)$^\ast$ \\
$p_{atm}$ & atmospheric pressure (mbar)$^\ast$\\
$v_w$ & wind speed (m/s)\\
$h$ & elevation above sea level (m)\\
$\mathit{hr}$ & hour of day local time (h)\\
\midrule
\multicolumn{2}{c}{\textit{Subarray Inputs}}\\
$\beta_s$ & subarray \textbf{tilt} angle (deg) \\
$\gamma_s$ & subarray \textbf{azimuth} angle (deg)\\
\midrule
\multicolumn{2}{c}{\textit{Intermediate Outputs}}\\
$P_{dc}$ & module power (W)\\
$V_{dc}$ & module voltage (V)\\
$I_{dc}$ & module current  (A)\\
$V_{oc}$ & operating open circuit voltage (V) \\
$I_{sc}$ & operating closed circuit current (A)\\
\midrule
\multicolumn{2}{c}{\textit{Outputs}}\\
$\eta_m$ & module efficiency (\%)\\
$T_c$ & cell temperature ($^\circ$C)\\
\midrule
\multicolumn{2}{c}{$^\ast$Only required for heat transfer cell temperature model (Section~\ref{sec-tcheattransfer}).}\\

\end{tabular}
\label{tab-modulevars}
\end{center}
\end{table}

\section{Module Models}\label{sec-moduleoptions}

SAM calculates the a single photovoltaic module's DC output with one of the module models listed below \citep{blair2013}. The simple efficiency and Sandia module models are point-value models that calculate module power at one or more discrete points on the module's I-V curve. The CEC and IEC-61853 models are single-diode equivalent circuit models that represent the module I-V curve as a continuous function.

\begin{itemize}
\item \textbf{Simple Efficiency Module Model} (Section~\ref{sec-simplemodule}) is a simple representation of module performance that calculates the module's DC output at the maximum power point from the module area, a table of conversion efficiency values over a range of irradiances, and temperature correction parameters. The simple efficiency model is the least accurate of the three models for predicting the performance of specific modules. It is useful for preliminary performance predictions before you have selected a specific module, and allows you to explicitly specify the module efficiency, which is useful for analyses involving sensitivity or parametric analysis.
\item \textbf{California Energy Commission (CEC) Performance Model with Module Database} (Section~\ref{sec-cecmodule}) is an implementation of the six-parameter, single-diode equivalent circuit model used in the CEC New Solar Homes Partnership Calculator \citep{gsc2014a}, and is an extension of the five-parameter model described in \citet{desoto2004a}. The model calculates the photovoltaic module DC output using equations with parameters stored in SAM's CEC module library (see Section~\ref{sec-libraries}). The library contains data provided by the California Energy Commission \citep{gsc2014a} \citep{gsc2014b}.
\item \textbf{CEC Performance Model with User Entered Specifications}  (Section~\ref{sec-cecmodule}) is the same implementation as the CEC Performance Model with Module Database, but with a coefficient calculator \citep{dobos2012a} to calculate the model parameters from the standard module specifications provided on manufacturer data sheets. This makes it possible to use the six-parameter model for modules not included in the CEC module library.
\item \textbf{IEC 68153 Single Diode Model} (Section~\ref{sec-iecmodule}) is a ten-parameter single-diode model adapted from the five-parameter single-diode model. It calculates model parameter values from a table of module test data that follow the IEC 61853 rating standard \citep{dobos2014}. 
\item \textbf{Sandia PV Array Performance Model with Module Database}  (Section~\ref{sec-sandiamodule}) is an implementation of the Sandia National Laboratories photovoltaic module and array performance model \citep{king2004}. This empirical model calculates module voltage and power at five points on the module's I-V curve using data measured from modules and arrays in realistic outdoor operating conditions. The database stored in SAM's Sandia module library (see Section~\ref{sec-libraries}) includes modules with different cell types, including crystalline silicon, and various thin film technologies.
\end{itemize}

In SSC, the \texttt{pvsamv1} input variable \texttt{module\_model} determines the module model as shown in Table~\ref{tab-modulesubmodels}.

\begin{table}
\begin{center}
\caption{Module Models in SSC}
\begin{tabular}{lc}
\midrule
Name in SAM & \texttt{module\_model} \\
\midrule
Simple Efficiency Module Model & 0 \\
CEC Performance Model with Module Database & 1 \\
CEC Performance Model with User Entered Specifications & 2 \\
Sandia PV Array Performance Model with Module Database & 3 \\
\hline
\end{tabular}
\label{tab-modulesubmodels}
\end{center}
\end{table}

\section{Cell Temperature Models}\label{sec-celltempoptions}

Each module model uses one of the cell temperature models shown in Table~\ref{tab-tempcorr} to calculate the photovoltaic cell temperature in a given time step. Each module model uses a single cell temperature model, except for the CEC module model with database parameters, which offers two cell temperature model options.

\begin{table}
\begin{center}
\caption{Cell Temperature Models}
\begin{tabular}{lll}
\midrule
Module Model & Temperature Model & Section\\
\midrule
Simple efficiency & Sandia & \ref{sec-tcsandia}\\
Sandia & Sandia & \ref{sec-tcsandia} \\
CEC with database parameters & NOCT or Heat Transfer & \ref{sec-tcnoct} or \ref{sec-tcheattransfer}\\
CEC with user-specified parameters & NOCT & \ref{sec-tcnoct}\\
\hline
\end{tabular}
\label{tab-tempcorr}
\end{center}
\end{table}

In each time step, SAM calls the appropriate cell temperature model once before calling the module model. When the subarray mismatch option (Section~\ref{sec-mismatch}) is enabled, it also calls the cell temperature model as part of that set of calculations.

\section{Module Degradation}

SAM's module models calculate the DC power output of a single module in a single time step. For a given simulation, SAM runs the module model for each time step (8760 times for a simulation with hourly time steps) to calculate the output over a single year. Because the model calculates the output in a single year, it does not have the information required to model degradation of module output over a number of years of operation.

SAM's photovoltaic performance model calculates the AC output of the photovoltaic system for each time step in one year. The financial models add up the time series values to calculate the system's total AC output in one year, and treat this value as the system's total output in the first year of the system's operation.

The financial models use a lifetime in years and annual degradation factor to calculate the system's output in the second year and later for cash flow calculations. This degradation factor can approximate the impact of module degradation on the performance of a photovoltaic system. However, because the degradation factor applies to the total annual \textit{AC output} of the system, it only very roughly represents module degradation. In an actual system, module degradation reduces the array's \textit{DC output}, and this reduction can affect inverter sizing decisions in a way that SAM cannot model.

If the weather file for a given analysis contains typical-year data (rather than data for a specific year), using the AC degradation factor to represent module degradation is further complicated by the fact that the solar resource input to the module model represents the resource over a multi-year period.

%%%%%%%%%%%%%%%%%%%%%%%%%%%%%%%%%%%%%%%%%%%%%%%%%%
%%%%%%%%%%%%%%%%%%%%%%%%%%%%%%%%%%%%%%%%%%%%%%%%%%
\section{Sandia Module Model}\label{sec-sandiamodule}

SAM's Sandia Module Model is an implementation of the empirical model of a photovoltaic module described in \citet{king2004}. See Section~\ref{sec-moduleoptions} for a general description of all of SAM's module models.

For a description of the Sandia cell temperature model, see Section~\ref{sec-tcsandia}.

The Sandia module model uses the module parameters in Table~\ref{tab-sandiamodulevars} along with the general module input variables in Table~\ref{tab-modulevars}. The parameters in Table~\ref{tab-sandiamodulevars} are from the Sandia module database maintained by Sandia National Laboratories \citep{sandia-testeval}.  In SAM, the parameters are stored in the Sandia module library. In SSC, the implementation of the Sandia module model described here is part of the \texttt{pvsamv1} module and does not include a parameter library (See Section~\ref{sec-libraries}).

%%
% How does SAM use fill factor? See cmod_pvsamv1.cpp 769, and Equation 6 in King 2004.

\begin{table}
\begin{center}
\caption{Sandia Module Model Inputs}
\begin{tabular}{lll}
\midrule
Symbol & Description / \textbf{Name in SAM} & Name in SSC \\
\midrule
$V_{mp,ref}$ & reference \textbf{Max Power Voltage} (V)&\texttt{snl\_vmpo} \\
$I_{mp,ref}$ & reference \textbf{Max Power Current} (A)& \texttt{snl\_i\_mpo} \\
$V_{oc,ref}$ & reference \textbf{Open Circuit Voltage} (V)& \texttt{snl\_voco} \\
$I_{sc,ref}$ & reference \textbf{Short Circuit Current} (A) & \texttt{snl\_isco} \\
$\alpha_{sc,ref}$ & normalized short circuit current temperature coefficient   (1/$^\circ$C) & \texttt{snl\_aisc}\\
$\beta_{oc,ref}$ & open circuit voltage temperature coefficient  (V/$^\circ$C)& \texttt{snl\_bvoco} \\
$\beta_{mp,ref}$ & maximum power voltage temperature coefficient  (V/$^\circ$C)& \texttt{snl\_bvmpo} \\
$\gamma_{mp,ref}$ & maximum power temperature coefficient  (W/$^\circ$C)& \texttt{snl\_aimp} \\
$M_{\beta oc}$ & relates $\beta_{oc,ref}$ to effective irradiance (V/$^\circ$C)& \texttt{snl\_mbvoc} \\
$M_{\beta mp}$ & relates $\beta_{mp,ref}$ to effective irradiance (V/$^\circ$C)& \texttt{snl\_mbvmp} \\
$s$ & number of cells in series & \texttt{snl\_series\_cells} \\
$C_0$, $C_1$ & coefficients relating $I_{mp}$ to $G$ & \texttt{snl\_c0}, \texttt{snl\_c1} \\
$C_2$, $C_3$ & coefficients relating $V_{mp}$ to $G$ ($C_3$ is in 1/V)& \texttt{snl\_c2}, \texttt{snl\_c3} \\
$C_4$, $C_5$ & coefficients relating $I_x$ to $G$ & \texttt{snl\_c4}, \texttt{snl\_c5} \\
$C_6$, $C_7$ & coefficients relating $I_{xx}$ to $G$ & \texttt{snl\_c6}, \texttt{snl\_c7} \\
$n$ & diode factor & \texttt{snl\_n} \\
$f_d$ & fraction of diffuse irradiance used by module & \texttt{snl\_fd} \\
$a_{0...4}$ & air mass coefficients 0...4 & \texttt{snl\_a[\textit{0...4}]} \\
$b_{0...5}$ & incidence angle modifier coefficients 0...5 & \texttt{snl\_b[\textit{0...5}]} 
\\
\hline
\end{tabular}
\label{tab-sandiamodulevars}
\end{center}
\end{table}

The reference conditions for the Sandia module model parameters are total incident irradiance of 1,000 W/m$^2$ and reference cell temperature of 25\degree C. 

When you choose a module on SAM's Module input page for the \textbf{Sandia PV Array Performance Model with Module Database}, SAM displays some of the module's reference parameters in Table~\ref{tab-sandiamodulevars} from the Sandia Module library as uneditable values.

The absolute air mass is a relative measure of the optical path length that sunlight travels through the atmosphere and depends on the sun zenith angle in degrees (Section~\ref{sec-sunangles}) and the elevation above sea level ($AM=1$ at sea level with the sun directly overhead):
%lib_sandia.cpp 245, 154
\begin{equation}\label{eqn-sandiaam}
AM = \left[ \cos(\frac{\pi}{180} Z ) + 0.5057 (96.08 - Z)^{-1.634} \right]^{-1} e^{-0.0001184 h}
\end{equation}

The $F_1$ polynomial relates the spectral effects on $I_{sc}$ to the variation of air mass over the day:
%lib_sandia.cpp 248, 136
\begin{equation}
F_1 = a_0 + a_1 AM + a_2 AM^2 + a_3 AM^3 + a_4 AM^4
\end{equation}

The $F_2$ polynomial relates the optical effects on $I_{sc}$ to the angle of incidence $\AOI$ (Section~\ref{sec-theta}):
%lib_sandia.cpp 251, 118
\begin{equation}
F_2 = b_0 + b_1\AOI
		+ b_2\AOI^2
		+ b_3\AOI^3
		+ b_4\AOI^4
		+ b_5\AOI^5
\end{equation}

The short circuit current is a function of the cell temperature (Equation~, where $T_c$ is from Equation~\ref{eqn-sandiatemp}), effective irradiance (Section~\ref{sec-effectiveirradiance}) and the short circuit temperature coefficient in A/\degree C. The diffuse utilization factor $F_d$ is one of the parameters from the Sandia module library, and is equal to one for flat plate modules:
%lib_sandia.cpp 254, 96
\begin{equation}
I_{sc} = I_{sc,ref} F_1 \left( \frac{G_b F_2 + f_d (G_d+G_r)}{1000} \right) \left[1+\alpha_{sc,ref} (T_c-25)\right]
\end{equation}

The effective irradiance on the module surface to which the cells respond:
%lib_sandia.cpp 256, 169
\begin{equation}
E_e = \frac{I_{sc}}{I_{sc,ref} \left[1 + \alpha_{sc,ref} (T_c - 25)\right]}
\end{equation}

The current at the maximum power point:
%lib_sandia.cpp 257, 107
\begin{equation}
I_{mp} = I_{mp,ref} (C_0 E_e + C_1 E_e^2) \left[1 + \alpha_{sc,ref} (T_c - 25)\right]
\end{equation}

The following two intermediate values are required for the maximum power voltage equations:
%lib_sandia.cpp 18, 45
\begin{align}
\Delta T_c &= n \left(\frac{1.38066\times10^{-23} (T_c + 273.15)}{1.60218\times10^{-19}}\right) \notag\\
%\beta_{oc} & = \beta_{oc,ref} + M_{\beta oc} (1 - E_e) \notag\\
\beta_{mp} & = \beta_{mp,ref} + M_{\beta mp} (1 - E_e)
\end{align}

%The open circuit voltage is zero if $E_e \leq 0$. Otherwise:
%%lib_sandia.cpp 263, 18
\begin{equation}
V_{oc} = V_{oc,ref}+ s~\Delta T_c \ln(E_e)+ \beta_{oc} (T_c - 25)
\end{equation}

The voltage at the maximum power point is zero if $E_e \leq 0$. Otherwise:
%lib_sandia.cpp 266, 48
\begin{equation}
V_{mp} = V_{mp,ref} +
 C_2~s~\Delta T_c~\ln(E_e) +
 C_3~s~\left[\Delta T_c \ln(E_e)\right]^2 +
 \beta_{mp} (T_c - 25)
\end{equation}

The module's DC power output is at the maximum power point:
%lib_sandia.cpp 295
\begin{equation}
P_{dc,m} = V_{mp}~I_{mp}
\end{equation}

%%%%%%%%%%%%%%%%%%%%%%%%%%%%%%%%%%%%%%%%%%%%%%%%%%
%%%%%%%%%%%%%%%%%%%%%%%%%%%%%%%%%%%%%%%%%%%%%%%%%%
\section{CEC Module Model}\label{sec-cecmodule}

SAM's CEC Module Model is an implementation of the single-diode equivalent circuit model of a photovoltaic module described in \citet{desoto2004a}, and with more detail in \citet{desoto2004b}. It is also described in Section 23.2 of  \citet{duffie2013}. See Section~\ref{sec-moduleoptions} for a general description of all of SAM's module models.

SAM includes two options for the model (see Section~\ref{sec-module}): It may use a set of reference parameters from the CEC module library with data provided by the California Energy Commission \citep{gsc2014a}, or a set of reference parameters generated by SAM's coefficient generator from module specifications that you provide as input to the model. The equations described in this section are for the module's DC output, not for the module coefficients. For a description of the coefficient generator equations, see \citet{dobos2012a}. 

Table~\ref{tab-cecmodulevars} shows the five module specifications and six parameters that are input variables specific to the CEC module model. The model also uses the general module input variables in Table~\ref{tab-modulevars}.

In SSC, the implementation of the CEC module model described here is part of the \texttt{pvsamv1} module. The coefficient generator is available in SSC as a separate compute module called \texttt{6parsolve}.

\begin{table}
\begin{center}
\caption{CEC Module Model Inputs}
\begin{tabular}{lll}
\midrule
Symbol & Description / \textbf{Name in SAM} & Name in SSC \\
\midrule
$I_{mp,ref}$ & reference \textbf{Max Power Current}  (A)& \texttt{cec\_i\_mp\_ref} \\
$V_{oc,ref}$ & reference \textbf{Open Circuit Voltage} (V) & \texttt{cec\_v\_oc\_ref} \\
$I_{sc,ref}$ & reference \textbf{Short Circuit Current} (A) & \texttt{cec\_i\_sc\_ref} \\
$\alpha_{sc,ref}$ & short circuit current temperature coefficient (A/\degree C) & \texttt{cec\_alpha\_sc} \\
$\beta_{oc,ref}$ & open circuit voltage temperature coefficient (V/\degree C) & \texttt{cec\_beta\_oc} \\
$I_{L,ref}$ & reference light current, \textbf{I\_L\_ref} (A) & \texttt{cec\_i\_l\_ref} \\
$I_{o,ref}$ & reference diode saturation current, \textbf{I\_o\_ref} (A) & \texttt{cec\_i\_o\_ref} \\
$R_{s,ref}$ & reference series resistance, \textbf{R\_s} ($\Omega$) & \texttt{cec\_r\_s\_ref} \\
$a_{ref}$ & reference ideality factor, \textbf{A\_ref} (V) & \texttt{cec\_a\_ref} \\
$R_{sh,ref}$ & reference shunt resistance \textbf{R\_sh\_ref} ($\Omega$)& \texttt{cec\_r\_sh\_ref} \\
$\mathit{adjust}$ & temperature coefficient adjustment factor & \texttt{cec\_adjust} \\
\hline
\end{tabular}
\label{tab-cecmodulevars}
\end{center}
\end{table}

The reference conditions for the CEC module model parameters are total incident irradiance of 1,000 W/m$^2$ and reference cell temperature of 25\degree C. 

When you select a module on SAM's Module input page for the \textbf{CEC Performance Model with Module Database}, SAM displays the module's reference parameters shown in Table~\ref{tab-cecmodulevars} from the CEC Module library as uneditable values.

The five-parameter single-diode equivalent circuit equation for the module current $I$ at a given voltage $V$ is:
\begin{equation}\label{eqn-cec5par}
I = I_L - I_o \left[ \exp\left(  \frac{V+IR_S}{a} \right) -1 \right] - \frac{V + IR_S}{R_{sh}}
\end{equation}

The temperature coefficient of short circuit current $\mu_{isc}$ and open circuit voltage $\beta_{voc}$ in the model are adjusted from the reference coefficients using a sixth parameter, $\mathit{adjust}$, which may be either from the CEC module library, or calculated by SAM's coefficient generator:
%lib_cec6par.cpp 302
\begin{align}
\mu_{isc} & = \alpha_{sc,ref} \left( 1-\frac{\mathit{adjust}}{100}\right) \notag\\
\beta_{voc} & = \beta_{oc,ref} \left( 1+\frac{\mathit{adjust}}{100}\right) 
\end{align}

The global effective irradiance at the top of the module cover:
\begin{equation}
G = G_b + G_d + G_r
\end{equation}

Equations for the transmittance through the module cover use three constants:
%lib_cec6par.cpp 210
\begin{align}\label{eqn-coverproperties}
n & = 1.526 \quad\text{(refractive index of glass)} \notag\\
L & = 0.002 \quad\text{(thickness of glass cover in meters)} \notag\\
K &= 4 \quad\text{(proportionality constant in meters$^{-1}$)}
\end{align}

%lib_cec6par.cpp 230
The angle of refraction, assuming that the refractive index of air is one:
\begin{equation}
\theta_r = \arcsin\left(\frac{1}{n} \sin \AOI \right)
\end{equation}

The transmittance as a function of incidence angle $\theta_i$ in radians:
%lib_cec6par.cpp 232
\begin{equation}\label{eqn-transmittance}
\tau(\theta) = e^{-K L/\cos \theta_r} \left[1 - \frac{1}{2} \left( \frac{\sin^2(\theta_r-\theta_i)}{\sin^2(\theta_r+\theta_i)}
			+ \frac{\tan^2(\theta_r-\theta_i)}{\tan^2(\theta_r+\theta_i)} \right)\right] 
\end{equation}

The incidence angle for the sky diffuse and ground-reflected components of the effective irradiance:
%lib_cec6par.cpp 265
\begin{align}
\theta_d &= 59.7 - 0.1388 \beta + 0.001497 \beta^2 \quad\text{(sky diffuse angle)} \notag\\
\theta_g &= 90 - 0.5788 \beta  + 0.002693 \beta^2 \quad\text{(ground-reflected diffuse angle)} 
\end{align}

The transmittance for each component of the effective irradiance is calculated using Equation~\ref{eqn-transmittance} with the module cover properties in Equation~\ref{eqn-coverproperties} and the following incidence angle values:
%lib_cec6par.cpp 258, De Soto Eq 13
\begin{align}
(\tau\alpha)_n: \theta_i &= 1 \quad\text{(surface normal)} \notag\\
(\tau\alpha)_b: \theta_i &= \theta \quad\text{(beam)} \notag\\
(\tau\alpha)_d: \theta_i &= \theta_d \quad\text{(sky diffuse)} \notag\\
(\tau\alpha)_g: \theta_i &= \theta_g  \quad\text{(ground diffuse)}
\end{align}

Note that $(\tau\alpha)_n$ is set to 1 degree (a very small number) instead of zero to avoid potential divide-by-zero errors.

The incidence angle modifier for each component of the effective irradiance is then:
%lib_cec6par.cpp 258, De Soto Eq 13
\begin{align}
K_{\tau\alpha,b} = \frac{(\tau\alpha)_b}{(\tau\alpha)_n} \notag\\
K_{\tau\alpha,d} = \frac{(\tau\alpha)_d}{(\tau\alpha)_n} \notag\\
K_{\tau\alpha,g} = \frac{(\tau\alpha)_g}{(\tau\alpha)_n}
\end{align}

The irradiance absorbed by the photovoltaic cell is:
%lib_cec6par.cpp 310, De Soto Eq 19
\begin{equation}
G_0 = G_b K_{\tau\alpha,b} + G_d K_{\tau\alpha,d} + G_r K_{\tau\alpha,g}
\end{equation}

%lib_cec6par.cpp 318
The model limits sun zenith angle, setting its value to the minimum or maximum when it is exceeded:
\begin{equation}
0 < Z < 86\degree  
\end{equation}

The transmittance-absorptance product:
%lib_cec6par.cpp 325
\begin{equation}
\tau\alpha = 0.9 \frac{G_0}{G}
\end{equation}

The air mass, with an exponential correction factor adapted from the Sandia model for elevation above sea level:
%lib_cec6par.cpp 332, De Soto Eq 18
\begin{equation}\label{eqn-cecam}
\textit{AM} = \left[\cos\left( \frac{\pi}{180}Z \right)+0.5057(96.080-Z)^{-1.634} \right]^{-1} e^{-0.0001184\mathit{h}}
\end{equation}

The air mass modifier $M$ accounts for the effect of air mass on spectral distribution, and is from \citet{king2004} with coefficients for polycrystalline cells:
%lib_cec6par.cpp 334, De Soto Eq 17
\begin{align}
a_0 &= 0.918093 \notag\\
a_1 &= 0.086257 \notag\\
a_2 &= -0.024459 \notag\\
a_3 &= 0.002816 \notag\\
a_4 &= -0.000126 \notag\\
\textit{M} &= a_0 + a_1 \mathit{AM} + a_2 \mathit{AM}^2 + a_3\mathit{AM}^3+ a_4\mathit{AM}^4
\end{align}

The adjusted effective transmitted irradiance:
%lib_cec6par.cpp 335
\begin{equation}\label{eqn-effectivetransmittedirradiance}
G =  MG_0
\end{equation}

%lib_cec6par.cpp 338
The module DC output is only calculated when $G>1$.

The cell temperature $T_c$ is calculated using either the NOCT cell temperature model (Section~\ref{sec-tcnoct}) or the heat transfer cell temperature model (Section~\ref{sec-tcheattransfer}) as described in Section \ref{sec-celltempoptions}.

For the next set of equations, the cell temperature is in Kelvin, and the reference cell temperature of $25\degree C$ in Kelvin is $25\degree \text{C} + 273.15 \text{K} = 298.15 \text{K}$.
The light current $I_L$:
%lib_cec6par.cpp 343
\begin{equation}\label{eqn-lightcurrent}
I_L = \frac{G}{1000}\left[I_{L,ref} + \mu_{isc} (T_c - 298.15 )\right]
\end{equation}

%lib_cec6par.cpp 220
The diode reverse saturation current $I_o$, assuming a reference band-gap energy of silicon of 1.12 eV:
%lib_cec6par.cpp 347
\begin{align}
k &= 8.618 \times 10^{-5} &\quad\text{Boltzmann constant in eV/K} \notag\\
E_{bg} &= 1.12 \left[1-0.0002677(T_c-T_{c,ref})\right] &\quad\text{cell material band-gap energy in eV}\notag\\
I_o &= I_{o,ref} \left( \frac{T_c}{298.15} \right)^3 \exp\left[ \frac{1}{k} \left( \frac{1.12}{T_{c,ref}} - \frac{E_{bg}}{T_c} \right) \right] &
\end{align}

At the open circuit voltage $V_{oc}$, $I=0$, and Equation~\ref{eqn-cec5par} reduces to:
%lib_cec6par.cpp 193
\begin{equation}\label{eqn-cec5parvoc}
I = I_L - I_o \left[ \exp\left(  \frac{V_{oc}+IR_S}{a} \right) -1 \right] - \frac{V_{oc}}{R_{sh}}
\end{equation}

SAM uses the bisection method to solve Equation~\ref{eqn-cec5parvoc} for $V_{oc}$ with the reference open circuit voltage $V_{oc,ref}$ as an initial guess and the following parameter values:
%lib_cec6par.cpp 351
\begin{align} 
a &= a_{ref} \frac{T_c}{T_{c,ref}} \notag\\
R_{sh} &= R_{sh,ref} \frac{1000}{G}
\end{align}

The short circuit current:
%lib_cec6par.cpp 352
\begin{equation} \label{eqn-cecisc}
I_{sc} = \frac{I_L }{1+\frac{R_{s,ref}}{R_sh}}
\end{equation}

For the CEC module model, SAM can calculate the module operating voltage either using the string mismatch equations described in Section~\ref{sec-mismatch}, or the equations described below.%not clear where module_voltage is calculated without mismatch option

%The voltage at maximum power:
%%lib_cec6par.cpp 385
%\begin{equation}
%V_{mp} = \frac{1}{2} \left[ V_{mp,ref} + \beta_{voc} (T_c - T_{c,ref}) \right]
%\end{equation}

%lib_cec6par.cpp 393
For the current at maximum power $I_{mp}$, SAM uses Newton's method to find a solution of Equation~\ref{eqn-cec5par} using the calculated values for $I_L$, $I_o$, $a$, and $R_{sh}$ at the maximum power point voltage. Because the effect of series resistance on the maximum power point is relatively small, for $R_S$, the model uses the reference value so that $R_S=R_{S,ref}$. Also, $I_L$ from Equation~\ref{eqn-lightcurrent} is multiplied by a factor of 0.9.

%lib_cec6par.cpp 367
For systems with one subarray, or for systems with more than one subarray with the subarray mismatch option disabled, SAM calculates the maximum power point voltage using a golden section search to determine the voltage between 0 and $V_{oc}$ that results in the maximum power given the current at maximum power.

%lib_cec6par.cpp 403
For systems with more than one subarray and the subarray mismatch option enabled (Section~\ref{sec-mismatch}, SAM sets the maximum power voltage to the value calculated by the mismatch algorithm.

The module DC output is the power at maximum power:
%lib_cec6par.cpp 391, 407
\begin{equation}
P_{dc,m} = V_{mp}~I_{mp}
\end{equation}

%%%%%%%%%%%%%%%%%%%%%%%%%%%%%%%%%%%%%%%%%%%%%%%%%%
%%%%%%%%%%%%%%%%%%%%%%%%%%%%%%%%%%%%%%%%%%%%%%%%%%
\section{Simple Efficiency Module Model}\label{sec-simplemodule}

SAM's Simple Efficiency Module Model  (see Section~\ref{sec-moduleoptions}) is a basic single point efficiency representation of a photovoltaic module that uses an efficiency table and the Sandia cell temperature model (Section~\ref{sec-tcsandia}) to calculate the module's DC power output.

The simple efficiency module model uses the module parameters shown in Table~\ref{tab-spemodulevars} along with the general module input variables in Table~\ref{tab-modulevars}. Note that the model does not use the reference parameters $V_{mp,ref}$ and $V_{oc,ref}$. Those values are required by the sizing algorithm that is part of SAM's user interface (Section~\ref{sec-sizing}).

The cell temperature parameters $a$, $b$, and $\Delta T$ depend on the \textbf{Module Structure and Mounting} option (\texttt{spe\_module\_structure} in SSC) as listed in Table~\ref{tab-sandiamodstruct}. Note that the ``use database values" option is not available for the simple efficiency module model, so that  $\mathtt{spe\_module\_structure}=0$ is for the ``glass/cell/polymer sheet - open rack" option instead of the``use database option" as shown in the table for \texttt{snl\_module\_structure}.

\begin{table}
\begin{center}
\caption{Simple Efficiency Module Model Inputs}
\begin{tabular}{lll}
\midrule
Symbol & Description / \textbf{Name in SAM} & Name in SSC \\
\midrule
$V_{mp,ref}$ & reference \textbf{Maximum Power Voltage} (V)& - \\
$V_{oc,ref}$ & reference \textbf{Open Circuit Voltage} (V)& - \\
$A_m$ & module \textbf{Area} (m$^2$)& \texttt{spe\_area} \\
$\gamma_{mp,ref}$ & maximum power \textbf{Temperature Coefficient} (\%/$^\circ$C) & \texttt{spe\_temp\_coeff} \\
$a$ & Sandia temperature parameter \textbf{a} & \texttt{spe\_a} \\
$b$ & Sandia temperature parameter \textbf{b} & \texttt{spe\_b} \\
$\Delta T$ & Sandia temperature parameter \textbf{dT} ($^\circ$C) & \texttt{spe\_dT} \\
$G_{0...4}$ & radiation values in efficiency table (W/m$^2$)& \texttt{spe\_rad[\textit{0...4}]} \\
$G_{ref}$ & reference radiation value from efficiency table (W/m$^2$) & \texttt{spe\_reference} \\
$\eta_{0...4}$ & module efficiency values in efficiency table (\%)& \texttt{spe\_eff[\textit{0...4}]} \\
$f_d$ & \textbf{Diffuse utilization factor} & \texttt{spe\_fd} \\
\hline
\end{tabular}
\label{tab-spemodulevars}
\end{center}
\end{table}

The reference irradiance condition for the simple efficiency module model is the value that you specify, typically 1,000 W/m$^2$. The reference cell temperature is 25\degree C. 

The effective total irradiance (the diffuse utilization factor $F_d$ is adapted from the Sandia module model, and is equal to one for flat plate modules):
%lib_pvmodel.cpp 101
\begin{equation}
G = G_b + f_d(G_d+G_r)
\end{equation}

%lib_pvmodel.cpp 102
SAM determines the module's conversion efficiency $\eta_m$ by using linear interpolation on the radiation and efficiency values in the efficiency table to estimate the conversion efficiency at the effective total irradiance value $G$.

The module's DC power output:
%lib_pvmodel.cpp 103
\begin{equation}
P_{dc,m} = G \eta_m A_m \frac{\gamma_{mp,ref}}{100} (T_c - 25)
\end{equation}


\section{NOCT Cell Temperature Model} \label{sec-tcnoct}

The NOCT photovoltaic cell temperature model is from Chapter 23.3 of \citet{duffie2013}, also described in \citet{desoto2004b}. In SAM's implementation of the model, it is available with the \textbf{CEC Performance Model with Module Database} module model on the Module input page as the \textbf{NOCT cell temp model} option ($\mathtt{cec\_temp\_corr\_mode}=0$ in SSC \texttt{pvsamv1}).

\begin{table}
\begin{center}
\caption{NOCT Cell Temperature Model Variable Definitions}
\begin{tabular}{lll}
\midrule
Symbol & Description / \textbf{Name in SAM} & Name in SSC \\
\midrule
\multicolumn{3}{c}{Inputs}\\
$G$ & effective irradiance transmitted to cell (W/m$^2$)& - \\
$\tau \alpha$ & effective transmittance-absorbtance product & - \\
$v_{w}$ & wind speed (m/s) & - \\
- & \textbf{Array height} (m)& \texttt{cec\_height} \\
- & \textbf{Mounting standoff} & \texttt{cec\_standoff} \\
$T_{noct}$ & module NOCT temperature rating ($^\circ$C)& \texttt{6par\_tnoct} \\
$I_{mp}$ & module maximum power current rating (A)& \texttt{6par\_imp} \\
$V_{mp}$ & module maximum power voltage rating (V) & \texttt{6par\_vmp} \\
$A_m$ & module area in meters (m$^2$)& \texttt{6par\_area} \\
\midrule
\multicolumn{3}{c}{Output}\\
$T_c$ & cell temperature ($^\circ$C)& \texttt{subarray[\textit{n}]\_celltemp} \\
\hline
\end{tabular}
\label{tab-tempnoct}
\end{center}
\end{table}

The cell temperature is only calculated when the effective transmitted irradiance from Equation~\ref{eqn-effectivetransmittedirradiance} is greater than zero, $G>0$.

The reference module efficiency is based on the module specifications at 1,000 W/m$^2$ incident irradiance:
%lib_cec6par.cpp 469
\begin{equation}
\eta_{ref} = \frac{I_{mp} V_{mp}}{1000 A_m}
\end{equation}

The wind speed adjusted for height above the ground depends on the \textbf{Array height} option on the Module page (\texttt{cec\_height} in SSC):
%lib_cec6par.cpp 472
\begin{equation}
v_{w,adj} = \left\{
\begin{array}{ll}
0.51 v_{w} & \text{one story or lower} \\
0.61 v_{w} & \text{two stories or higher} \\
\end{array}
\right.
\end{equation}

The NOCT adjusted for mounting stand-off type depends on the \textbf{Mounting standoff} option on the Module page (\texttt{cec\_standoff} in SSC):
%lib_cec6par.cpp 476, cmod_pvsamv1.cpp 778
\begin{equation}
T_{noct,adj} =\left\{
\begin{array}{ll}
T_{noct} + 2  & \text{building integrated, greater than 3.5 in, or ground/rack mounted}\\
T_{noct} + 2 & \text{2.5 to 3.5 in} \\
T_{noct} + 6 & \text{1.5 to 2.5 in} \\
T_{noct} + 11 & \text{0.5 to 1.5 in} \\
T_{noct} + 18 & \text{less than 0.5 in} \\
\end{array}
\right.
\end{equation}

The cell temperature in degrees Celsius (\degree C):
%lib_cec6par.cpp 477
\begin{equation}
T_c = T_a + \frac{G}{800} \left(T_{noct,adj} - 20 \right) \left(1-\frac{\eta_{ref}}{\tau \alpha}\right) \frac{9.5}{5.7+3.8v_{w,adj}}
\end{equation}

\section{Heat Transfer Cell Temperature Model}\label{sec-tcheattransfer}

The heat transfer cell temperature model is from the descriptions in Chapters 4 and 5 and the FORTRAN implementation in Appendix H of \citet{neises2011}. SAM's model is adapted from this implementation, and its description is omitted from this section because of its length.

In SAM, the heat transfer cell temperature model is available with the \textbf{CEC Performance Model with Module Database} module model on the Module input page as the \textbf{Mounting specific cell temp model} option\\($\mathtt{cec\_temp\_corr\_mode}=1$ in SSC \texttt{pvsamv1}).

%inputs cmod_pvsamv1.cpp 802
\begin{table}
\begin{center}
\caption{Heat Transfer Cell Temperature Model Variable Definitions}
\begin{tabular}{ll}
\midrule
Description / \textbf{Name in SAM} & Name in SSC \\
\midrule
\multicolumn{2}{c}{Inputs}\\
\textbf{Mounting Configuration} & \texttt{cec\_mounting\_config} \\ 
\textbf{Heat Transfer Dimensions} & \texttt{cec\_heat\_transfer} \\ 
\textbf{Mounting Structure Orientation} & \texttt{cec\_mounting\_orientation} \\ 
\textbf{Module Width} (m)& \texttt{cec\_module\_width} \\ 
\textbf{Module Length} (m)& \texttt{cec\_module\_length} \\ 
\textbf{Rows of modules in array} & \texttt{cec\_array\_rows} \\ 
\textbf{Columns of modules in array} & \texttt{cec\_array\_cols} \\ 
\textbf{Temperature behind the module} ($^\circ$C) & \texttt{cec\_backside\_temp} \\ 
\textbf{Gap Spacing} (m)& \texttt{cec\_gap\_spacing} (m) \\ 
\multicolumn{2}{c}{Irradiance and Weather Inputs}\\
wind speed (m/s)&-\\
POA beam (W/m$^2$)&-\\
POA sky diffuse (W/m$^2$)&-\\
POA ground diffuse (W/m$^2$)&-\\
atmospheric pressure (mbar)&-\\
dry-bulb temperature ($^\circ$C)&-\\
dew-point temperature ($^\circ$C)&-\\
\midrule
\multicolumn{2}{c}{Output}\\
cell temperature & \texttt{subarray[\textit{n}]\_celltemp} \\
\hline
\end{tabular}
\label{tab-tempheattransfer}
\end{center}
\end{table}

\section{Sandia Cell Temperature Model}\label{sec-tcsandia}

The Sandia photovoltaic cell temperature model is the cell temperature model described in \citet{king2004} as part of the Sandia photovoltaic array performance model. In SAM's implementation of the model, it is available with the \textbf{Sandia PV Array Performance Model with Module Database} module model on the Module input page.

The Sandia cell temperature model requires the effective incident irradiance and three empirical parameters from the Sandia module library, which stores data provided by the Sandia National Laboratories Photovoltaic Test and Evaluation program \citep{sandia-testeval}.

\begin{table}
\begin{center}
\caption{Sandia Cell Temperature Model Variable Definitions}
\begin{tabular}{lll}
\midrule
Symbol & Description / \textbf{Name in SAM} & Name in SSC \\
\midrule
\multicolumn{3}{c}{Inputs}\\
$G_b$ & effective beam irradiance (W/m$^2$)& - \\
$G_d$ & effective sky diffuse irradiance (W/m$^2$) & - \\
$G_r$ & effective ground-reflected irradiance (W/m$^2$) & - \\
$v_{w}$ & wind speed (m/s) & - \\
$T_{a}$ & ambient temperature ($^\circ$C) & - \\
$a$ & Temperature coefficient \textbf{a} & \texttt{snl\_a} or \texttt{snl\_ref\_a}\\
$b$ &  Temperature coefficient \textbf{b} & \texttt{snl\_b} or \texttt{snl\_ref\_b}\\
$\Delta T$ & Temperature coefficent \textbf{dT} ($^\circ$C) & \texttt{snl\_dtc} or \texttt{snl\_ref\_dtc} \\
\midrule
\multicolumn{3}{c}{Output}\\
$T_c$ & cell temperature & \texttt{subarray[\textit{n}]\_celltemp} \\
\hline
\end{tabular}
\label{tab-tempsandia}
\end{center}
\end{table}

SAM's Module input page provides the \textbf{Module Structure and Mounting} option, which allows you to choose to use $a$, $b$, and $\Delta T$ values from the Sandia module library, choose from a set of pre-determined values for different module types and air circulation options, or provide your own ``user-defined" values. In SSC, the variable \texttt{snl\_module\_structure} determines the option, and the variables shown in Table~\ref{tab-tempsandia} with '\texttt{\_ref}' are for user-specified values. Table~\ref{tab-sandiamodstruct} shows the $a$, $b$, and $\Delta T$ values associated with each option in both SAM and SSC.

\begin{table}
\begin{center}
\caption{Sandia Module Structure Options}
\begin{tabular}{lcccc}
\midrule
Structure and Mounting & \texttt{snl\_module\_structure} & $a$ & $b$ & $\Delta T$ \\
\midrule
Use Database Values & 0 & - & - & - \\
Glass/Cell Polymer Sheet - Open Rack & 1 & -3.56 & -0.075 & 3 \\
Glass/Cell/Glass - Open Rack & 2 & -3.47 & -0.0594 & 3 \\
Polymer/Thin Film/Steel - Open Rack & 3 & -3.58 & -0.113 & 3 \\
Glass/Cell/Polymer Sheet - Insulated Rack & 4 & -2.81 & -0.0455 & 0 \\
Glass/Cell/Glass - Close Roof Mount & 5 & -2.98 & -0.0471 & 1 \\
User Defined & 6 & - & - & - \\
\hline
\end{tabular}
\label{tab-sandiamodstruct}
\end{center}
\end{table}

The effective irradiance:
\begin{equation}
G = G_b + G_d + G_r
\end{equation}

The module back temperature in degrees Celsius (\degree C):
\begin{equation}
T_m = G e^{a+b v_w} + T_a
\end{equation}

The cell temperature in degrees Celsius (\degree C):
\begin{equation}\label{eqn-sandiatemp}
T_c = T_m + \frac{G}{1000}  \Delta T
\end{equation}

%%%%%%%%%%%%%%%%%%%%%%%%%%%%%%%%%%%%%%%%%%%%%%%%%%
%%%%%%%%%%%%%%%%%%%%%%%%%%%%%%%%%%%%%%%%%%%%%%%%%%
\section{IEC 61853 Single Diode Module Model}\label{sec-iecmodule}

The IEC-61853 Single Diode model is a detailed method for predicting the performance of flat plate photovoltaic modules.  The model uses data from modules tested according to the International Electrotechnical Commission (IEC) power and rating standard, IEC 61853, \textit{Irradiance and Temperature Performance Measurements and Power Rating}.  A summary of the model follows, with full details available in~\cite{dobos2014}.

IEC-61853 defines a matrix of 23 temperature and irradiance pairs (Table~\ref{tab_testmatrix}) at which a module must be tested for maximum power $P_{mp}$, maximum power current $I_{mp}$, open circuit voltage $V_{oc}$, and short circuit current $I_{sc}$. 

\begin{table}[h!]
\begin{center}
\begin{tabular}{lllll}
 W/m$^2$ & 15$^\circ$C & 25$^\circ$C & 50$^\circ$C & 75$^\circ$C \\
\hline
1100 & & x & x & x \\
1000 & x & x & x & x \\
800 & x & x & x & x \\
600 & x & x & x & x \\
400 & x & x & x &  \\
200 & x & x & x &  \\
100 & x & x &  &  \\
\end{tabular}
\caption{IEC-61853 Module Test Matrix}
\label{tab_testmatrix}
\end{center}
\end{table}

The IEC-61853 model extends the basic single diode model which represents a photovoltaic module's current-voltage characteristic by Eqn.~\ref{eqn_singlediode}.
\begin{equation}
I = I_L - I_o\left(\exp \left[\frac{V+IR_s}{a}\right]-1\right) - \frac{V+IR_s}{R_{sh}}
\label{eqn_singlediode}
\end{equation}

This current-voltage equation is governed by five parameters: the modified diode nonideality factor $a$, the light current $I_L$, the reverse saturation current $I_o$, the series resistance $R_s$, and the shunt resistance $R_{sh}$.  The modified nonideality factor representing the whole module is related to the diode factor $n$ of a single cell by Eqn.~\ref{eqn_a}, where $N_s$ is the number of cells in series.
\begin{equation}\label{eqn_a}
a = N_s\frac{kT}{q}n
\end{equation}

The five parameters are measured at STC, and are translated to operating conditions via auxiliary equations.  The standard CEC single diode model in SAM implements Eqns.~\ref{eqn_aux1}-\ref{eqn_aux5} to translate the parameters given effective irradiance $S$ and cell temperature $T_c$, with $S_{ref}=1000$~W/m$^2$, $T_{ref}=25^\circ$C, and $E_{g,ref} = 1.121$~eV.  The series resistance $R_s$ is assumed to be constant.

\begin{equation}\label{eqn_aux1} a = a_{ref}\cdot T_c / T_{ref} \end{equation}
\begin{equation}\label{eqn_aux2} I_L = \frac{S}{S_{ref}}\cdot(I_{L,ref} + \alpha(T_c-T_{ref})) \end{equation}
\begin{equation}\label{eqn_aux3}
\frac{I_o}{I_{o,ref}} = \left[ \frac{T_{c}}{T_{ref}} \right] ^ 3 \exp \left[ \frac{1}{k} \left( \frac{E_g}{T}\Bigg|_{T_{ref}} - \frac{E_g}{T}\Bigg|_{T_{c}} \right) \right]
\end{equation}
\begin{equation}\label{eqn_aux4}
\frac{E_g}{E_{g,ref}} = 1 - 0.0002677(T_{c} - T_{ref})
\end{equation}
\begin{equation}\label{eqn_aux5}
R_{sh} = R_{sh,ref}\cdot\frac{S_{ref}}{S}
\end{equation}

Model parameters result from simultaneously solving the diode equation at five points on the STC I-V curve: short circuit ($V=0, I=I_{sc}$), open circuit ($V=V_{oc}, I=0$), maximum power ($V=V_{mp}, I=I_{mp}$), derivative of maximum power $\frac{d(IV)}{dV}|_{mp}=0$, and open circuit temperature coefficient $\beta = \frac{dV}{dT}|_{I=0}$.  Details are given in \cite{desoto2004a} and \cite{dobos2012a}.

We assume that the standard single diode model equation is capable of representing actual I-V curve data at operating conditions.  Experience suggests that the primary reason for the single diode model's inaccuracies emerges from the choices of auxiliary equations that translate the five parameters to operating conditions.  Consequently, the five parameters are assumed to be independent functions of temperature and irradiance.  We estimate values for $a$, $I_L$, $I_o$, $R_s$, and $R_{sh}$ at each IEC-61853 test condition, and fit the results to new auxiliary equations that are functions of both temperature and irradiance .

At each test measurement in the IEC-61853 matrix, three points on the actual I-V curve are known.  Additionally, the derivative of maximum power $\frac{d(IV)}{dV}|_{mp}=0$ condition can be applied, yielding four constraint equations.  Reasonably assuming that the cell diode factor $n$ should remain constant across temperature and irradiance - after all, it aims to represent an intrinsic physical property of the diode junction - the four constraints are sufficient to solve for $I_L$, $I_o$, $R_s$, and $R_{sh}$.  The modified diode factor $a$ is calculated by Eqn.~\ref{eqn_a}.

Next, an estimate must be made for the cell diode factor $n$.  The empirical Sandia PV Array Performance Model provides an equation for open circuit voltage that incorporates the diode factor, cell temperature, and the open circuit temperature coefficient $\beta$.  The Sandia $V_{oc}$ equation is asymptotically equivalent to the single diode model equation~\cite{hansen2013}, and is analogous to the ``$V_{oc}$ Method'' for cell temperature described in IEC-60904-9.  While this equation is best suited for crystalline silicon modules, the Sandia model has been shown to be suitable for all cell types~\cite{sapm}, and thus it is readily used here.  Rearranging terms yields an equation for $n$ (Eqn.~\ref{eqn_snlvoc}) at any temperature $T_c$ and irradiance $S$.  $V_T$ is thermal voltage $V_T=kT_c/q$.  The temperature coefficient $\beta$ at 1000 W/m$^2$ is obtained directly by calculating the change in $V_{oc}$ as a function of temperature from the test data.  Eqn.~\ref{eqn_snlvoc} is evaluated at all temperature $T_c$ and irradiance $S$ conditions in the IEC-61853 test matrix, and the average value yields the final diode factor $n$ that is used in subsequent calculations.

\begin{equation}\label{eqn_snlvoc}
n = \frac{ (V_{oc} - \beta( T_c - T_{ref} ) - V_{oc,ref}) }
            { N_s \cdot V_T \cdot \ln( S/S_{ref} ) }
\end{equation}

With an estimate for $n$, the four constraint equations can now be solved at each test condition.  The multidimensional nonlinear Newton's method solves $\mathbf{F}(\mathbf{x})=0$, where $\mathbf{x} = \left[ I_L,  I_o, R_s. R_{sh} \right]'$, and 

\begin{equation}
\mathbf{F} = \left[ \begin{array}{l} 
I_L - I_o\left(\exp\left[\frac{I_{sc}R_s}{a}\right]-1\right) - \frac{I_{sc}R_s}{R_{sh}} - I_{sc} \\ 

I_L - I_o\left(\exp \left[\frac{V_{oc}}{a}\right]-1\right) - \frac{V_{oc}}{R_{sh}} \\ 

I_L - I_o\left(\exp \left[\frac{V_{mp}+I_{mp}R_s}{a}\right]-1\right) - \frac{V_{mp}+I_{mp}R_s}{R_{sh}} - I_{mp} \\

I_{mp} - V_{mp}\left[ \frac{  \frac{I_o}{a}\exp\left[ \frac{V_{mp} + I_{mp}R_s}{a}\right] + \frac{1}{R_{sh}} } { 1  + \frac{I_o R_s}{a}\exp \left[ \frac{V_{mp} + I_{mp}R_s}{a}\right] + \frac{R_s}{R_{sh}} } \right] \\
\end{array} \right]
\end{equation}

With Newton's method successfully solving for $I_L$, $I_o$, $R_s$, and $R_{sh}$ at each test irradiance and temperature, we fit new auxiliary equations for series resistance (Eqn.~\ref{eqn_auxrsnew}) and shunt resistance (Eqn.~\ref{eqn_auxrshnew}).  Successive nonlinear least-squares fitting is applied to estimate the parameters $C_{1,2,3}$ and $D_{1,2,3}$.

\begin{equation}\label{eqn_auxrsnew}
R_s = D_1 + D_2(T_c-T_{ref}) + D_3\left( 1-\frac{S}{S_{ref}}\right)\left(\frac{S_{ref}}{S}\right)^2
\end{equation}

\begin{equation}\label{eqn_auxrshnew}
R_{sh} = C_1 + C_2\left[ \left(\frac{S_{ref}}{S}\right)^{C_3}-1 \right]
\end{equation}

The standard auxiliary equations (Eqns.~\ref{eqn_aux1}-\ref{eqn_aux4}) are retained for $a$, $I_L$, $I_o$, and $E_g$, though we note that $E_{g,ref}$ is fitted to the best fit for $I_o$ across the operating range.  

Finally, with a full set of equations describing the variation of single diode model parameters across the entire range of irradiance and temperature conditions, the model calculates each parameter's value at every time step and uses a golden search maximization process to find the maximum power ($P=IV$) by iteratively solving the current-voltage relation to determine $I$ and $V$.  

The IEC-61853 model gives the user a choice of standard glass or glass with an antireflective coating.  The model applies an angle of incidence correction to adjust the direct beam irradiance to account for reflection losses.  This model is the same as used in PVWatts version 5, and is an extension of approach in~\cite{desoto2004a}.

For standard glass, we use a single slab model, calculating the transmittance through glass with index of refraction of 1.526.  This follows the treatment in ~\cite{desoto2004a}, with the simplification of removing the absorptance term which is determined below to have a negligible effect.

For glass with an antireflective coating, a two slab approach is used to model the improvement in angular response.  The two slab model involves predicting the transmittance of irradiance through two materials using a physical representation for unpolarized radiation described in~\cite{desoto2004a} twice: once for the anti-reflective (AR) coating, and subsequently for the glass cover.
        
\begin{figure}[h!]
\begin{center}
\scalebox{0.5}{\includegraphics{figures/fig_twoslab_model.pdf}}
\end{center}
\caption{Diagram of two slab module cover reflection loss model}
\label{fig_twoslab_model}
\end{figure}

First, the angle of refraction $\theta_2$ into the AR coating is calculated with Snell's law given angle of incidence $\theta_1$:

\begin{equation}\label{snell1}
\theta_2 = \arcsin\left( \frac{n_{air}}{n_{AR}} \sin(\theta_1 ) \right)
\end{equation}

Next, the transmittance through the AR coating is calculated from Fresnel's equation for non-reflected unpolarized radiation, which takes the average of parallel and perpendicular components.

\begin{equation}\label{fresnel1}
\tau_{AR} = 1 - 0.5
        \left( \frac{\sin(\theta_2-\theta_1)^2}{\sin(\theta_2+\theta_1)^2} 
               + \frac{\tan(\theta_2-\theta_1)^2}{\tan(\theta_2+\theta_1)^2}  \right)
\end{equation}

The angle of refraction into the glass cover $\theta_3$ is again determined from Snell's law.

\begin{equation}\label{snell1}
\theta_3 = \arcsin\left( \frac{n_{AR}}{n_{glass}} \sin(\theta_2 ) \right)
\end{equation}

The transmittance through the glass is calculated similarly given $\theta_2$ and $\theta_3$.

\begin{equation}\label{fresnel1}
\tau_{glass} = 1 - 0.5
        \left( \frac{\sin(\theta_3-\theta_2)^2}{\sin(\theta_3+\theta_2)^2} 
               + \frac{\tan(\theta_3-\theta_2)^2}{\tan(\theta_3+\theta_2)^2}  \right)
\end{equation}

Finally, the effective transmittance through the AR coated module cover is given by:

\begin{equation}\label{tau_eff}
\tau_{cover} = \tau_{AR}\tau_{glass}
\end{equation}

\cite{desoto2004a} suggests using Bouguer's law to estimate the absorption, but for typical dimensions ($\sim$2~mm) and extinction coefficients ($K\approx 4$) in this application, the absorption is predicted to be less than 0.1~\% and is thus ignored.

Normalizing the response curve to normal incidence shows that the model predicts an improvement in light capture at high incidence angles for AR glass, which agrees with data from manufacturers of AR coated modules (Fig.~\ref{fig_twoslab_norm}).

\begin{figure}[h!]
\begin{center}
\scalebox{0.8}{\includegraphics{figures/fig_twoslab_norm.pdf}}
\end{center}
\caption{Two slab model angular response compared with single slab model - normalized}
\label{fig_twoslab_norm}
\end{figure}

The AR glass option will predict slightly higher output compared with the standard option due to the improved angular response.  The normal incidence transmittance of the module cover is assumed to be captured in the nameplate rating of the system. 

Finally, we note that the IEC-61853 module model handles spectral effects using the Sandia effective air mass polynomial to calculate an air mass modifier coefficient that reduces the available effective irradiance to the cell. The module temperature is estimated using the standard CEC single diode model approach.


%%%%%%%%%%%%%%%%%%%%%%%%%%%%%%%%%%%%%%%%%%%%%%%%%%
%%%%%%%%%%%%%%%%%%%%%%%%%%%%%%%%%%%%%%%%%%%%%%%%%%
\chapter{Array DC Output}\label{sec-arraydcoutput}

The photovoltaic array may consist of a single subarray, or up to four subarrays. Each subarray may have different orientation, tracking and ground coverage ratio (GCR). SAM calculates the each subarray's DC output by multiplying a single module's DC output (Section~\ref{sec-module}) by the number of modules in the array. This assumes that all of the modules in the array operate uniformly at the maximum power point of a single module. 

SAM does not explicitly calculate electrical losses due to maximum power point mismatches between modules that would occur in a real system. You can account for those losses using the DC module mismatch loss input  (Section~\ref{sec-dclosses}).

For an array with more than one subarray, the array's DC output is the sum of the subarray output. In some cases, SAM can estimate losses due to maximum power point mismatches between the subarrays (Sections~\ref{sec-dcstringvoltage} and \ref{sec-mismatch}).

\begin{table}
\begin{center}
\caption{Array DC Output Variable Definitions}
\begin{tabular}{lll}
\midrule
Symbol & Description / \textbf{Name in SAM} & Name in SSC \\
\midrule
\multicolumn{3}{c}{Inputs}\\
$V_{mp}$ & module DC voltage (V)& - \\
$P_{mp}$ & module DC power (W) & - \\
$N_{modules}$ & \textbf{Modules per string} & \texttt{modules\_per\_string} \\
$N_{parstrings}$ & \textbf{Strings in parallel} & \texttt{strings\_in\_parallel} \\
$N_{strings,n}$ & \textbf{Strings allocated to subarray} \textit{n} & \texttt{subarray[\textit{n}]\_nstrings} \\
$N_{sub}$ & number of enabled subarrays & - \\
$L_{mismatch,n}$ & \textbf{Module mismatch} subarray \textit{n} (\%)&  \texttt{subarray[\textit{n}]\_dcloss} \\
$L_{diodeconn,n}$ & \textbf{Diodes and connections} subarray \textit{n} (\%)&  \texttt{subarray[\textit{n}]\_dcloss} \\
$L_{dcwiring,n}$ & \textbf{DC wiring} subarray \textit{n} (\%)&  \texttt{subarray[\textit{n}]\_dcloss} \\
$L_{tracking,n}$ & \textbf{Tracking error} subarray \textit{n} (\%)&  \texttt{subarray[\textit{n}]\_dcloss} \\
$L_{nameplate,n}$ & \textbf{Nameplate} subarray \textit{n} (\%)&  \texttt{subarray[\textit{n}]\_dcloss} \\
$L_{dcoptimizer,n}$ & \textbf{DC power optimizer loss} subarray \textit{n} (\%)&  \texttt{subarray[\textit{n}]\_dcloss} \\
\midrule
\multicolumn{3}{c}{Outputs}\\
$V_{dc,n}$ & \textbf{Subarray [\textit{n}] DC string voltage} (V)& \texttt{subarray[\textit{n}]\_dc\_voltage} \\
$V_{dc}$ & \textbf{Inverter DC input voltage} (V)& \texttt{inverter\_dc\_voltage} \\
$P_{dc}$ & \textbf{Array power (DC)} (kW)& \texttt{dc\_net} \\
\hline
\end{tabular}
\label{tab-arraydcoutputvars}
\end{center}
\end{table}

\section{DC String Voltage}\label{sec-dcstringvoltage}

SAM calculates the array's DC string voltage in each time step to determine the inverter's input voltage. SAM assumes that each module operates at its maximum-power-point (MPPT) voltage. For systems with more than one subarray, SAM assumes that all subarrays have the same number of modules per string. The DC string voltage $V_{dc}$ in a given time step is the module MPPT voltage times the number of modules per string:
%cmod_pvsamv1.cpp 1481
\begin{equation}
V_{dc} = V_{dc,m} N_{modules}
\end{equation}

However, under certain conditions, SAM adjusts the DC string voltage:

\begin{itemize}
\item The string voltage falls outside of the range defined by the inverter's minimum and maximum input MPPT voltage ratings (Section~\ref{sec-invclip})
\item The system consists of more than one subarray, and the subarrays have different maximum power points, for example because of different orientation or type of tracking (Section~\ref{sec-subarraymismatch}).
\end{itemize}

\section{Gross DC Power Output}

For systems with a single subarray, the array's gross DC power output is the output of a single module multiplied by the number of modules in the array, and, for fixed arrays or arrays with  with one-axis tracking tracking, by the optional fixed self shading DC loss factor (Equation~\ref{eqn-selfshadedcderate}):
%cmod_pvsamv1.cpp 1499, 1501 
\begin{equation}
P_{dc,gross} = N_{modules}~N_{parstrings}~P_{dc,m}~F_{dcss}
\end{equation}

For systems with more than one subarray, the array's gross DC output is the sum of the subarray values:
%cmodpvsamv1.cpp 1505
\begin{equation}
P_{dc,gross} = \sum_{n=1}^{N_{sub}} P_{dc,gross,n}
\end{equation}

The gross DC power output is an intermediate variable that SAM uses for internal calculations but does not report in the model outputs.

\section{DC Electrical Losses} \label{sec-dclosses}

SAM models electrical losses on the DC side of the system using a single DC loss factor for each subarray in the system that applies to the gross DC power output. The DC loss factor for each subarray $F_{dc,n}$ is calculated from the six loss percentages on the Losses input page:
%cmod_pvsamv1.cpp 981
\begin{align}\label{eqn-dcderate}
F_{dc,n} &= F_{dc1,n}~F_{dc2,n}~F_{dc3,n}~F_{dc4,n}~F_{dc5,n}~F_{dc6,n} \notag\\
F_{dc1,n} &= 1-\frac{L_{\textit{mismatch,n}}}{100} \notag\\
F_{dc2,n} &= 1-\frac{L_{\textit{diodeconn,n}}}{100} \notag\\
F_{dc3,n} &= 1-\frac{L_{\textit{dcwiring,n}}}{100} \notag\\
F_{dc4,n} &= 1-\frac{L_{\textit{tracking,n}}}{100} \notag\\
F_{dc5,n} &= 1-\frac{L_{\textit{nameplate,n}}}{100} \notag\\
F_{dc6,n} &= 1-\frac{L_{\textit{dcoptimizer,n}}}{100}
\end{align}

Note that the single DC power optimizer loss applies to all subarrays.

\section{DC Snow Losses}\label{sec-snow}

The snow cover model estimates the portion of the array that is covered by snow and calculates a loss caused by the snow that applies to the subarray's gross DC power output. The model calculates a reduction in subarray power output instead of a reduction in incident irradiance because it was developed empirically by measuring system electrical output for several photovoltaic systems installed in snowy climates, as described in \citet{marion-snowmodel} and \citet{ryberg-snowmodel}.

To use the snow model in SAM, check \textbf{Estimate losses due to snow} on the Shading and Snow input page. Be sure to use a weather file with snow depth data. The \textbf{Maximum snow depth} value on the Location and Resource page indicates whether there is snow depth data in the weather file: A value of "NaN" indicates there is no snow depth data in the file. Also make sure that the \textbf{Number of modules along side} on the Shading and Snow input page is consistent with the number of modules in the subarray. In SSC, set $\texttt{en\_snow\_model} = 1$ to enable the snow model, and check that the value of \texttt{subarray[\textit{n}]\_nmody} is correct.

\begin{table}
\begin{center}
\caption{Snow Loss Model Variable Definitions}
\begin{tabular}{lll}
\midrule
Symbol & Description / \textbf{Name in SAM} & Name in SSC \\
\midrule
\multicolumn{3}{c}{Inputs}\\
- & \textbf{Estimate losses due to snow} & \texttt{en\_snow\_model} \\
$M_{side}$ & \textbf{Number of modules along side of row} &  \texttt{subarray[\textit{n}]\_nmody}\\
$\beta_0$ & \textbf{tilt} & \texttt{subarray[\textit{n}]\_tilt}\\
$\beta_s$ & surface tilt angle (calculated output) & \texttt{subarray[\textit{n}]\_surf\_tilt}\\
$D_{snow}$ & snow depth (cm) & - \\
$T_{dry}$ & ambient dry bulb temperature ($^\circ$C) & - \\
$G_{b}$ & POA beam irradiance (W/m$^2$)& - \\
$G_{d}$ & POA diffuse irradiance (sky and ground) & - \\
$\mathit{sunup}$ & \textbf{Sun up over horizon (0/1)} & \texttt{sunup} \\
\midrule
\multicolumn{3}{c}{Outputs}\\
$P_{snow}$ & \textbf{Power loss due to snow} (W) & \texttt{dc\_snow\_loss} \\
$C_{snow}$ & \textbf{Snow depth from weather file} (cm) & \texttt{snowdepth} \\
$P_{snow,n}$ & \textbf{Subarray [\textit{n}] Power loss due to snow} (W) &  \texttt{subarray[\textit{n}]\_snow\_loss} \\
$C_{snow,n}$ & \textbf{Subarray [\textit{n}] Snow coverage} (0..1) &  \texttt{subarray[\textit{n}]\_snow\_coverage} \\
\hline
\end{tabular}
\label{tab-snowvars}
\end{center}
\end{table}

Snow coverage is a fraction of the height of a subarray row as shown in Figure \ref{fig-snowCoverage}, where $C_{snow}=1$ represents a subarray fully covered by snow. The snow model only considers snow sliding off of the array as a removal mechanism. It does not consider other snow removal processes such as melting or removal by wind.

\begin{figure}
\begin{center}
\includegraphics[scale=0.5]{snow-coverage-diagram}
\caption{Snow coverage diagram on PV arrays. Snow coverage is shown in blue and is measured as the percent of each row's height which is covered (i.e. 0\% coverage corresponds to no snow on the paned while 100\% coverage corresponds to a completely covered panel). The coverage in the figure would correspond to roughly 40\%.}
\label{fig-snowCoverage}
\end{center}
\end{figure}

The model makes the following checks and issues a simulation message if the subarray tilt angle $\beta_S$ is out of range, and stops the simulation if the snow depth $D_{snow}$ is out of range for more than 500 time steps:
\begin{align*}
10^\circ &< \beta_{S} < 45^\circ \notag \\
0~\text{cm} &< D_{snow} < 610~\text{cm}
\end{align*}

SAM's snow model algorithm is adapted from \citet{marion-snowmodel}. For each time step:

\textbf{1. Determine snow coverage}

If \textit{all} of the following conditions are met, then set $C_{snow}=1$:

\begin{itemize}
\item Snow depth $D_{snow}$ is greater than its value in the previous time step.
\item $D_{snow}$ in the current time step is greater than 1 cm.
\item The difference between $D_{snow}$ in the current and previous time steps is greater than 1 cm/hr.
\end{itemize}

If any of the above conditions are not met, set $C_{snow}$ to the previous time step's value.

In addition, the following modification to the algorithm described in \citet{marion-snowmodel} ensures that snow does not remain on a horizontal or slightly tilted subarray when there is no snow on the ground: 

\begin{itemize}
\item If $\beta_0<10^\circ$ and $D_{snow} < 1~\text{cm}$, then set $C_{snow}=0$.
\end{itemize}

\textbf{2. Decrease amount of snow coverage if sliding occurs}

The following test determines whether the ambient temperature for the current time step is high enough to initiate snow sliding:

\begin{equation}\label{eqn-snowslidetemp}
T_{dry}> \frac{G_{b}+G_{d}}{m}
\end{equation}

The constant parameter $m=-80\ W/m^2/^\circ C$ and relationship in Equation~\ref{eqn-snowslidetemp} were determined experimentally as described in \citet{marion-snowmodel}.

If snow sliding occurs, then reduce the amount of snow coverage $C_{snow}$ on the subarray by A. The model divides the row height into ten tenths so that the snow slides in increments of one tenth the length of the row's side:

\begin{equation}	\label{eqn-snowSlideAmount}
	A = 0.10 \left(1.97  \sin \beta_{S} \right)
\end{equation}

If sliding does not occur, set the snow coverage $C_{snow}$ for the current time step to its value in the previous time step. 

For a simulation with subhourly time steps, the snow coverage $C_{snow}$ is reduced by the time step's fraction of an hour to ensure that the reduction over the hour is 10\% of the previous hours snow coverage value. For example, for 15-minute time steps, the reduction would be $\frac{15}{60}A$.

Because SAM does not calculate surface angles for night-time time steps ($sunup = 0$), the snow model sets the night-time subarray tilt angle to the base value $\beta_0$ to ensure it accounts for snow sliding that may occur at night. Note that $\beta_0$ is the the subarray tilt angle input and, for one-axis tracking arrays, is different from the surface tilt angle $\beta_S$ calculated based on the position of the sun.

\textbf{3. Calculate reduction in subarray's DC output}

After determining the snow coverage $C_{snow}$, the model calculates the resulting reduction in DC power for the subarray as the loss factor $F_{snow}$. SAM assumes that if a string is partially covered by snow, its output is zero. For example, Figure~\ref{fig-snowCoverage} shows three rows of an array with two modules along the side of each row. In this case, SAM considers each row to consist of two parallel strings of modules. One string in each row is partially covered by snow, so the the snow loss would be 50\%, so that array's DC output would be half of what it would be with no snow.

To calculate the DC snow loss factor,
\begin{equation}
F_{snow} = 1-\frac{\lceil C_{snow}M_{side}\rceil}{M_{side}}
\end{equation}

\section{Array DC Power}

The DC power output of each subarray is each subarray's gross DC power output multiplied by the subarray's DC loss factors:
%cmod_pvsamv1.cpp 1505
\begin{equation}
P_{dc,n} = P_{dc,gross,n}~F_{dc,n}~F_{snow,n}
\end{equation}

The array's DC power output is sum of the subarray DC power output values :
%cmod_pvsamv1.cpp 1506
\begin{equation}
P_{dc} = \sum_{n=1}^{N_{sub}} P_{dc,n}
\end{equation}

\section{Maximum Power Point Tracking}

SAM does not explicitly model a maximum power point tracker or other power conditioning equipment. However,  SAM does assume that the array operates at its maximum power point, so it implicitly assumes that photovoltaic system is equipped with a maximum power point tracking system. 

You can account for electricity consumption of maximum power point tracking equipment and other related losses using the DC losses (see Section~\ref{sec-dclosses}).

\section{Subarray Mismatch Losses}\label{sec-subarraymismatch}
For systems with more than one subarray, when the subarrays have different orientations, tracking, shading factors, or DC loss factors, the maximum power point of each subarray is different. In such cases, SAM calculates the array's DC output voltage in one of two ways:
\begin{itemize}
\item By averaging the values of the subarray string voltages (available with all module models): 
%cmod_pvsamv1.cpp 1488
\begin{equation}
V_{dc} =  \frac{1}{N_{sub}}\sum_{n=1}^{N_{sub}} V_{dc,n}
\end{equation}
\item By running the module model iteratively for each subarray to determine the array's maximum string voltage (available with the CEC and IEC 61853 module models only) as described below.
\end{itemize}

When you choose the CEC or IEC 61853 module model, the \textbf{Calculate maximum power voltage for array and associated losses due to subarray mismatch} check box at the bottom of the System Design input page ($\mathtt{enable\_mismatch\_vmax\_calc}=1$ in the SSC \texttt{pvsamv1} module) determines what method SAM uses to calculate the array's maximum power point DC voltage.

For a system with more than one subarray, SAM can estimate losses due to maximum power point mismatches between subarrays, but only for the single-diode equivalent circuit module models (CEC and IEC 61853 models) because they represent the subarray's I-V curve as a continuous function, which makes it possible to estimate the subarray power at points that are off of the maximum power point. The point-value models (simple efficiency and Sandia module) calculate values at discrete points on the subarray's I-V curve, and are not suitable for estimates of module power at other points on the curve \citep{dobos2012b}.

The subarray mismatch loss algorithm \citep{dobos2012b} is intended to model situations that may occur in residential rooftop systems when the array is installed on different parts of the roof with different orientations and connected to a single inverter. The group of modules on each roof surface is a subarray, and may have its own set of parameters defining its tilt and azimuth angles, tracking, shading factors, and DC losses. 

In larger systems, modules are typically oriented uniformly across the entire array. SAM's subarray mismatch algorithm is not suitable for such systems. SAM does not calculate mismatch losses due to maximum power point losses between individual modules in the array or in each subarray.

%cmod_pvsamv1.cpp 1405
The algorithm determines the system's maximum power point in a given time step by running the CEC or IEC 61853 module model over a range of maximum power point voltage values to find the voltage that results in the complete array's highest maximum power given each of its subarray's effective incident irradiance, module performance and temperature parameters.

The subarray mismatch algorithm runs the cell temperature model (Section~\ref{sec-celltempoptions}) and module model separately from the model runs for the main system energy calculations. Therefore, in addition to the input variables listed in Table~\ref{tab-arraydcoutputvars}, the algorithm also uses the inputs listed in Tables~\ref{tab-modulevars} and \ref{tab-cecmodulevars}.

%%%%%%%%%%%%%%%%%%%%%%%%%%%%%%%%%%%%%%%%%%%%%%%%%%
%%%%%%%%%%%%%%%%%%%%%%%%%%%%%%%%%%%%%%%%%%%%%%%%%%
\chapter{Inverter AC Output}\label{sec-inverter}

SAM uses one of the inverter models listed in Section~\ref{sec-inverteroptions} to calculate the total AC power output of all of the inverters in the system. For systems with more than one inverter, SAM models the inverters as a single large inverter that operates with the array DC string voltage as the inverter's input voltage. SAM cannot model a system with more than one type of inverter.

The inverter models calculate the inverter's DC-to-AC power conversion efficiency at rated and part-load operating power. They do not explicitly account for the effect of temperature on inverter performance or for the impact on inverter performance of power factor control or grid outages. 

SAM accounts for two kinds of inverter losses, or "clipping" described in Section~\ref{sec-invclip}.

\begin{table}
\begin{center}
\caption{Inverter Model Variable Definitions}
\begin{tabular}{lll}
\midrule
Symbol & Description / \textbf{Name in SAM} & Name in SSC\\
\midrule
\multicolumn{3}{c}{Inputs}\\
$P_{dc}$ & \textbf{Array power} (DC) (kW)& \\
$N$ & number of inverters & \texttt{inverter\_count} \\
$V_{dc}$ & DC string voltage (V)& - \\
$V_{dc,max}$ & \textbf{Maximum DC voltage} (V)& [...]\texttt{\_vdcmax} \\
$I_{dc,max}$ & \textbf{Maximum DC current} (A) &  [...]\texttt{\_pdco} \\
$V_{dc,0}$ & \textbf{Nominal DC voltage} (V)&  [...]\texttt{\_vdco} \\
$P_{s,0}$ & \textbf{Power consumption during operation} (W)&  [...]\texttt{\_pso} \\
$P_{nt,0}$ & \textbf{Power consumption at night} (W)&  [...]\texttt{\_pnt} \\
$P_{ac,0}$ & \textbf{Maximum AC power} (W)&  [...]\texttt{\_paco} \\
$P_{dc,0}$ & \textbf{Maximum DC power} (W) &  [...]\texttt{\_pdco} \\
$V_{ac,0}$ & \textbf{Nominal AC voltage} (V) &  [...]\texttt{\_pdco} \\
$V_{dc,mppt max}$ & \textbf{Minimum MPPT DC voltage} (V) &  \texttt{mppt\_low\_inverter} \\
$V_{dc,mppt min}$ & \textbf{Maximum MPPT DC voltage} (V) &  \texttt{mppt\_hi\_inverter} \\
\midrule
\multicolumn{3}{c}{Outputs}\\
$P_{ac}$ & \textbf{Power generated by system} (kW) & \texttt{gen} \\
$\eta_{inv}$ & \textbf{Inverter efficiency} (\%)& \texttt{inv\_eff}  \\
$P_{clip}$ & \textbf{Inverter clipping loss} (kW)& \texttt{inv\_cliploss}  \\
$P_{so}$ & \textbf{Inverter power consumption loss} (kW)& \texttt{inv\_psoloss}  \\
$P_{nt}$ & \textbf{Inverter night time loss} (kW)& \texttt{inv\_ntloss}  \\
\midrule
\multicolumn{3}{c}{[...] indicates the inverter model name in SSC: \texttt{inv\_snl}, \texttt{inv\_ds}, or \texttt{inv\_pd} }\\

\end{tabular}
\label{tab-invertervars}
\end{center}
\end{table}

\section{Inverter Models}\label{sec-inverteroptions}

SAM calculates the a single inverter's AC output with one of three inverter models \citep{blair2013}. Table~\ref{tab-invertervars} lists the variables that are the same for the all three inverter models. Variables that apply to each specific inverter module are listed in the relevant section:

\begin{itemize}
\item \textbf{Inverter CEC Database} (Section~\ref{sec-sandiainverter}), also called the Sandia inverter model, is an empirical model that uses manufacturer specifications with four empirically derived coefficients $C_0$, $C_1$, $C_2$, $C_3$ \citep{king2007}. It uses parameters from a database maintained by the California Energy Commission \citep{gsc2014a}.
\item \textbf{Inverter Datasheet} (Section~\ref{sec-sandiainverter}) is an implementation of the Sandia inverter model that sets the values of the empirical coefficients to zero so that the inverter can be modeled with only manufacturer specifications.
\item \textbf{Inverter Part Load Curve}) (Section~\ref{sec-partloadinverter}) is a model developed by NREL for SAM that uses a table of efficiency values at different inverter load levels to represent the inverter's performance.
\end{itemize}

In the SSC \texttt{pvsamv1} module, the input variable \texttt{inverter\_model} determines the module model as shown in Table~\ref{tab-invertersubmodels}.



\begin{table}
\begin{center}
\caption{Inverter Models in SSC}
\begin{tabular}{lc}
\midrule
Name in SAM & \texttt{inverter\_model} \\
\midrule
Inverter CEC Database (Sandia) & 0 \\
Inverter Datasheet & 1 \\
Inverter Part Load Curve & 2 \\
\hline
\end{tabular}
\label{tab-invertersubmodels}
\end{center}
\end{table}

%%%%%%%%%%%%%%%%%%%%%%%%%%%%%%%%%%%%%%%%%%%%%%%%%%
%%%%%%%%%%%%%%%%%%%%%%%%%%%%%%%%%%%%%%%%%%%%%%%%%%
\section{Sandia Inverter Model}\label{sec-sandiainverter}

SAM's Sandia Inverter model is an implementation of the empirical model described in \citet{king2007}. As explained in Section~\ref{sec-inverteroptions}, SAM has two implementations of the Sandia inverter model: One uses  a set of reference parameters from the Sandia Inverters library with data provided by the California Energy Commission \citep{gsc2014a}, and the other uses manufacturer specifications without the additional $C$ coefficients included in the library. Both versions are described in this section.

In SSC, the implementation of the Sandia inverter model described here is part of the \texttt{pvsamv1} module. The SSC module \texttt{pvsandiainv} is a standalone implementation of the Sandia inverter model that is not used by SAM, but is available as part of the SSC software development kit.

The Sandia Inverter model variables are listed in Tables~\ref{tab-invertervars} and~\ref{tab-sandiainvertervars}.

\begin{table}
\begin{center}
\caption{Sandia Inverter Model Inputs}
\begin{tabular}{lll}
\midrule
Symbol & Description / \textbf{Name in SAM} & Name in SSC \\
\midrule
$c_0$ & Curvature between AC power and DC power (1/W) & \texttt{inv\_snl\_c0} \\
$c_1$ & Coefficient of $P_dc,0$ variation with DC input voltage (1/V) &  \texttt{inv\_snl\_c1} \\
$c_2$ & Coefficient of $P_{so}$ variation with DC input voltage (1/V) &\texttt{inv\_snl\_c2} \\
$c_3$ & Coefficient of $C_0$ variation with DC input voltage (1/V) & \texttt{inv\_snl\_c3} \\
\hline
\end{tabular}
\label{tab-sandiainvertervars}
\end{center}
\end{table}

For the \textbf{Inverter Datasheet} option ($\mathtt{inverter\_model}=1$ in SSC), there is an additional input for the inverter's nominal DC-to-AC power conversion efficiency $\eta_{inv,0}$. SAM calculates the maximum DC input power from the rated efficiency and rated maximum AC output power values:
%cmod_pvsamv1.cpp 1006
\begin{equation}
P_{dc,0} = \frac{P_{ac,0}}{\eta_{inv,0}}
\end{equation}

The Sandia inverter model parameters:
%lib_sandia.cpp 329
\begin{align}
A &=P_{dc,0} \left[ 1+ C_1 ( V_{dc} - V_{dc,0} )\right] \notag\\
B &= P_{s,0}  \left[ 1+ C_2 ( V_{dc} - V_{dc,0} )\right] \notag\\
C &= C_0  \left[ 1+ C_3 ( V_{dc} - V_{dc,0} )\right]
\end{align}

The Sandia inverter model equation:
%lib_sandia.cpp 333
\begin{equation}\label{eqn-sandiainverter}
P_{ac} = \left[ \frac{P_{ac,0}}{A-B} - C ( A - B ) \right] ( P_{dc} - B ) + C ( P_{dc} - B )^2
\end{equation}

For the special case of the model that is the \textbf{Inverter Datasheet} option in SAM ($\mathtt{inverter\_model} = 1$ in SSC), the Sandia model $C$ coefficients are all set to zero so that Equation~\ref{eqn-sandiainverter} reduces to:
\begin{equation}
P_{ac} = \frac{P_{ac,0}}{P_{dc,0}-P_{s,0}}~( P_{dc} - P_{so,0} )^2
\end{equation}

The coefficient $B$ accounts for operating power loss. By setting $B=0$ in Equation~\ref{eqn-sandiainverter}, the operating power less operating power losses is:
%lib_sandia.cpp 340
\begin{equation}
P_{ac,s=0} = \left[ \frac{P_{ac,0}}{A} - C A \right] P_{dc} + C P_{dc}^2
\end{equation}

The operating power losses (inverter self-consumption):
%lib_sandia.cpp 341
\begin{equation}
P_{s} = \left\{
\begin{array}{ll}
P_{ac,s=0} - P_{ac} & \text{(\textbf{Inverter CEC Database} option, and } \\
 & \text{~\textbf{Inverter Datasheet} option with manufacturer efficiency)} \\
0 & \text{(\textbf{Inverter Datasheet} option with weighted efficiency)}
\end{array}
\right.
\end{equation}

For the \textbf{Inverter Datasheet} option, the operating losses are zero.

SAM considers the inverter to be operating at night when the DC input power is less than the operating power losses $P_{dc} < P_{so}$. At night:
%lib_sandia.cpp 345
\begin{align}
P_{ac} &= -P_{nt,0} \notag\\
P_{nt} &= P_{nt,0}
\end{align}

When the inverter's output power exceeds the inverter's rated capacity (maximum AC power) $P_{ac} > P_{ac,0}$, SAM clips the inverter's output to the rated capacity and records the remaining power as the clipping loss:
%lib_sandia.cpp 354
\begin{align}
P_{ac,noclip} &= P_{ac} \notag\\
P_{ac} &= P_{ac,0} \notag\\
P_{clip} &= P_{ac,noclip} - P_{ac}
\end{align}

The inverter's DC-to-AC power conversion efficiency:
%lib_sandia.cpp 362
\begin{equation}
\eta_{inv} = \frac{P_{ac}}{P_{dc}}
\end{equation}

%%%%%%%%%%%%%%%%%%%%%%%%%%%%%%%%%%%%%%%%%%%%%%%%%%
%%%%%%%%%%%%%%%%%%%%%%%%%%%%%%%%%%%%%%%%%%%%%%%%%%
\section{Inverter Part Load Curve Model}\label{sec-partloadinverter}

NREL developed the Inverter Part Load Curve model specifically for SAM. The model's variables are listed in Tables~\ref{tab-invertervars} and \ref{tab-partloadinvertervars}.

\begin{table}
\begin{center}
\caption{Inverter Part Load Curve Model Inputs}
\begin{tabular}{lll}
\midrule
Symbol & Description / \textbf{Name in SAM} & Name in SSC \\
\midrule
$\eta_{\mathrm{inv},0-N}$ & column of $N$ efficiency values (\%)& \texttt{inv\_ds\_efficiency} \\
$P_{ac,0-N}$ & column of $N$ AC power output values (W)& \texttt{inv\_ds\_partload} \\
\hline
\end{tabular}
\label{tab-partloadinvertervars}
\end{center}
\end{table}

The part-load ratio for the current time step depends on the inverter's DC input power and its maximum rated DC power:
% lib_pvinv.cpp 34
\begin{equation}
x = \frac{P_{dc}}{P{dc,0}}
\end{equation}

The model then uses linear interpolation to calculate the inverter's DC-to-AC conversion efficiency $\eta_{inv}$ at the DC input power.

The inverter's AC power output for hours when $P_{dc}>0$:
% lib_pvinv.cpp 76, 93
\begin{equation}
P_{ac} = \left\{
\begin{array}{ll}
\eta_{inv} P_{dc} & \text{if $\eta_{inv} P_{dc} \leq P_{ac,0}$}\\
P_{ac,0} & \text{if $\eta_{inv} P_{dc} > P_{ac,0}$} 
\end{array}\right.
\end{equation}

For hours when the AC power is greater than the maximum rated value, the clipping loss value is:
% lib_pvinv.cpp 88
\begin{equation}
P_{clip} = P_{ac} - P_{ac,0}
\end{equation}

At night when $P_{dc} \leq 0$:
% lib_pvinv.cpp 79
\begin{align}
P_{ac} = -P_{nt,0} \notag\\
P_{nt} = P_{nt,0}
\end{align}

\section{Inverter Clipping Losses}\label{sec-invclip}

The photovoltaic performance model calculates two types of inverter clipping losses:

\begin{itemize}
\item MPPT clipping is a reduction in the inverter DC input power when the input voltage (DC string voltage) falls outside of the operating range defined by the inverter's minimum and maximum MPPT DC voltage ratings.
\item Power clipping is a reduction in the inverter output AC power caused by inverter saturation when the inverter power exceeds its nameplate rating.
\end{itemize}

When the DC string voltage is less than the inverter's minimum MPPT voltage rating, SAM sets the DC string voltage to the minimum rating. Similarly, when the DC string voltage is greater than the inverter's maximum MPPT voltage rating, SAM sets the DC string voltage to the maximum rating.

Power clipping is handled by the inverter model as described in the sections above for each inverter model.



%%%%%%%%%%%%%%%%%%%%%%%%%%%%%%%%%%%%%%%%%%%%%%%%%%
%%%%%%%%%%%%%%%%%%%%%%%%%%%%%%%%%%%%%%%%%%%%%%%%%%
\chapter{Battery Storage}\label{battery}

The battery storage model calculates power to and from a bank of lead-acid or lithium-ion batteries and estimates battery lifetime.

The battery storage model allows users to connect lead-acid and lithium-ion battery systems to their PV system for behind-the-meter applications.  In order to enable the battery model, a "Photovoltaic (detailed)" performance model must be selected with either "Residential (distributed)", "Commercial (distributed)", or "Third party ownership" financial models.  Under the "Battery Storage" page, "Enable Battery" must then be selected from the drop down menu near the top of the page.  If batteries are being considered as part of a detailed financial analysis, it is recommended that "PV simulation over analysis period" is selected on the "Lifetime" page.  Battery costs are entered on the "System Costs" page.

SAM models the connection by assuming that the battery system is connnected on the AC bus such that incoming AC power must be internally rectificed by the battery.  DC power output by the battery is internally inverted before being sent to the grid or load.  Single-point efficiencies allow the user to specify the AC-DC conversion loss and the DC-AC conversion loss.  Internal battery losses are applied due to lifetime degradation, thermal effects, and differences in charging and discharging voltages.

Customizable profiles allow specification of when the batteries can be charged from the grid and excess PV generation, and discharged to meet a custom portion of the electric load.  Users may enter detailed  inputs for voltage, capacity, lifetime, and thermal properties, or may use defaults provided by SAM.   Battery costs and replacement criteria may also be entered to consider the full lifetime costs of installing a battery system. A technical reference is available for a complete description of the battery model, including information about the validation process \citep{diorio2015a}. An economic case-study  is available as an example for how the battery model may be used to do detailed financial analysis of potential battery systems considering specific utility rate structures, battery costs, and dispatch criteria \citep{diorio2015b}.

\section{Battery Model Inputs and Outputs}\label{sec-batterymodel-inputs}
The battery model requires multiple inputs and provides multiple outputs.  Inputs are described in Table~\ref{tab-batteryinputs} and Table~\ref{tab-batteryinputs2} .  Battery outputs are described in Table~\ref{tab-batteryoutputs}.  Variable length outputs will be a single value if a lifetime simulation is not run, and will be an array of values for lifetime simulations, listed under the "Annual" data dropdown.  For simulations involving lead-acid batteries, two additional outputs are available, shown in Table~\ref{tab-batteryoutputs-lead}.  

\begin{table}
\begin{center}
\caption{Battery Inputs}
\begin{tabular}{lll}
\midrule
Description / \textbf{Name in SAM} & Name in SSC \\
\midrule
\multicolumn{2}{c}{Battery Bank Sizing Inputs}\\
\textbf{Desired bank capacity} (kWh) & \texttt{batt\_bank\_size} \\
\textbf{Desired bank voltage} (V) & \texttt{batt\_bank\_voltage} \\
\textbf{Total cells in series} & \texttt{batt\_bank\_ncells\_serial} \\
\textbf{Total number of strings} & \texttt{batt\_bank\_nstrings} \\
Specify desired bank size or specify cells & \texttt{batt\_size\_choice} \\
\midrule
\multicolumn{2}{c}{Battery Chemistry Inputs}\\
Lead-acid or lithium-ion chemistry type & \texttt{batt\_type} \\
\midrule
\multicolumn{2}{c}{Voltage Inputs}\\
\textbf{Cell nominal voltage} (V) & \texttt{batt\_Vnom\_default} \\
\textbf{Internal resistance} ($\Omega$) & \texttt{batt\_resistance} \\
\textbf{C-rate of discharge curve} & \texttt{batt\_C\_rate} \\
\textbf{Fully charged cell-voltage} (V) & \texttt{batt\_Vfull} \\
\textbf{Exponential zone cell voltage} (V) & \texttt{batt\_Vexp} \\
\textbf{Nominal zone cell voltage} (V) & \texttt{batt\_Vnom} \\
\midrule
\multicolumn{2}{c}{Current \& Capacity Inputs}\\
\textbf{Cell capacity} (Ah) & \texttt{batt\_Qfull} \\
\textbf{Max C-rate charge} & \texttt{batt\_C\_rate\_max\_charge} \\
\textbf{Max C-rate discharge} & \texttt{batt\_C\_rate\_max\_discharge} \\
\midrule
\multicolumn{2}{c}{Power Coverters Inputs}\\
\textbf{AC to DC conversion efficiency} (\%) & \texttt{batt\_ac\_dc\_efficiency} \\
\textbf{DC to AC conversion efficiency} (\%) & \texttt{batt\_dc\_ac\_efficiency} \\
\midrule
\multicolumn{2}{c}{Storage Dispatch Controller Inputs}\\
Manual or automated dispatch selection & \texttt{batt\_dispatch\_choice} \\
PV dispatch priority choice & \texttt{batt\_pv\_choice} \\
\textbf{Minimum state of charge} (\%) & \texttt{batt\_minimum\_SOC} \\
\textbf{Maximum state of charge} (\%) & \texttt{batt\_maximum\_SOC} \\
\textbf{Minimum time at charge state} (min) & \texttt{batt\_minimum\_modetime} \\
Target grid power input type (0/1) & \texttt{batt\_target\_choice}\\
Target grid power time series (kW) & \texttt{batt\_target\_power}\\
Target grid power single or monthly (kW) & \texttt{batt\_target\_power\_monthly}\\
\textbf{Charge from grid} (period \texttt{\textit{n}}) & \texttt{pv.storage.p[\textit{n}].gridcharge} \\
\textbf{Charge from PV} (period \texttt{\textit{n}}) & \texttt{pv.storage.p[\textit{n}].charge} \\
\textbf{Allow discharge} (period \texttt{\textit{n}}) & \texttt{pv.storage.p[\textit{n}].discharge} \\
\textbf{\% capacity to discharge (\%)} (period \texttt{\textit{n}}) & \texttt{batt\_discharge\_percent\_[\textit{n}].discharge} \\
\textbf{\% capacity to gridcharge} (\%) (period \texttt{\textit{n}}) & \texttt{batt\_gridcharge\_percent\_[\textit{n}].gridcharge} \\
Manual dispatch schedule & \texttt{dispatch\_manual\_sched} \\
\midrule
\multicolumn{2}{c}{Battery Lifetime Inputs}\\
Battery lifetime matrix & \texttt{batt\_lifetime\_matrix} \\
PV lifetime simulation & \texttt{pv\_lifetime\_simulation} \\
\hline
\end{tabular}
\label{tab-batteryinputs}
\end{center}
\end{table}

\begin{table}
\begin{center}
\caption{Battery Inputs Continued}
\begin{tabular}{lll}
\midrule
Description / \textbf{Name in SAM} & Name in SSC \\
\midrule
\multicolumn{2}{c}{Battery Bank Replacement Inputs}\\
Battery bank replacement option & \texttt{batt\_replacement\_option} \\
\textbf{Battery bank replacement cost} (\$) & \texttt{batt\_replacement\_cost} \\
\textbf{Battery cost escalation above inflation} (\%) & \texttt{batt\_replacement\_cost\_escal} \\
\textbf{Battery bank replacement threshold} & \texttt{batt\_replacement\_capacity} \\
\textbf{Battery bank replacement schedule} & \texttt{batt\_replacement\_schedule} \\

\midrule
\multicolumn{2}{c}{Thermal Behavior Inputs}\\
Battery specific heat capacity (J/kg/K) & \texttt{batt\_Cp} \\
Battery heat transfer coefficient (W/m2/K)& \texttt{batt\_h\_to\_ambient} \\
Battery room temperature (C) & \texttt{T\_room} \\
Battery capacity vs. temperature table & \texttt{cap\_vs\_temp} \\
\textbf{Specific energy per mass} (Wh/kg) & \texttt{batt\_specific\_energy\_per\_mass} \\
\textbf{Specific energy per volume} (Wh/L) & \texttt{batt\_specific\_energy\_per\_volume} \\
\hline
\end{tabular}
\label{tab-batteryinputs2}
\end{center}
\end{table}

\begin{table}
\begin{center}
\caption{Battery Outputs}
\begin{tabular}{lll}
\midrule
Description / \textbf{Name in SAM} & Name in SSC \\
\midrule
\multicolumn{2}{c}{Variable Length Outputs}\\
\textbf{Annual energy exported to grid} (kWh) & \texttt{annual\_export\_to\_grid\_energy} \\
\textbf{Annual energy imported from grid} (kWh) & \texttt{annual\_import\_from\_grid\_energy} \\
\textbf{Battery annual energy charged} (kWh) & \texttt{batt\_annual\_charge\_energy} \\
\textbf{Battery annual energy charged from grid} (kWh) & \texttt{batt\_annual\_charge\_from\_grid} \\
\textbf{Battery annual energy charged from pv} (kWh) & \texttt{batt\_annual\_charge\_from\_pv} \\
\textbf{Battery annual energy discharge} (kWh) & \texttt{batt\_annual\_discharge\_energy} \\
\textbf{Battery annual energy loss} (kWh) & \texttt{batt\_annual\_energy\_loss} \\
\textbf{Battery bank replacements per year} & \texttt{batt\_bank\_replacement} \\
\textbf{Battery replacement cost} (\$)& \texttt{cf\_batt\_bank\_replacement\_cost} \\
\textbf{Battery replacement cost schedule} & \texttt{cf\_batt\_replacement\_cost\_schedule} \\
\midrule
\multicolumn{2}{c}{Single Value Outputs}\\
\textbf{Average battery cycle efficiency} (\%)& \texttt{average\_cycle\_efficiency} \\
\textbf{Battery bank installed capacity} (kWh)& \texttt{batt\_bank\_installed\_capacity} \\
\textbf{Battery percent charged from PV} (\%)& \texttt{batt\_pv\_charge\_percent} \\
\midrule
\multicolumn{2}{c}{Monthly Value Outputs}\\
\textbf{Energy to load from PV} (kWh) & \texttt{monthly\_pv\_to\_load} \\
\textbf{Energy to load from battery} (kWh) & \texttt{monthly\_batt\_to\_load} \\
\textbf{Energy to load from grid} (kWh) & \texttt{monthly\_grid\_to\_load} \\
\midrule
\multicolumn{2}{c}{Lifetime Outputs at Simulation Timestep}\\
\textbf{Battery capacity percent for lifetime} (\%) & \texttt{batt\_capacity\_percent} \\
\textbf{Battery cycle depth of discharge} (\%) & \texttt{batt\_DOD} \\
\textbf{Battery number of cycles} & \texttt{batt\_cycles} \\
\textbf{Battery state of charge} (\%) & \texttt{batt\_SOC} \\
\textbf{Power of PV+ battery} (kW) & \texttt{pv\_batt\_gen} \\
\textbf{Power to battery from PV} (kW) & \texttt{pv\_to\_batt} \\
\textbf{Power to battery from grid}  (kW)& \texttt{grid\_to\_batt} \\
\textbf{Power to load from PV} (kW) & \texttt{pv\_to\_load} \\
\textbf{Power to load from battery} (kW) & \texttt{batt\_to\_load} \\
\textbf{Power to load from grid} (kW) & \texttt{grid\_to\_load} \\
\textbf{Power to/from battery} (kW) & \texttt{batt\_power} \\
\textbf{Power to/from grid} (kW) & \texttt{grid\_power} \\
\midrule
\multicolumn{2}{c}{Outputs at Simulation Timestep}\\
\textbf{Battery capacity percent for temperature} (\%) & \texttt{batt\_capacity\_thermal\_percent} \\
\textbf{Battery cell voltage} (V)& \texttt{batt\_voltage\_cell} \\
\textbf{Battery current} (A) & \texttt{batt\_I} \\
\textbf{Battery max charge} (Ah) & \texttt{batt\_qmax} \\
\textbf{Battery temperature} ($^\circ$C) & \texttt{batt\_temperature} \\
\textbf{Battery total charge} (Ah) & \texttt{batt\_q0} \\
\textbf{Battery voltage} (V) & \texttt{batt\_voltage} \\
\hline
\end{tabular}
\label{tab-batteryoutputs}
\end{center}
\end{table}

\begin{table}
\begin{center}
\caption{Battery Lead-Acid Specific Outputs}
\begin{tabular}{lll}
\midrule
Description / \textbf{Name in SAM} & Name in SSC \\
\midrule
\multicolumn{2}{c}{Variable Length Outputs}\\
\textbf{Battery available charge (Ah)} & \texttt{batt\_q1} \\
\textbf{Battery bound charge (Ah)} & \texttt{batt\_q2} \\
\hline
\end{tabular}
\label{tab-batteryoutputs-lead}
\end{center}
\end{table}

\section{Battery With No PV System}\label{sec-batterymodel-nopv}
There is often a desire to consider a standalone battery system.  To model this in SAM, begin a detailed Photovoltaic project with a Residential, Commercial, or Third-Party ownership financial model. On the losses page, within the "Curtailment and Availability" section, click on "Edit losses".  Make the "Contant loss (\%)" equal to 100\%.  This zeros out the contribution of PV production to the system analysis.  Next, it is important to zero out the PV system cost.  On the "System Costs" page, make the module "\$/Wdc"  and inverter "\$/Wdc fields equal to 0.  It is assumed that the cost of inverters required by the battery system to place power on the AC bus is included in the battery bank cost.

\section{DC Connected Battery}\label{sec-batterymodel-dc}
SAM currently assumes that the battery is placed on the AC bus, such that PV power has been inverted before being transferred to the battery.  Another popular configuration is to connect the battery to the DC side of the PV system such that DC/AC and AC/DC conversion losses can be avoided when charging the battery from excess PV generation.  To mimic a DC connection, the "AC to DC conversion efficiency" input can be increased, such that it is assumed no input conversion penalty is present.  Even in a DC connected system there is likely to be some loss through a charge controller.  The "DC to AC conversion efficiency" can be left alone, since power coming out of the battery will be inverted to place on the AC bus.  It is not currently possible to increase the PV system's DC-to-AC ratio and use excess DC power to charge the battery.

\section{Advanced Dispatch Options} \label{sec-batterymodel-dispatch}
SAM's manual dispatch controller is described in detail in \citep{diorio2015a}.  The controller as described in that report prioritizes PV energy to meet the electric load before charging the battery.  In the latest version of SAM, the user can choose to prioritize PV energy to charge the battery before meeting the electric load or meet the load before charging the battery.  This option is available for all dispatch models.  To set the control to meet the electric load with PV first, set \textbf{batt\_pv\_choice} to 0.  To charge the battery first, set \textbf{batt\_pv\_choice} to 1.

Several additional modes have since been added to provide more capabilities to the user.  Regardless of which dispatch model is selected, the minimum and maximum states of charge, the minimum time at charge state and the PV priority must be specified.  There are currently four available dispatch models, including the manual dispatch model.

Automated one-day look ahead and one-day look behind peak shaving controllers have been added to perform demand reduction. The one-day look ahead controller uses a perfect 24-hour prediction on PV production and electric load requirements to generate a dispatch strategy for the battery which reduces the grid power required as much as possible over the 24 hours within the state-charge and power limitations of the system.  The one-day look behind assumes that the last 24 hours are a good reflection of what to expect for the next 24 hours, such that yesterdays PV and electric load are used to generate a dispatch strategy for the battery.  This mode is meant as an illustration of a potentially more realistic controller that seeks to use past information to predict the future.  When either of these automated peak shaving modes are selected, the manual dispatch inputs will gray out since they are not required.  

There is also an automated grid power target model, which allows the user to input the maximum grid power limit. This limit can be input for every timestep or monthly.  To enter monthly information, \textbf{batt\_target\_choice} should be set to 0, and \textbf{batt\_target\_power\_monthly} should be passed in as a length 12 array.  The target power for each timestep within the month will be set to the monthly input.  To enter timeseries target power information, \textbf{batt\_target\_choice} should be set to 1, and \textbf{batt\_target\_power} should be passed in as an array with as many entries as that of the weather file.  The automated controller uses a 24-hour look ahead to predict the grid power required.  The battery is then dispatched to reduce all grid power events which exceed the target threshold for that timestep.  For all timesteps where the grid power is less than the threshold, the battery is allowed to charge from the grid until the power is equal to the threshold. 

%%%%%%%%%%%%%%%%%%%%%%%%%%%%%%%%%%%%%%%%%%%%%%%%%%
%%%%%%%%%%%%%%%%%%%%%%%%%%%%%%%%%%%%%%%%%%%%%%%%%%
\chapter{System AC Output}

The system AC output is the electricity generated by the photovoltaic system and may be delivered to any of the following, depending on the financial model that is coupled with the photovoltaic performance model:
\begin{itemize}
\item Electric power grid
\item Building or facility electric load
\item Electric storage battery
\end{itemize}
\begin{table}
\begin{center}
\caption{System AC Output Variable Definitions}
\begin{tabular}{lll}
\midrule
Symbol & Description / \textbf{Name in SAM} & Name in SSC \\
\midrule
\multicolumn{3}{c}{Inputs}\\
$N_{inv}$ & \textbf{Number of inverters} & \texttt{inverter\_count} \\
$P_{ac}$ & AC output of a single inverter (W)& - \\
$L_{ac}$ & \textbf{Total AC power loss} (\%)& \texttt{ac\_loss} \\
$L_{\textit{adjust}}$ & Curtailment and availability losses (\%)& \texttt{adjust} \\
\midrule
\multicolumn{3}{c}{Outputs}\\
$P_{\textit{gen}}$& \textbf{Power generated by system}  (W)& \texttt{gen} \\
\hline
\end{tabular}
\label{tab-systemacoutputvars}
\end{center}
\end{table}

\section{AC Losses}\label{sec-aclosses}

SAM models electrical losses on the AC side of the system using a single AC loss percentage.  In SAM, the model calculates the AC loss percentage from the AC wiring and step-up transformer loss percentages on the Losses input page:
\begin{align}\label{eqn-acderate}
L_{ac} &= 100~(1-F_1~F_2)\notag\\
F_{1} &= 1-\frac{L_{\textit{acwiring}}}{100} \notag\\
F_{2} &= 1-\frac{L_{\textit{transformer}}}{100}
\end{align}

In the SSC \texttt{pvsamv1} module, the AC loss $L_{ac}$ is the single value input \texttt{ac\_loss}. SSC applies the loss categories (AC wiring and step-up transformer losses) to the total AC loss to estimate the contribution of each category to the total loss. It does not calculate the value of the total loss from the AC wiring and transformer loss cateogories as SAM does. In order for the AC wiring loss and step-up transformer losses to be consistent with the total AC loss, you must calculate the value of the $L_{ac}$ input using Equation~\ref{eqn-acderate} and the values you assign to $L_{\textit{acwiring}}$ (\texttt{acwiring\_loss}) and $L_{\textit{transformer}}$ (\texttt{transformer\_loss}).


\section{Curtailment and Availability Losses}

Curtailment and availability losses may be used to account for operating losses imposed on the system by factors other than the solar resource and system's design, such as forced, scheduled, and unplanned outages, or other factors that reduce the system's AC power output.

SAM models curtailment and availability losses using an array of hourly loss values that apply to each time step of the simulation. In SAM, the losses are defined using the Edit Losses window, which is accessible from the Losses input page. 

In SSC, the losses may be defined in any of three ways:

\begin{itemize}
\item Constant loss (\texttt{adjust:constant}) is a single loss percentage that applies to all hours of the year.
\item Hourly losses (\texttt{adjust:hourly}) is an $8760\times1$ array of loss percentages that apply to each hour of the year.
\item Hourly losses with custom periods (\texttt{adjust:periods}) is an $n\times3$ matrix that defines losses over ranges of hours. For example this $2\times3$ array, $\texttt{adjust:perods=[[12,99,5][3146,3916,3]]}$, would apply a loss of 5\% to hours 12 through 99, and a loss of 3\% to hours 3146 through 3916.
\end{itemize}

\section{Power Generated by System} \label{sec-hourlyenergy}

The \textbf{Power generated by system} result variable is the electricity generated by the renewable energy system after all losses and adjustments. 

\begin{equation}
P_{gen}=  N_{inv}~P_{ac}~\left(1-\frac{L_{ac}}{100}\right)~\left(1-\frac{L_{\textit{adjust}}}{100}\right)
\end{equation}


For a photovoltaic project that uses one of the Utility IPP financial models, \textbf{Power generated by system} is the electricity delivered to the grid and sold at a negotiated power price. For projects with either the residential or commercial financing options it is the electricity generated by the photovoltaic system that may meet all or part of a building or facility's electric load, be sold to the grid, or charge a battery.

%%%%%%%%%%%%%%%%%%%%%%%%%%%%%%%%%%%%%%%%%%%%%%%%%%
%%%%%%%%%%%%%%%%%%%%%%%%%%%%%%%%%%%%%%%%%%%%%%%%%%
%%%%%%%%%%%%%%%%%%%%%%%%%%%%%%%%%%%%%%%%%%%%%%%%%%

%%%%%%%%%%%%%%%%%%%%%%%%%%%%%%%%%%%%%%%%%%%%%%%%%%
% REFERENCES

% bibliography
\cleardoublepage
\bibliographystyle{nrel}
%\bibintoc
\label{sec:Bib}
%\bibliography{files/bibliography}

%\bibliographystyle{plainnat}
\begin{thebibliography}{99}

\bibitem[Appelbaum(1979)]{appelbaum1979} Appelbaum, J.; Bany, J. ``Shadow effect of adjacent solar collectors in large scale systems." \textit{Solar Energy} (23) 1979; pp. 497-507.

\bibitem[Blair(2013)]{blair2013} Blair, N.; Dobos, A.; Gilman, P. (April 2013). ``Comparison of Photovoltaic Models in the System Advisor Model." Preprint. Prepared for American Solar Energy Society National Solar Conference Solar 2013, April 16-20, 2013. NREL/CP-6A20-58057. Golden, CO: National Renewable Energy Laboratory, 6 pp. Accessed February 27, 2014: \url{http://www.nrel.gov/docs/fy13osti/58057.pdf}

\bibitem[De Soto(2004a)]{desoto2004a} De Soto, W.; Klein, S.; Beckman, W. (2004) ``Improvement and Validation of a Model for Photovoltaic Array Performance." \textit{Solar Energy} (80:1); pp. 78-88.

\bibitem[De Soto(2004b)]{desoto2004b} De Soto, W. (2004). ``Improvement and Validation of a Model for Photovoltaic Array Performance." University of Wisconsin-Madison. Accessed February 26, 2014: \url{http://sel.me.wisc.edu/publications/theses/desoto04.zip}

\bibitem[Deline(2013)]{deline2013a} Deline, C.; Dobos, A.; Janzou, S.; Meydbrey, J.; Donoval, M. (2013). ``A Simplified Model of Uniform Shading in Large Photovoltaic Arrays." \textit{Solar Energy} (96); pp. 274-282. \url{http://www.sciencedirect.com/science/article/pii/S0038092X13002739}

\bibitem[Deline(2013a)]{deline2013b} Deline, C. (2013). ``SAM Shade Geometry V3." Unpublished. Golden, CO: National Renewable Energy Laboratory.

\bibitem[Dobos(2012a)]{dobos2012a} Dobos, A. (2012). ``An Improved Coefficient Calculator for the California Energy Commission 6 Parameter Photovoltaic Module Model." \textit{Journal of Solar Energy Engineering} (134:2).

\bibitem[Dobos(2012b)]{dobos2012b} Dobos, A. (June 2012). ``Modeling of Annual DC Energy Losses due to Off Maximum Power Point Operation in PV Arrays." Prepared for 38th IEEE Photovoltaic Specialists Conference (PVSC 2012). 3 pp. Accessed March 17, 2014: \url{http://ieeexplore.ieee.org/xpl/articleDetails.jsp?arnumber=6318207}

\bibitem[Dobos(2013a)]{dobos2013a} Dobos, A. (2013). ``PVWatts Version 1 Technical Reference." TP-6A20-60272. Golden, CO: National Renewable Energy Laboratory. Accessed February 20, 2014. \url{http://www.nrel.gov/docs/fy14osti/60272.pdf}

\bibitem[Dobos(2013b)]{dobos2013b} Dobos, A. (2013). ``5-Parameter PV Module Model." Presented at the 2013 Sandia PV Performance Modeling Workshop on May 1, 2013. Accessed March 4, 2014: \url{http://pvpmc.org/home/2013-pv-performance-modeling-workshop/}

\bibitem[Dobos(2014)]{dobos2014} Dobos, A.; MacAlpine, S. ``Procedure for Applying IEC-61853 Test Data to a Single Diode Model." Prepared for 40th IEEE Photovoltaic Specialist Conference (PVSC 2014). 4 pp. Accessed November 25, 2015: \url{http://ieeexplore.ieee.org/xpl/articleDetails.jsp?arnumber=6925525}

\bibitem[Duffie and Beckman(2013)]{duffie2013} Duffie, J.; Beckman, W. (2013). \textit{Solar Engineering of Thermal Processes, 4th ed}. New York, NY: Wiley.

\bibitem[Dunlap(2007)]{dunlap2007} Dunlap, J. (2007). \textit{Photovoltaic Systems}. Homewood, IL: American Technical Publishers.

\bibitem[EnergyPlus Weather(2014)]{epw} ``EnergyPlus Energy Simulation Software: Weather Data." U.S. Department of Energy. Accessed March 19, 2014: \url{http://apps1.eere.energy.gov/buildings/energyplus/weatherdata_about.cfm}

\bibitem[Freeman(2016)]{freeman2016} Freeman, J.; Freestate, D.; Hobbs, W.; Riley, C. (2016). ``Using Measured Plane-of-Array Data Directly in Photovoltaic Modeling: Methodology and Validation." [Poster] 43rd IEEE Photovoltaic Specialists Conference, 5-10 June 2016, Portland, Oregon. 1 pp. NREL/PO-6A20-66524. Accessed August 15, 2016: \url{http://www.nrel.gov/docs/fy16osti/66524.pdf}

\bibitem[Goswami(1989)]{goswami1989} Goswami, D.; Stefanakos, E.; Hassan A.; Collis,W. (1989). ``Effect of Row-to-Row Shading on the Output of Flat-Plate South-Facing Photovoltaic Arrays." \textit{Journal of Solar Energy Engineering} (111:3); pp. 257-259.

\bibitem[Go Solar California(2014b)]{gsc2014b} ``Incentive Eligible Photovoltaic Modules in Compliance with SB1 Guidelines." (2014). Go Solar California. Accessed March 4, 2014: \url{http://www.gosolarcalifornia.ca.gov/equipment/pv_modules.php}

\bibitem[Iqbal(1983)]{iqbal1983} Iqbal, M. (1983) \textit{An Introduction to Solar 
Radiation}. New York, NY: Academic Press.

 \bibitem[King(2004)]{king2004} King, D.; Boyson, W.; and Kratochvil, J. (2004). ``Photovoltaic Array Performance Model." 41 pp.; Albuquerque, NM: Sandia National Laboratories. SAND2004-3535. Accessed February 22, 2014: \url{http://prod.sandia.gov/techlib/access-control.cgi/2004/043535.pdf}

\bibitem[King(2007)]{king2007} King, D.; Gonzalez, S.; Galbraith, G.; Boyson, W. (2007). ``Performance Model for Grid-Connected Photovoltaic Inverters." 47 pp.; Albuquerque, NM: Sandia National Laboratories. SAND2007-5036. Accessed March 7, 2014: \url{http://prod.sandia.gov/techlib/access-control.cgi/2007/075036.pdf}

\bibitem[Liu(1963)]{liu1963} Liu, B.; Jordan, R. (1963). ``A Rational Procedure for Predicting The Long-term Average Performance of Flat-plate Solar-energy Collectors." \textit{Solar Energy} (7:2); pp. 53-74.

\bibitem[Michalsky(1988)]{michalsky1988} Michalsky, J. (1988). ``The Astronomical Almanac's Algorithm for Approximate Solar Position (1950-2050)." \textit{Solar Energy} (40:3); pp. 227-235.

\bibitem[Neises(2011)]{neises2011} Neises, T. (2011). ``Development and Validation of a Model to Predict the Temperature of a Photovoltaic Cell." University of Wisconsin-Madison. Accessed March 3, 2014: \url{http://sel.me.wisc.edu/publications/theses/neises11.zip}

\bibitem[Go Solar California(2014a)]{gsc2014a} ``New Solar Homes Partnership Calculator." (2014). Go Solar California. Accessed March 4, 2014: \url{http://www.gosolarcalifornia.ca.gov/tools/nshpcalculator/index.php}

\bibitem[Go Solar California(2014b)]{gsc2014b} ``Incentive Eligible Photovoltaic Modules in Compliance with SB1 Guidelines." (2014). Go Solar California. Accessed March 19, 2014:
 \url{http://www.gosolarcalifornia.ca.gov/equipment/pv_modules.php}

\bibitem[Go Solar California(2014c)]{gsc2014c} ``List of Eligible Inverters per SB1 Guidelines." (2014). Go Solar California. Accessed March 19, 2014:
 \url{http://www.gosolarcalifornia.ca.gov/equipment/pv_inverters.php}

\bibitem[Marion(2013)]{marion-snowmodel} Marion, W. (2013). ``Measured and modeled photovoltaic system energy losses from snow for Colorado and Wisconsin locations." Solar Energy 97 (2013): pp. 112-121.

\bibitem[NSRDB(2014)]{nsrdb} ``National Solar Radiation Database (NSRDB)." (2014). National Renewable Energy Laboratory.  Accessed November 13, 2013. \url{http://rredc.nrel.gov/solar/old_data/nsrdb/}.

\bibitem[NSRDB(2014a)]{tmy3} ``National Solar Radiation Database (NSRDB) 1991 - 2005 Update: Typical Meteorological Year 3." (2014). National Renewable Energy Laboratory. Accessed November 13, 2013. \url{http://rredc.nrel.gov/solar/old\_data/nsrdb/1991-2005/tmy3/}.

\bibitem[O'Brien(2012)]{stackoverflow2012} O'Brien, J. (2012).``Position of the sun given time of day, latitude and longitude." \textit{stackoverflow.com}. Accessed February 13, 2014. \url{http://stackoverflow.com/questions/8708048/position-of-the-sun-given-time-of-day-latitude-and-longitude}

\bibitem[PVCDROM(2014)]{pvcdrom} ``PVCDROM." (2014) pveducation.org. Accessed February 26, 2014. \url{http://www.pveducation.org/pvcdrom}

\bibitem [PVPMC Modeling Steps(2014)]{pvpmc}``PVPMC: Modeling Steps." (2014) PV Performance Modeling Collaborative. Accessed November 13, 2013. \url{http://pvpmc.org/modeling-steps/}.

\bibitem[PVPMC - Irradiance and Insolation(2014)]{pvpmc-irradinsol} ``PVPMC: Irradiance and Insolation." (2014). PV Performance Modeling Collaborative. Accessed November 14, 2013. \url{http://pvpmc.org/modeling-steps/irradiance-and-weather-2/irradiance-and-insolation/}

\bibitem[Passias(1984)]{passias1984} Passias, D.; Kallback, B.  (1984). ``Shading effects in rows of solar cell panels", Solar Cells (11); pp. 281-291.

\bibitem[Perez(1988)]{perez1988} Perez, R.; Stewart, R.; Seals, R.; Guertin, T. (1988) ``The Development and Verification of the Perez Diffuse Radiation Model." SAN88-7030. Albuquerque, NM: Sandia National Laboratories. Accessed February 20, 2014: \url{http://prod.sandia.gov/techlib/access-control.cgi/1988/887030.pdf}

\bibitem[Perez(1990)]{perez1990} Perez, R.; Ineichen, P.; Seals, R.; Michalsky, J.; Stewart, R. (1990) ``Modeling Daylight Availability and Irradiance Components from Direct and Global Irradiance." \textit{Solar Energy} (44:5); pp. 271-289.

\bibitem[Perez Sky Diffuse Model(2014)]{pvmcperez} ``Perez Sky Diffuse Model."  \textit{Modeling Steps}. PV Performance Modeling Collaborative. Albuquerque, NM: Sandia National Laboratories.  Accessed February 20, 2014: \url{http://pvpmc.org/modeling-steps/incident-irradiance/plane-of-array-poa-irradiance/calculating-poa-irradiance/poa-sky-diffuse/perez-sky-diffuse-model/}

\bibitem[Reindl(1988)]{reindl1988} Reindl, D. (1988). ``Estimating Diffuse Radiation on Horizontal Surfaces and Total Radiation on Tilted Surfaces." Masters Thesis. University of Wisconsin Madison, Solar Energy Laboratory. \url{http://sel.me.wisc.edu/publications/theses/reindl88.zip}

\bibitem[Ryberg(2015)]{ryberg-snowmodel} Ryberg, D.; Freeman, J. (2015). ``Integration, Validation and Application of a PV Snow Coverage Model in SAM." TP-6A20-64260. Golden, CO: National Renewable Energy Laboratory. Accessed June 1, 2016. \url{http://www.nrel.gov/docs/fy15osti/64260.pdf}

\bibitem[SAM Help - Performance Adjustment(2014)]{help-performanceadjustment} ``SAM Help: Performance Adjustment." (2014). SAM 2014.1.14 Help System. National Renewable Energy Laboratory. Available on line at \url{https://www.nrel.gov/analysis/sam/help/html-php/index.html?fin_annual_performance.htm} (accessed March 18, 2014).

\bibitem[SAM Help - Shading(2014)]{help-shading} ``SAM Help: Shading." (2014). SAM 2014.1.14 Help System. National Renewable Energy Laboratory. Available on line at \url{https://www.nrel.gov/analysis/sam/help/html-php/index.html?pv_shading.htm} (accessed February 26, 2014).

\bibitem[SAM Help - Sizing the Flate Plate PV System(2014)]{help-sizing} ``SAM Help: Sizing the Flat Plate PV System." (2014). SAM 2014.1.14 Help System. National Renewable Energy Laboratory. Available on line at \url{https://www.nrel.gov/analysis/sam/help/html-php/index.html?pvfp\_sizing\_the\_pv\_system.htm} (accessed March 20, 2014).

\bibitem[SAM Help - Weather File Formats(2014)]{help-weatherfileformats} ``SAM Help: Weather File Formats." (2014) SAM 2014.1.14 Help System. National Renewable Energy Laboratory. Available on line at  \url{https://www.nrel.gov/analysis/sam/help/html-php/index.html?weather_format.htm} (accessed November 13, 2013).

\bibitem[SAM SDK(2014)]{sdk} ``SAM Simulation Core SDK." (2014). National Renewable Energy Laboratory. Accessed March 19, 2014: \url{https://sam.nrel.gov/content/sam-simulation-core-sdk}

\bibitem[SAM Source Code(2016)]{source} ``SAM Source Code: Self-shading Geometry Calculations for One-axis Trackers." (2016). National Renewable Energy Laboratory. Accessed October 20, 2016: \url{https://sam.nrel.gov/source}

\bibitem[Sandia(2014)]{sandia-testeval} ``Sandia National Laboratories Photovoltaic Test and Evaluation Program." Sandia National Laboratories. Accessed March 7, 2014: \url{http://energy.sandia.gov/?page_id=279}

\bibitem[Solar Prospector(2015)]{solarprospector}``Solar Prospector." National Renewable Energy Laboratory. Accessed March 3, 2015: \url{http://maps.nrel.gov/prospector}

\bibitem[Van Schalwijk(1991)]{vanschalkwijk1991} Van Schalkwijk, M.; Kil, T.; Van der Weiden, J. (1991). ``Dependence of diffuse light blocking on the ground cover ratio for stationary PV arrays", Solar Energy (61:1); pp. 381-387.

\bibitem[Wilcox(2008)]{tmy3manual} Wilcox, S.; Marion, W. (2008). ``Users Manual for TMY3 Datasets." TP-581-43156. Golden, CO: National Renewable Energy Laboratory. Accessed November 13, 2013. \url{http://www.nrel.gov/docs/fy08osti/43156.pdf}.


\end{thebibliography}


\end{document}

